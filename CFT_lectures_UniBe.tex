\documentclass[a4paper,12pt]{article}

%\usepackage[left=25mm, right=25mm, top=30mm, bottom=30mm]{geometry}
\usepackage[utf8]{inputenc}
\usepackage[english]{babel}
\usepackage{amsmath}
\usepackage{amsfonts}
\usepackage{amssymb}
\usepackage{mathrsfs}
\usepackage{dsfont}
\usepackage{graphicx}
\usepackage{cite}
\usepackage{xcolor}
\usepackage{mdframed}
\usepackage[colorlinks, allcolors=blue, linktocpage=true]{hyperref}

\newcommand{\ket}[1]{\left| #1 \right\rangle}
\newcommand{\bra}[1]{\left\langle #1 \right|}


% double angle-bracket notation:
%\makeatletter
%\newsavebox{\@brx}
%\newcommand{\llangle}[1][]{\savebox{\@brx}{\(\m@th{#1\langle}\)}%
%  \mathopen{\copy\@brx\kern-0.5\wd\@brx\usebox{\@brx}}}
%\newcommand{\rrangle}[1][]{\savebox{\@brx}{\(\m@th{#1\rangle}\)}%
%  \mathclose{\copy\@brx\kern-0.5\wd\@brx\usebox{\@brx}}}
%\makeatother

\numberwithin{equation}{section}

\newcounter{exercise}[section]
\newenvironment{exercise}[1][]%
	{\refstepcounter{exercise}\bigskip
	\begin{mdframed}[backgroundcolor=gray!20, linewidth=0]
	\noindent\textbf{Exercise~\thesection.\theexercise #1} \rmfamily}
  	{\end{mdframed}\bigskip}
\newcommand\hint[1]{\emph{Hint: #1}}


%%%%%%%%%%%%%%%%%%%%%%%%%%%%%%%%%%%%%%%%%%%%%%%%%%%%%%%%%%%%%%%%%%%%%%%

\title{%
Conformal Field Theory
\\[1em]
\Large
Lecture notes}

\author{Marc Gillioz}

\date{Spring semester 2022}


\begin{document} 

\maketitle

\vspace{10mm}	
	
This is the transcript of my notes for the class 
``\href{https://www.itp.unibe.ch/studies/graduate_program/f22_conformal_field_theory/index_eng.html}{Conformal Field Theory}'' at the Institute for Theoretical Physics of the University of Bern.
These notes will be updated as the class advances, with the latest version always available at:
\begin{center}
	\url{https://github.com/gillioz/CFT_course}
\end{center}
Suggestions for improvements are welcome and can be directly submitted on the GitHub page, or sent to me by email at \href{mailto:marc.gillioz@sissa.it}{marc.gillioz@sissa.it}.
	
%\parbox{\textwidth}{%
%Conformal field theory (CFT) is an ubiquitous subject in modern theoretical physics. Every local quantum field theory approaches a CFT in the large- and small-distance limits, and even the study of quantum gravity is related to it through the AdS/CFT correspondence. CFT is also one of the rare frameworks in which quantum field theory can be studied outside the realm of perturbation theory.

%This is an introductory course in which the students will learn what is conformal field theory, why it is special, and get a glimpse of modern developments. The course will begin with a non-perturbative formulation of quantum field theory (Wightman functions, spectral representation), and then gradually focus on understanding the implications of scale and special conformal symmetry. The study of practical tools (embedding space formalism, radial quantization, state-operator correspondence, conformal blocks) will finally lead to the formulation of the modern conformal bootstrap and a review of its most recent results. Some advanced topics will be discussed depending on the students' interests (Virasoro symmetry in 2 dimensions, UV and IR divergences, Mellin representation, superconformal symmetry).}



%%%%%%%%%%%%%%%%%%%%%%%%%%%%%%%%%%%%%%%%%%%%%%%%%%%%%%%%%%%%%%%%%%%%%%%

\newpage

\tableofcontents

%%%%%%%%%%%%%%%%%%%%%%%%%%%%%%%%%%%%%%%%%%%%%%%%%%%%%%%%%%%%%%%%%%%%%%%

\newpage
\section{Introduction}


\subsection{What is conformal field theory?}

Conformal field theory is a quantum field theory with conformal symmetry:
\begin{equation}
	\text{conformal field theory}
	= \text{quantum field theory} + \text{conformal symmetry},
\end{equation}
where under conformal symmetry we understand
\begin{itemize}

\item
translations in time and space,

\item
rotations and Lorentz boosts,

\item
scale transformations, also known as dilatations,

\item
special conformal transformations.

\end{itemize}
The course will begin in section~\ref{sec:classical} with a precise definition of all these symmetries, even though the first two (the Poincaré symmetry) should be familiar from an elementary quantum field theory class, and you should even have heard of the importance of dilatations in the context of the \emph{renormalization group}.

Most courses on CFT put the emphasis on conformal symmetry as a whole, which makes a lot of sense because it forms a group (more on this latter). However, this approach often requires to work in Euclidean space (we will see why), and the connection with quantum field theory as we know it appears very late in the course, if at all.

Instead, in this lecture we would like to stay as close as possible to quantum field theory; in some sense, we will defined conformal field theory as
\begin{equation}
\begin{aligned}
	\text{conformal field theory}
	&= \text{relativistic quantum field theory} 
	\\
	& \quad
	+ \text{scale symmetry}
	\\
	& \quad
	+ \text{special conformal symmetry},
\end{aligned}
\end{equation}
and go through this definition carefully.
The first element is quantum field theory in flat Minkowski space-time. We have a good intuition of this part from particle physics. However, we do not want to restrict our attention to theories that have a nice classical limit and an understanding in terms of perturbation theory!
Therefore, we will discuss in section~\ref{sec:quantum} the basics of a non-perturbative quantum field theory.
Even though we care about unitary quantum field theory in Minkowski space-time, it will be useful to establish a connection with the same theory defined in flat Euclidean space. For this reason, we will work with the Minkowski metric convention
\begin{equation}
	\eta_{\mu\nu} = \left( \begin{array}{cccc}
		-1 &&& \\ & 1 && \\ && \ddots & \\ &&& 1
	\end{array} \right)_{\mu\nu},
	\qquad\quad
	\mu, \nu = 0, 1, \ldots, d-1,
\end{equation}
so that going from Minkowski space-time to Euclidean space will be achieved through a rotation of the time coordinate in the complex plane, $x^0 = t \to i \tau$.
Since for us the space-time will always be flat (no matter whether Minkowski or Euclidean), we will raise and lower the indices at will.

After giving a non-perturbative definition of relativistic quantum field theory, we will examine the role of scale symmetry. The main novelty here is that scale invariance forbids the presence of particles with a definite mass; instead, in a scale-invariant theory there are excitations of \emph{any} energy (unless we are in the very special case of a free theory).

Finally, special conformal symmetry will be discussed last. The point is that it (nearly always) come along with scale symmetry. Nevertheless, it provides extremely powerful tools to understand the theory from a non-perturbative point of view. This will lead us to a comprehensive study of conformally-invariant correlation functions in section~\ref{sec:correlators}, and to the formulation of an \emph{operator product expansion} (OPE) that has very nice features in section~\ref{sec:OPE}.

%- UV/IR divergences and anomalies 
%these are precisely subjects that will be covered in these lectures


\subsection{Why is conformal field theory interesting?}

Nature is certainly not scale invariant, so why do we care so much about conformal field theory?

First of all, scale transformations play an essential role in quantum field theory, even when this is not a symmetry of the theory. The Callan-Symanzik equation in renormalization is essentially a Ward identity associated with the (broken) scale symmetry. 
We will see that conformal transformations can be viewed as a \emph{local} generalization of scale transformations, in which the rescaling factor is different at every point in space-time, while still being consistent with Poincaré symmetry.

We also know that all end points of renormalization group trajectories are described by scale-invariant fixed points, and it turns out that \emph{all} known fixed points are in fact conformally invariant. Therefore, conformal field theory is automatically relevant for the description of the low- and high-energy limits of any quantum field theory.

But one of the main reasons why it is important to study conformal field theory now (or at least have a basic understanding of it) is that it has been yielding exciting results in the last 10-15 years. In section~\ref{sec:bootstrap}, we will give a short review of the  \emph{conformal bootstrap}, a technique that allows to completely solve conformal field theory in certain cases.
Even though the modern conformal bootstrap was invented to solve a particle physics problem, its results so far have had more impact in condensed matter physics. There, the concept of \emph{universality} is playing a central role: it turns out that the critical exponents of very different systems coincide. This has a simple explanation from the fact that there are only ``few'' conformal field theories that can exist at all (at least once global symmetries are specified together with some information about what are the relevant operators). In other words, the space of all conformal field theories is sparse, unlike in QFT where it is always possible to add some new operator in a Lagrangian to slightly deform the theory.
This concept is extremely successful in $d = 2$ and 3 dimensions, but there is also a big hope to classify all possible conformal field theories in $d = 4$ (or higher) dimensions. 
This goal would have a huge impact also for QFT with scale dependence, because it would strongly constrain the possible renormalization group flow.
But this is still a young field, and much more work needs to be done before anything like this can be claimed.

Besides its importance in quantum field theory and condensed matter physics, conformal field theory has also given very nice insights into:
\begin{itemize}

\item
Quantum gravity, through the AdS/CFT correspondence.

\item
String theory, as the worldsheet action is a CFT in $d = 2$.

\item
Foundations of quantum field theory beyond the perturbative approach.
CFT admits a mathematically rigorous definition, which does not require to ``quantize'' a classical Lagrangian theory.%
%
\footnote{In fact there needs not be a classical limit; for instance, in $d = 6$ dimensions there exists a theory with $(2,0)$ extended supersymmetry for which no Lagrangian can be written down. The theory is known to exists from string theory arguments, and it has also been ``observed'' in a conformal bootstrap setup.}
%

\end{itemize}


\subsection{Examples of conformal field theories}

Conformal field theories are everywhere in physics, and not necessarily limited to very special situations. Here are some examples, some of which you may already have encountered:%
%
\begin{itemize}

\item
Free massless theories are conformal. This is the case of the free boson and free fermion in any number of dimensions, but also of the $n$-form gauge theory in $d = 2n + 2$ dimensions (e.g.~the free vector theory in $d = 4$). The same is true of theories with any number of free fields.

\item
Theories in which the $\beta$-function admits a perturbative fixed points. Typical examples are deformations of the free scalar theory with a potential of the type $\phi^n$. For instance, consider
\begin{equation}
	S = \int d^dx \left[
	- \frac{1}{2} \partial_\mu \phi \partial^\mu \phi
	- \frac{g}{4!} \phi^4 \right].
\end{equation}
$g$ is dimensionless in $d = 4$ dimensions. If one considers the theory defined in $d = 4 - \varepsilon$ dimensions, then the beta function for $g$ is
\begin{equation}
	\beta_g = \mu \frac{dg}{d\mu}
	= - \varepsilon + \frac{3g}{(4\pi)^2}
	+ \mathcal{O}(g^2).
\end{equation}
It vanishes when
\begin{equation}
	\frac{g_*}{(4\pi)^2} = \frac{\varepsilon}{3}.
\end{equation}
Corrections of higher order in $g$ can be neglected in the limit $\varepsilon \ll 1$.
There are similar fixed points for a $\phi^3$ interaction in $d = 6 + \varepsilon$ dimensions, of for a $\phi^6$ interaction around $d = 3$.


\item
A similar type of fixed point can be found in the beta function of a $SU(N_c)$ gauge theory with $N_f$ fermions (in the fundamental representation), which is given at leading order in the gauge coupling $\alpha = g^2/(4\pi)^2$ by
\begin{equation}
	\beta_\alpha = \mu \frac{d\alpha}{d\mu}
	= -\frac{2}{3} \alpha^2 \left(11 N_c - 2 N_f \right)
	+ \mathcal{O}(\alpha^3).
\end{equation}
When $N_f = \frac{11}{2} N_c$, the leading order term in this beta function vanishes, so the next-to-leading term becomes important. Around that value, the first two terms are of similar importance, and one finds a perturbative fixed point of when $N_f \lesssim \frac{11}{2} N_c$. This is called the (Caswell-)Banks-Zaks fixed point. A theory in this situation is asymptotically free like QCD, but in the low-energy limit it approaches an interacting conformal field theory.
For an $SU(3)$ gauge theory like QCD this critical value is at $N_f = 16.5$. There is strong evidence from lattice simulation that a theory with $N_f = 16$ is conformal. On the other hand a theory with low $N_f$ (QCD has only 3 light quarks) is clearly confining, meaning that its low-energy limit is a theory massless Goldstone bosons (if the quarks are massless), or of massive pions. There is a critical value $N_f^* \approx 12$ above which we expect an interacting conformal field theory in the low-energy limit. The domain $N_f^* \leq N_f \leq \frac{11}{2} N_c$ is called the \emph{conformal window}. Note that gauge theories with different gauge groups and/or fermions coupling differently to the gauge fields (i.e.~in different representations) can also have a conformal window. It is even possible to engineer gauge theories with perturbative UV fixed points using not only fermions but also scalars.

\item
There are theories with extended supersymmetry in which the beta function is exactly zero at all orders in perturbation theory, for instance $\mathcal{N} = 4$ supersymmetric Yang-Mills. More generically, in theories with sufficiently many supersymmetries, it is often sufficient to engineer the matter content to make the beta function zero at leading order, and then non-renormalization theorems ensure their vanishing at all orders. Theories with conformal symmetry and supersymmetry are called \emph{superconformal}.

\item
An example of truly non-perturbative fixed point in $d = 3$ dimensions is given by the following action:
\begin{equation}
	S = \int d^3x \left[
	-\frac{1}{2} \partial_\mu \phi \partial^\mu \phi
	- \frac{1}{2} m^2 \phi^2 - \frac{g}{4!} \phi^4 \right],
\end{equation}
in which $g > 0$ for the potential to be bounded below.
Note that the free scalar field has mass dimension $\left[ \phi \right] = \frac{1}{2}$ in $d = 3$, and therefore
\begin{equation}
	\left[ m^2 \right] = 2,
	\qquad\qquad
	\left[ g \right] = 1.
\end{equation}
Since both $m^2$ and $g$ have positive mass dimensions, they are \emph{relevant} operators: they determine the dynamics in the low-energy limit (IR), but their importance decreases at higher and higher energies (UV): at energies $E \gg g, |m|$, this is approaches the theory of a massless free scalar.
On the other hand, at low energies the physics depends obvioulsy on $g$ and $m$: when $m^2 \gg g^2$, it is a theory of a massive scalar of mass $m$, whereas when $m^2 \ll -g^2$, then the potential has two minima at
\begin{equation}
	\langle \phi \rangle = \pm \sqrt{- \frac{6m^2}{g}},
\end{equation}
with excitations fo mass $\sqrt{2} |m|$ around it. Clearly this theory has two phases, and there must therefore be an intermediate value of $m^2$ where the phase transition happens (or working in units set by $g$, a specific value of the dimensionless ratio $m^2/g^2$). Note that this theory has a $\mathds{Z}_2$ symmetry in the UV corresponding to $\phi \to -\phi$, which is spontaneously broken in one phase and not in the other.

It turns out that the theory describing the IR physics exactly at the phase transition is a conformal field theory. Unlike the two phases surrounding it, it admits excitations of arbitrarily small energies (but they are not particles). How do we know that? Feynman diagram computations cannot be trusted in the IR: the approximation given by the (asymptotic) perturbative series is valid in the UV, but it breaks down in the IR, as in QCD. One way of understanding the phase transition is examining the theory in $d \neq 3$ dimensions: the same theory in $d = 4 - \varepsilon$ expansion has a perturbative fixed point, and computations at the fixed point can be performed in an expansion in $\varepsilon$. The value of the fixed point depends on the renormalization scheme, but there are other quantities that are scheme-independent. This is for instance the case of the anomalous dimension of the operator $\phi$.
Using a 6-loop beta-function computation, one finds
\begin{equation}
	\gamma_\phi \approx 0.0182.
\end{equation}
This is more conveniently expressed as a \emph{scaling dimension} of $\phi$,
\begin{equation}
	\Delta_\phi = \frac{d-2}{2} + \gamma_\phi \approx 0.5182,
\end{equation}
which in position space describes how the correlation between two points decays with the distance
\begin{equation}
	\langle \phi(x) \phi(0) \rangle \propto |x|^{-2\Delta_\phi}.
\end{equation}

A surprising thing about this scaling dimension is that it coincides preciesely with that of a statistical physics model. The \emph{Ising model} is a theory of classical spins $s_i = \pm 1$ on a lattice with nearest-neighbor interactions, characterized by the Hamiltonian
\begin{equation}
	H = - J \sum_{\langle ij \rangle} s_i s_j.
\end{equation}
This model has a critical value of $J$ whose continuum limit is described by a CFT. At this value, Monte-Carlo simulations show that the correlation between two spins decay with the distance with power given by
\begin{equation}
	\Delta_s \approx 0.5181.
\end{equation}
The two theories have completely different microscopic descriptions: one is a quantum field theory describing particles in Minkowski space-time, the other a simple theory on a Euclidean lattice. 
Even more surprisingly, the same critical exponent is found in other systems, such as the critical point of water and other liquids. This is an example of \emph{universality}.

The explanation for this coincidence is that there are not many candidate conformal field theories to describe the phase transition of the $\phi^4$ theory and the Ising model. In fact, these two theories have in common:
\begin{itemize}

\item
A global $\mathbb{Z}_2$ symmetry ($s_i \to - s_i$) that is broken in one phase and unbroken in the other;

\item
Exactly 2 relevant operators 
(in addition to $J$ the Ising model can be parametrized in terms of a coupling to an external magnetic field $\delta H = - \mu \sum s_i$).

\end{itemize}
%
The conformal bootstrap philosophy is quite different from the two examples above, in the sense that it does not care about the microscopic details of the theory: instead, it attempts to solve the conformal field theory that is expected to be there at the fixed point, based on symmetry arguments only. 
We shall see that the conformal bootstrap can be used to show that there exists a unique conformal field theory with 2 relevant operators and a $\mathds{Z}_2$ symmetry, and that the scaling dimension of its leading operator is equal to
\begin{equation}
	\Delta_s \approx 0.5181489.
\end{equation}
In fact, the conformal bootstrap does not only establish that, but it gives access to the whole spectrum of operators in that conformal field theory.
This is truly a success story of the bootstrap.

\item
Finally, another example of CFT is the family of theories obtained by considering a quantum field theory in $d+1$-dimensional anti de Sitter (AdS) space-time. AdS admits a compactification to a sphere with a boundary, and correlation functions on that boundary are isomorphic with that of a $d$-dimensional conformal field theory.
A generic (e.g.~perturbative) theory in AdS gives rise to a CFT on the boundary that does not satisfy all the ordinary assumptions of a quantum field theory (for instance it does not have an energy-momentum tensor), but there is evidence that a theory that includes gravity does.
Note that the same type of correspondence applies to late-time correlators in de Sitter space-time.

\end{itemize}


\subsection{Literature}

There a several excellent modern reviews on conformal field theory, and large parts of these lecture notes are directly inspired by them.
The essentials are covered in the works of conformal bootstrap experts:
\begin{itemize}

\item
Slava Rychkov's EPFL lectures
\cite{Rychkov:2016iqz};
this review also contains interesting historical comments.

\item
David Simmons-Duffin's TASI lectures
\cite{Simmons-Duffin:2016gjk}.

\item
Shai Chester's Weizmann Lectures
\cite{Chester:2019wfx}.

\end{itemize}
%
Some other useful material can be found in:
%
\begin{itemize}

\item
A review article on the state-of-the-art bootstrap methods~\cite{Poland:2018epd}.

\item
Hugh Osborn's lecture notes at Cambridge, available at: \\ 
\url{https://www.damtp.cam.ac.uk/user/ho/CFTNotes.pdf}

\item
Joshua Qualls' lectures \cite{Qualls:2015qjb}.

\end{itemize}
%
The reviews mentioned above are all mostly focused on the Euclidean approach to CFT. For a Lorentzian perspective, see:
\begin{itemize}

\item
Slava Rychkov's \emph{Lorentzian methods in conformal field theory}, available at \url{https://courses.ipht.fr/node/226}
(\href{https://www.ipht.fr/Docspht/articles/t19/229/public/Lecture-notes-Rychkov-IPHT.pdf}{link} to the lecture notes)

\end{itemize}
%
There are also modern reviews specifically on the topic of CFT in $d = 2$ dimensions:
\begin{itemize}

\item
Sylvain Ribault's notes~\cite{Ribault:2014hia}, 
also accepting suggestions on GitHub: \\ 
\url{https://github.com/ribault/CFT-Review}

\item
Schellekens ``Conformal Field Theory'' lecture notes~\cite{Schellekens:1996tg}, and a recent version available at:
\url{https://www.nikhef.nl/\textasciitilde t58/CFT.pdf}.

\end{itemize}
More classical literature on the subject can be found in:
\begin{itemize}

\item
The ``yellow book'' by Di Francesco, Mathieu, and Sénéchal
\cite{DiFrancesco:1997nk}.

\item
Polchinski's \emph{String theory} vol.~1
\cite{Polchinski:1998rq}.

\item
Lectures by Paul Ginsparg~\cite{Ginsparg:1988ui}.
%emphasis on statistical physics and string theory applications

\end{itemize}
%
For theories combining supersymmetry and conformal symmetry:
\begin{itemize}

\item
Lorenz Eberhardt's lecture notes 
\cite{Eberhardt:2020cxo}.

\end{itemize}

%%%%%%%%%%%%%%%%%%%%%%%%%%%%%%%%%%%%%%%%%%%%%%%%%%%%%%%%%%%%%%%%%%%%%%

\section{Classical conformal transformations}
\label{sec:classical}

One of the most fundamental principles of physics is independence of the reference frame: observers living at different points might have different perspectives, but the underlying physical laws are the same.
This is true in space (invariance under translations and rotations), but also in space-time (e.g.~invariance under Lorentz boosts).

\subsection{Infinitesimal transformations}

In mathematical language, this means that if we have a coordinate system $x^\mu$, the laws of physics do not change under a transformations 
\begin{equation}
	x^\mu \to x'^\mu.
	\label{eq:diffeo}
\end{equation}
This principle applies to all maps that are invertible (isomorphisms) and differentiable (smooth transformations), hence it is usually called \emph{diffeomorphism} invariance.
Being differentiable, the transformation \eqref{eq:diffeo} can be Taylor-expanded to write
\begin{equation}
	x^\mu \to x'^\mu = x^\mu + \varepsilon^\mu(x),
	\label{eq:diffeo:infinitesimal}
\end{equation}
in terms of an infinitesimal vector $\varepsilon^\mu$ (meaning that we will always ignore terms of order $\varepsilon^2$).

In addition to the coordinate system, the description of a physical system requires a way of measuring distances that is provided by a metric $g_{\mu\nu}(x)$. Distances are measured integrating the line element
\begin{equation}
	ds^2 = g_{\mu\nu}(x) dx^\mu dx^\nu.
\end{equation}
Since all observers should agree on the measure of distances, we must have
\begin{equation}
	g'_{\mu\nu}(x') dx'^\mu dx'^\nu = g_{\mu\nu}(x) dx^\mu dx^\nu,
\end{equation}
Here $g_{\mu\nu}$ could be the Euclidean metric $\delta_{\mu\nu}$ or the Minkowski metric $\eta_{\mu\nu}$; for simplicity we only consider the case in which $g_{\mu\nu}$ is flat, i.e.~$\partial_\alpha g_{\mu\nu} = 0$.
In this case, we can write
\begin{equation}
\begin{aligned}
	g'_{\mu\nu} &= g_{\alpha\beta}
	\frac{\partial x^\alpha}{\partial x'^\mu}
	\frac{\partial x^\beta}{\partial x'^\nu}
	\\
	&= g_{\alpha\beta}
	\left( \delta^\alpha_\mu - \partial_\mu \varepsilon^\alpha \right)
	\left( \delta^\beta_\nu - \partial_\nu \varepsilon^\beta \right)
	\\
	&= g_{\mu\nu}
	- \left( \partial_\mu \varepsilon_\nu
	+ \partial_\nu \varepsilon_\mu \right).
\end{aligned}
\end{equation}
If we require the different observers to also agree on the metric, then  we must have $g'_{\mu\nu} = g_{\mu\nu}$, which gives a constraint on what kind of coordinate transformations are possible: we must have
\begin{equation}
	\partial_\mu \varepsilon_\nu
	+ \partial_\nu \varepsilon_\mu = 0.
\end{equation}
This condition admits as a most general solution
\begin{equation}
	\varepsilon^\mu = a^\mu + \omega^\mu_{~\nu} x^\nu,
\end{equation}
where $a^\mu$ is a constant vector and $g_{\mu\rho} \omega^\rho_{~\nu} = \omega_{\mu\nu}$ an antisymmetric tensor.
The transformation
\begin{equation}
	x^\mu \xrightarrow{P} x^\mu + a^\mu
\end{equation}
is obviously a translation and
\begin{equation}
	x^\mu \xrightarrow{M} \omega^\mu_{~\nu} x^\nu
\end{equation}
a rotation/Lorentz transformation around the origin $x = 0$. The composition of these two operations generates the Poincaré group.
This is the fundamental symmetry of space-time underlying all relativistic quantum field theory. It is a symmetry of nature to a very good approximation, at least up to energy scales in which quantum gravity becomes important.

However, one can also consider the situation in which the two observers use different systems of units, i.e.~they disagree on the overall definition of scale, but agree otherwise on the metric being flat.
In this case we must have $g'_{\mu\nu} \propto g_{\mu\nu}$, and therefore the constraint becomes
\begin{equation}
	\partial_\mu \varepsilon_\nu
	+ \partial_\nu \varepsilon_\mu = 2 \lambda g_{\mu\nu},
\end{equation}
for some real number $\lambda$,
with the most general solution
\begin{equation}
	\varepsilon^\mu = a^\mu + \omega^\mu_{~\nu} x^\nu
	+ \lambda x^\mu.
\end{equation}
The new infinitesimal transformation is
\begin{equation}
	x^\mu \xrightarrow{D} (1 + \lambda) x^\nu.
\end{equation}
It is a scale transformation. Note that scale symmetry is not a good symmetry of nature: there is in fact a fundamental energy scale on which all observer must agree (this can be for instance chosen to be the mass of the electron).
Nevertheless, there are some systems in which this is a very good approximate symmetry. One can also make very interesting \emph{Gedankenexperimente} that have scale symmetry built in. These are reasons that make it worth studying.

If one pushes this logic further, in a scale-invariant world in which observers have no physical mean of agreeing on a fundamental scale, they might even decide to change their definition of scale as they walk around. This would correspond to the case in which the metric $g'_{\mu\nu}$ of one observer can differ from the original metric $g_{\mu\nu}$ by a function of space(-time):
\begin{equation}
	g'_{\mu\nu}(x) = \Omega(x) g_{\mu\nu}.
\end{equation}
Note that we are not saying that $g'_{\mu\nu}$ is completely arbitrary: at every point in space time it is related to the flat metric by a scale transformation. But the scale factor is different at every point.
The condition on $\varepsilon^\mu$ becomes in this case
\begin{equation}
	\partial_\mu \varepsilon_\nu
	+ \partial_\nu \varepsilon_\mu = 2 \sigma g_{\mu\nu},
	\label{eq:conformalkillingeq}
\end{equation}
where $\Omega(x) = 1 + 2\sigma(x)$. To find the most general solution to this equation, note that contracting the indices with $g^{\mu\nu}$ gives
\begin{equation}
	\partial_\mu \varepsilon^\mu  = d \sigma,
\end{equation}
where $d$ is the space(-time) dimension,
while acting with $\partial^\nu$ gives
\begin{equation}
	\partial_\mu \partial_\nu \varepsilon^\nu
	+ \partial^2 \varepsilon_\mu
	= 2 \partial_\mu\sigma,
\end{equation}
so that we get
\begin{equation}
	\partial^2 \varepsilon_\mu
	= (2 - d) \partial_\mu\sigma.
\end{equation}
Acting once again with $\partial^\mu$, we arrive at 
\begin{equation}
	(d - 1) \partial^2 \sigma = 0,
\end{equation}
while acting with $\partial^\nu$ and symmetrizing the indices, we find
\begin{equation}
	(2 - d) \partial_\mu\partial_\nu \sigma
	= g_{\mu\nu} \partial^2 \sigma
	= 0.
\end{equation}
In $d > 2$, we obtain therefore the condition $\partial_\mu \partial_\nu \sigma = 0$, which is solved by 
\begin{equation}
	\sigma(x) = \lambda + 2 b \cdot x.
\end{equation}
We have therefore
\begin{equation}
	\varepsilon^\mu = a^\mu + \omega^\mu_{~\nu} x^\nu
	+ \lambda x^\mu
	+ 2 (b \cdot x) x^\mu - x^2 b^\mu.
	\label{eq:conformalkillingvec}
\end{equation}
In addition to the transformations found before, we also find
\begin{equation}
	x^\mu \xrightarrow{K} 2 (b \cdot x) x^\mu - x^2 b^\mu,
\end{equation}
which is a \emph{special conformal transformation}.
If we examine the Jacobian for this transformation, we find
\begin{equation}
	\frac{\partial x'^\mu}{\partial x^\nu}
	= \left(1 + 2 b \cdot x \right) \delta^\mu_\nu
	+ 2 \left( b_\nu x^\mu - x_\nu b^\mu \right)
	\approx \left(1 + 2 b \cdot x \right)
	R^\mu_{~\nu}(x).
\end{equation}
We have written this as a position-dependent scale factor $(1 + 2 b \cdot x)$, multiplying an orthogonal matrix
\begin{equation}
	R^\mu_{~\nu}(x) = \delta^\mu_\nu
	+ 2 \left( b_\nu x^\mu - x_\nu b^\mu \right).
\end{equation}
This shows that special conformal transformation act locally as the composition of a scale transformation and a rotation (or Lorentz transformation). This also shows that conformal transformations preserve angles, which is the origin of their name.
Eq.~\eqref{eq:conformalkillingeq} is sometimes called the (conformal) Killing equation and its solutions \eqref{eq:conformalkillingvec} the Killing vectors.



Note that in our derivation the original metric $g_{\mu\nu}$ was flat, but the new metric $g'_{\mu\nu}$ is not. It is however conformally flat: it is always possible to make a change of coordinate after which it is flat. In general, transformations
\begin{equation}
	g_{\mu\nu}(x) \to \Omega(x)^2 g_{\mu\nu}(x)
\end{equation}
are called \emph{Weyl transformations}. They change the geometry of space-time. We found that any Weyl transformation which is at most quadratic in $x$ can be compensated by a change of coordinates to go back to flat space. The corresponding flat-space symmetry is called conformal symmetry.%
%
\footnote{This implies that the group of conformal transformation is a subgroup of diffeomorphisms. It is in fact the largest \emph{finite-dimensional} subgroup.}% Rychkov (1.53)



Note that in $d = 2$ the situation is a bit different: the conditions $\partial^2 \sigma = 0$ is sufficient to ensure that the Killing equation has a solution. This is most easily seen in light-cone coordinates,
\begin{equation}
	x^+ = \frac{x^0 + x^1}{2},
	\qquad
	x^- = \frac{x^0 - x^1}{2},
\end{equation}
in terms of which
\begin{equation}
	\partial^2 \sigma = \partial_+ \partial_- \sigma.
\end{equation}
This is satisfied by taking for $\sigma$ the sum of an arbitrary function of the left-moving variable $x^+$ and of another function of the right-moving variable $x^-$. In fact, if we write $\varepsilon^\pm = \varepsilon^0 \pm \varepsilon^1$, we can 	take arbitrary functions $\varepsilon^+(x^+)$ and $\varepsilon^-(x^-)$, and verify that eq.~\eqref{eq:conformalkillingeq} is satisfied with $\sigma = \frac{1}{2} \left( \partial_+ \varepsilon_+ + \partial_- \varepsilon_- \right)$.
In the Euclidean case we take
\begin{equation}
	z = \frac{x^1 + i x^2}{2},
	\qquad
	\bar{z} = \frac{x^1 - i x^2}{2},
\end{equation}
complex-conjugate to each other, and the same logic follows: we can apply arbitrary holomorphic and anti-holomorphic transformations on $z$ and $\bar{z}$, and the conformal Killing equation is always satisfied. This shows that there are infinitely more conformal transformations in $d = 2$ than in $d > 2$, and also that there is no significant difference between two-dimensional Euclidean and Minkowski conformal symmetry, as the symmetry acts essentially on the two light-cone/holomorphic coordinates independently.


\subsection{The conformal algebra}

The conformal Killing equation \eqref{eq:conformalkillingvec} determines the most general form of \emph{infinitesimal} conformal transformations. \emph{Finite} conformal transformations follow from a sequence of infinitesimal transformations.
However, one has to bear in mind that the infinitesimal conformal transformations do not all commute: for instance, a translation followed by a rotation is not the same as the opposite.
In fact, the conformal transformations form a \emph{group}: the composition of conformal transformations is again a conformal transformations.

As we all know from quantum field theory, a group is characterized by its \emph{generators} and their commutation relations (the \emph{algebra}). A generator $G$ describes an infinitesimal transformation in some direction, and finite transformation are obtained using exponentiation, $e^{i \theta G}$, with parameter $\theta$ (the factor of $i$ is the physicist's convention that make the generators Hermitian).
A representation of the conformal group is given on the functions of the coordinates, $f(x)$. For instance, under an infinitesimal translation, we have
\begin{equation}
	f(x) \xrightarrow{P} f(x') = f(x + a) \approx f(x)
	+ a^\mu \partial_\mu f(x)
\end{equation}
and we require this to be equal to $e^{-i a_\mu P^\mu} f(x)$, which means
\begin{equation}
	P_\mu = i \partial_\mu.
	\label{eq:P:fcts}
\end{equation}
Performing the same analysis for the other infinitesimal transformations given in eq.~\eqref{eq:conformalkillingvec}, we obtain for the other generators%
%
\footnote{The sign of these generators is an arbitrary convention. It defines once and for all the commutations relations that we will derive next. After that, we will always refer to the commutation relations as the definition of the generators.}
\begin{align}
	& \text{rotations/Lorentz transformations:} \quad &
	M^{\mu\nu} &= i \left( x^\mu \partial^\nu
	- x^\nu \partial^\mu \right)
	\label{eq:M:fcts}
	\\
	& \text{dilatations:} & 
	D &= i x^\mu \partial_\mu,
	\label{eq:D:fcts}
	\\
	& \text{special conformal transformations:} &
	K^\mu &= i \left( 2 x^\mu x^\nu \partial_\nu 
	- x^2 \partial^\mu \right).
	\label{eq:K:fcts}
\end{align}
The number of generators matches that of the Killing vectors: there are $d$ translations, $d$ special conformal transformations, $d (d-1)/2$ rotations/Lorentz transformations ($M^{\mu\nu}$ is a $d \times d$ antisymmetric matrix), and one scale transformation.
Therefore the total number of generators, i.e.~the dimension of this group, is $(d + 1)(d + 2)/2$. In $d = 4$ space-time dimension, the conformal group has 15 generators.

Using the above definition, one can verify that the following commutation relations are satisfied,
\begin{equation}
\begin{aligned}
	\left[ M^{\mu\nu}, M^{\rho\sigma} \right]
	&= -i \left( g^{\mu\rho} M^{\nu\sigma} - g^{\mu\sigma} M^{\nu\rho}
	- g^{\nu\rho} M^{\mu\sigma} + g^{\nu\sigma} M^{\mu\rho} \right]
	\\
	\left[ M^{\mu\nu}, P^\rho \right]
	&= -i \left( g^{\mu\rho} P^\nu - g^{\nu\rho} P^\mu \right)
	\\
	\left[ M^{\mu\nu}, K^\rho \right]
	&= -i \left( g^{\mu\rho} K^\nu - g^{\nu\rho} K^\mu \right)
	\\
	\left[ D, P^\mu \right] &= -i P^\mu 
	\\
	\left[ D, K^\mu \right] &= i K^\mu 
	\\
	\left[ P^\mu, K^\nu \right]
	&= 2i \left( g^{\mu\nu} D - M^{\mu\nu} \right)
\end{aligned}
\label{eq:conformalalgebra}
\end{equation}
while all other commutators vanish:
\begin{equation}
	\left[ M^{\mu\nu}, D \right]
	= \left[ P^\mu, P^\nu \right] 
	= \left[ K^\mu, K^\nu \right] 	
	= 0.
\end{equation}
The first two relations in eq.~\eqref{eq:conformalalgebra} are the familiar Poincaré algebra. 
The next one states that $K^\mu$ transforms like a vector (as $P^\mu$ does), whereas $D$ is obviously a scalar. The next two relations remind us that $K^\mu$ and $P^\mu$ have respectively the dimension of length and inverse length. 

\begin{exercise}
	Derive the commutation relations from the action \eqref{eq:P:fcts}--\eqref{eq:K:fcts} of the generators on functions of $x$.
\end{exercise}

Even though this is not immediately obvious, this algebra is isomorphic to that of the group $\text{SO}(d+1, 1)$ (if $g^{\mu\nu}$ is the Euclidean metric) or $\text{SO}(d, 2)$ (if it is the Minkowski metric).
To see that it is the case, let us introduce a $(d + 2)$-dimensional space with coordinates
\begin{equation}
	X^\mu, \quad X^{d+1}, \quad X^{d+2},
\end{equation}
and a metric defined by the line element
\begin{equation}
	ds^2 = g_{\mu\nu} dX^\mu dX^\nu + dX^{d+1} dX^{d+1}
	- dX^{d+2} dX^{d+2}
	\equiv \eta_{MN} dX^M dX^N.
\end{equation}
Then we write all conformal commutation relations as being defined by the Lorentzian algebra
\begin{equation}
	\left[ J^{MN}, J^{RS} \right]
	= -i \left( \eta^{MR} J^{NS} - \eta^{MS} J^{NR}
	- \eta^{NR} J^{MS} + \eta^{NS} J^{MR} \right],
\end{equation}
provided that we identify the antisymmetric generators $J^{MN}$ with the conformal generators as follows:
\begin{equation}
\begin{aligned}
	M^{\mu\nu} &= J^{\mu\nu},
	\\
	P^\mu &= J^{\mu, d+1} + J^{\mu, d+2},
	\\
	K^\mu &= J^{\mu, d+1} - J^{\mu, d+2},
	\\
	D &= J^{d+1, d+2}.
\end{aligned}
\end{equation}


\subsection{Finite transformations}

We just saw that the infinitesimal conformal transformations generate a group. But how can we describe finite conformal transformations? Let us see how each generator exponentiates into an element of the group; the most general conformal transformation can then be obtained as a composition of such finite transformations.

In some cases the exponentiation is trivial. For instance, with translations we obtain immediately
\begin{equation}
	x^\mu \xrightarrow{P} x^\mu + a^\mu,
\end{equation}
where $a$ is now any $d$-dimensional vector, not necessarily small.
The same is true of scale transformations,
\begin{equation}
	x^\mu \xrightarrow{D} \lambda x^\mu
\end{equation}
with finite $\lambda$.
Rotations or Lorentz transformations exponentiate as
\begin{equation}
	x^\mu \xrightarrow{M} \Lambda^\mu_{~\nu} x^\nu
\end{equation}
where $\Lambda^\mu_{~\nu}$ is a $\text{SO}(d)$ or $\text{SO}(1, d-1)$ matrix, depending whether the metric is Euclidean or Minkowski.
All of this is well-known and not surprising.

On the contrary, special conformal transformations do not exponentiate trivially. The easiest way to derive their finite form is to make the following observation: recall that in infinitesimal form we have
\begin{equation}
	x'^\mu = x^\mu + 2 (b \cdot x) x^\mu - x^2 b^\mu,
\end{equation}
which implies $x'^2 = \left( 1 + 2 b \cdot x \right) x^2$, and therefore (as always neglecting terms of order $b^2$)
\begin{equation}
	\frac{x'^\mu}{x'^2}
	= \frac{x^\mu}{x^2} - b^\mu.
	\label{eq:K:inversion}
\end{equation}
The ratio $x^\mu/x^2$ appearing on both side of the equation is the \emph{inverse} of the coordinate $x^\mu$, respectively $x'^\mu$: let us define the inversion as
\begin{equation}
	x^\mu \xrightarrow{I} \frac{x^\mu}{x^2}.
\end{equation}
This transformation does not have an infinitesimal form, but otherwise it shares the essential properties of a conformal transformation: its Jacobian is
\begin{equation}
	\frac{\partial x'^\mu}{\partial x^\nu}
	= \frac{1}{x^2}
	\left[ \delta^\mu_\nu - 2 \frac{x^\mu x_\nu}{x^2} \right],
\end{equation}
which is the product of a position-dependent scale factor $x^{-2}$ with an orthogonal matrix. To understand what this transformation does globally, let us consider a point $\vec{x} = (x, 0, \ldots 0) \in \mathds{R}^d$. Then the matrix is square bracket is diagonal, and equates $\text{diag}(-1, 1, \ldots, 1)$. This is an orthogonal matrix with determinant $-1$, which is part of $\text{O}(d)$ but not $\text{SO}(d)$. This shows that the inversion is a discrete transformations not connected to identity.
A conformally invariant theory might be invariant under inversions, but it needs not be.

Eq.~\eqref{eq:K:inversion} shows that infinitesimal special conformal transformations are obtained taking an inversion followed by a translation, followed by an inversion again. Since this process involves the inversion twice, and since inversion is its own inverse, it does not matter whether inversion is a true symmetry of the system or not.
The advantage of this representation is that it can easily be exponentiated: the composition of (infinitely) many infinitesimal special conformal transformation can be written as an inversion followed by a finite translation, followed by an inversion again. In other words, eq.~\eqref{eq:K:inversion} holds for finite $b^\mu$.
This can be used to show that
\begin{equation}
	x^\mu \xrightarrow{K}
	x'^\mu = \frac{x^\mu - x^2 b^\mu}
	{1 - 2 b \cdot x + b^2 x^2}.
	\label{eq:K:finite}
\end{equation}

\begin{exercise}
	Use eq.~\eqref{eq:K:inversion} to show \eqref{eq:K:finite}.\\
	\hint{Contract both sides of eq.~\eqref{eq:K:inversion} with $x_\mu$, $x'_\mu$ and $b_\mu$, and use the three equations that you get to solve for $x'^2$.}
\end{exercise}

What do special transformation do globally?
Let us look specifically in Euclidean space. There are some special points:
\begin{itemize}

\item
The origin of the coordinate system $x = 0$ is mapped onto itself.

\item
The point $b^\mu / b^2$ is mapped to $\infty$. 

\item
Conversely, the ``point'' $x \to \infty$ is mapped to the finite vector $-b^\mu/b^2$.

\end{itemize}
%
These properties can be understood from the fact that special conformal transformations and translations are related by inversion:
special conformal transformations keep the origin fixed but move every other points, including $\infty$; translations move every point \emph{except} $\infty$. The other two transformations, rotations and scale transformations, keep both $0$ and $\infty$ fixed.


An essential property of conformal transformations is that they allow to map any 3 points $(x_1, x_2, x_3)$ onto another set $( x'_1, x'_2, x'_3)$. This can be seen as follows: first, apply a translation to place $x_1$ at the origin, followed by a special conformal transformation that takes $x_3$ to $\infty$, after which the image of the original triplet is $(0, x_2'', \infty)$; then use rotations and scale transformations to move $x''_2$ to another point $x'''_2$, while keeping $0$ and $\infty$ fixed; finally apply again a special conformal transformation that takes $\infty$ to $x'_3 - x'_1$, and a translation by $x'^1$ to reach the configuration $( x'_1, x'_2, x'_3)$.
This property has an immediate physical consequence: in correlation functions of 2 or 3-points (see next sections for a definition), all kinematics is fixed by conformal symmetry. The only freedom encodes information about the operators themselves, not about their position in space.

Another interesting property of conformal transformations is that they map spheres to spheres: this is is an obvious property of translations, rotations and scale transformations, but it is also true of special conformal transformations. 

\begin{exercise}
	Show that under the special conformal transformation \eqref{eq:K:finite}, a sphere centered at the point $a^\mu$ and with radius $R$ gets mapped to a sphere centered at the point 
	$$
	a'^\mu = \frac{a^\mu - (a^2 - R^2) b^\mu}
	{1 - 2 a \cdot b + (a^2 - R^2) b^2}
	$$
	and with radius
	$$
	R' = \frac{R}{\left| 1 - 2 a \cdot b + (a^2 - R^2) b^2 \right|}.
	$$
	In the special case in which $b^\mu / b^2$ is on the original sphere, show that the sphere gets mapped to a plane orthogonal to the vector $a^\mu + (R^2 - a^2) b^\mu$.
	
\end{exercise}

Note that in $d = 2$, the additional conformal transformations (infinitely many of them!) mean that (nearly) any shape can be mapped onto another. This is known as the Riemann mapping theorem.%
%
\footnote{A neat example of how a disk is be mapped onto a polygon is available at \\
\url{https://herbert-mueller.info/uploads/3/5/2/3/35235984/circletopolygon.pdf}.}


\subsection{Compactifications}

It is often said that conformal symmetry is a symmetry of flat space(-time).
This is true, as we have just seen, provided that we treat the point $\infty$ as being part of the space.
This is quite straightforward in Euclidean space, but much more subtle in Minkowski space-time, as there are different ways of reaching $\infty$ there.

To gain a better understanding of this, it can be useful to map the flat Euclidean space $\mathds{R}^d$ or the Minkowski space-time $\mathds{R}^{1,d-1}$ onto a curved (but hopefully compact) manifold.
In Euclidean space this is for instance achieved by the (inverse) stereographic projection that maps $\mathds{R}^d \cup \{ \infty \}$ to the unit sphere $S^d$.
Geometrically, the stereographic projection is constructed as follows: embed $\mathds{R}^d$ as a (hyper-)plane in $\mathds{R}^{d+1}$, together with a sphere of unit radius centered at the origin. Every point on the plane has an image on the sphere obtained drawing a segment between the original point and the north pole of the sphere, and noting where it intersects the sphere. The origin is mapped to the south pole, $\infty$ to the north pole, and the sphere $S^{d-1}$ of unit radius to the equator.
Algebraically, this is achieved as follows: first write the Euclidean metric in spherical coordinates,
\begin{equation}
	ds^2 = dr^2 + r^2 d\Omega_{d-1}^2,
\end{equation}
where the solid angle is given in $d = 2$ by $d\Omega_1^2 = d\phi^2$, in $d = 3$ by $d\Omega_2^2 = d\theta^2 + \sin\theta^2 d\phi^2$, and more generically by the recursion relation $d\Omega_n^2 = d\theta^2 + \sin\theta^2 d\Omega_{n-1}^2$.
Let us perform the change of variable
\begin{equation}
	r = \frac{\sin\varphi}{1 - \cos\varphi},
\end{equation}
and interpret $\varphi \in [0, \pi]$ as the zenith angle on the sphere:
$\varphi = 0$ is the north pole, corresponding to $r \to \infty$, and $\varphi = \pi$ the south pole, corresponding to $r = 0$. In these coordinates we have
\begin{equation}
	ds^2 = \frac{1}{(1 - \cos\varphi)^2}
	\left( d\varphi^2 + \sin\varphi^2 d\Omega_{d-1}^2 \right)
	= \frac{1}{(1 - \cos\varphi)^2} d\Omega_d^2.
\end{equation}
This shows that the metric is flat up to an overall Weyl factor that depends on $\varphi$.
In these new coordinates, conformal transformations are always non-singular. 
For this reason, it is often very convenient to study \emph{classical} conformal symmetry on the sphere $S^d$ instead of Euclidean space.

However, this compactification is not as nice in a \emph{quantum} theory in which one wants to foliate the space along some direction: if one chooses $\varphi$ as the ``Euclidean time'', then the ``space'' direction is a sphere $S^{d-1}$ whose volume depends on $\varphi$. In other words, the generator of ``time'' translations is not a symmetry of the system. Being on a perfectly symmetric sphere, this is also true of any other choice of time direction.
This is not forbidden, but it complicates vastly the analysis.

Instead, another compactification is often preferred to the sphere: from the Euclidean metric in spherical coordinates, one can make the change of variable
\begin{equation}
	\tau = \log(r)
	\qquad\Rightarrow\qquad
	r = e^\tau,
\end{equation}
after which
\begin{equation}
	ds^2 = r^2 \left( d\tau^2 + d\Omega_{d-1}^2 \right).
\end{equation}
This is again a flat metric, up to a Weyl factor $r^2 = e^{2\tau}$. In this case however the metric is completely independent of $\tau$.
This space has the geometry of the \emph{cylinder}, $\mathds{R} \times S^{d-1}$. It is not fully compact: $\tau$ goes from $-\infty$ to $+\infty$. But it has a important advantage: translations in $\tau$ are generated by dilatations $D$, which will be taken to be a symmetry of quantum field theory. Foliating the space into surfaces of constant $\tau$ will later lead us to the famous radial quantization in conformal field theory.

Note that in $\text{SO}(d + 1, 1)$ language, the generator $D = J^{d+1, d+2}$ is completely equivalent to any other generator $J^{\mu, d+2} = \frac{1}{2} \left( P^\mu - K^\mu \right)$. So for instance we might as well look for a cylinder compactification in which the non-compact direction corresponds to translations in the direction of
$\frac{1}{2} \left( P^0 - K^0 \right)$. This generator obeys
\begin{equation}
	\frac{1}{2} \left( P^0 - K^0 \right)
	= i \left( \frac{1 - (x^0)^2 + \vec{x}^2}{2} \partial_0
	- x^0 \, \vec{x} \cdot \vec{\partial} \right),
\end{equation}
with two fixed points at $x^0 = \pm 1$ with $\vec{x} = 0$.

\begin{exercise}
	Find the change of coordinate that make the Euclidean metric Weyl-equivalent to a cylinder in which translations in the non-compact direction are generated by $\frac{1}{2} \left( P^0 - K^0 \right)$. \\
	\hint{Find a special conformal transformation followed by a translation that take $(0, \infty)$ to $(-1, 1)$, and apply it to the cylinder coordinates.}
\end{exercise}

The cylinder compactifications of Euclidean space are interesting by themselves, but they are also extremely convenient to understand the connection between Euclidean and Minkowski space-times: in this last form, performing a Wick rotation $\tau \to \pm i t$ defines a Lorentzian cylinder on which the conformal group with a simple action of the Lorentzian conformal group $\text{SO}(d, 2)$. 
But before going there, let us go back to flat Minkowski space-time and make some general remarks.


\subsection{Minkowski space-time}

Everybody is familiar with translations and Lorentz transformations in Minkowski space-time, and even with dilatations as this is a standard tool in the renormalization group analysis. But what do conformal transformations do?

To understand this, let us place an observer at the origin of Minkowski space-time. The presence of this observer breaks translations, but not Lorentz transformations (the observer is point-like), nor dilatations and special conformal transformations. For the observer, space-time is split into three regions: a future light cone, a past light cone, and a   space-like region from which they know nothing.
Lorentz and scale transformation preserve this causal structure: the future and past light-cones are mapped onto themselves. In other words, if a point $x$ is space-like separated from the observed, it will remain space-like separated no matter the choice of scale and Lorentz frame. Without loss of generality, let us choose this point to be at position $x = (0, \vec{n})$, where $\vec{n}$ is a unit vector (units can be chosen so that this is the case).
Now apply a special conformal transformation with parameter $b^\mu = (-\alpha, \alpha \vec{n})$, with $\alpha$ varying between 0 and 1.
This draws a curve $y^\mu $ in space-time, parameterized by $\alpha$,
with
\begin{equation}
	y^0(\alpha) = \frac{\alpha}{1 - 2 \alpha},
	\qquad\qquad
	\vec{y}(\alpha) = \frac{1 - \alpha}{1 - 2 \alpha} \vec{n}.
\end{equation}
This curves begins at the space-like point $y = (0, \vec{n})$, and ends in the past light cone at $y = (-1, \vec{0})$!
Note that $y$ never crosses a light cone: the image of $x$ is never null if $x$ is not itself null, since $x'^2 = x^2 / (1 - 2 x \cdot b + x^2 b^2)$.
Instead, we have
\begin{equation}
	y^2(\alpha) = \frac{1}{1 - 2\alpha} \neq 0.
\end{equation}
The point goes all the way to space-like infinity when $\alpha = \frac{1}{2}$, and comes back from past infinity.
Clearly, special conformal transformation break causality.

The resolution of this paradox is that conformal transformation do not act directly on Minkowski space, but rather on its universal cover that is isomorphic to the Lorentzian cylinder described in the previous section.
Time evolution on that cylinder is given by the Hamiltonian $H = \frac{1}{2} (P^0 - K^0)$. At any given time $t_0$, space is compactified in such a way that the notion of infinite distance is unequivocal: space-like infinity corresponds to a point on the sphere. If one takes any other point of that sphere and apply finite translations using the generator $P^\mu$, then this defines a compact \emph{Poincaré patch}. The full cylinder is a patchwork of Minkowski space-times, but every observer only has access to one.

The lesson that we need to learn from this is that only the infinitesimal form of special conformal transformation can be used in Minkowski space-time: \emph{any} finite special conformal transformation would bring part of space-time into another patch on the cylinder.
This is sometimes called \emph{weak conformal invariance}.

The fact that a quantum field theory defined on Minkowski space-time with \emph{weak conformal invariance} uniquely defines a quantum field theory on the Lorentzian cylinder with global conformal invariance is quite non-trivial. The proof was given by Martin Lüscher and Gerhard Mack in 1974 \cite{Luscher:1974ez}, both employed at the University of Bern at the time. In their own words:

\begin{center}

\parbox{11cm}{%
\emph{In picturesque language, [the superworld] consists of Minkowski space, infinitely many ``spheres of heaven'' stacked above it and infinitely many ``circles of hell'' below it.}}

\end{center}


\subsection{The energy-momentum tensor}

\ldots

%classical field theory and action principle
%
%postulate that conformal transformation are a symmetry of the theory
%
%by Noether's theorem, this implies the existence of a conserved current
%
% energy-momentum tensor
%
%use the metric as source?
%
%condition for Weyl invariance is tracelessness of energy-momentum tensor

\newpage
\bibliography{Bibliography}
\bibliographystyle{utphys}
\end{document}

%%%%%%%%%%%%%%%%%%%%%%%%%%%%%%%%%%%%%%%%%%%%%%%%%%%%%%%%%%%%%%%%%%%%%%%

\section{Quantum field theory with conformal symmetry}
\label{sec:quantum}

Wightman axioms

operators are not necessarily invariant under conformal symmetry, but they transform unitarily

unitary representations on Hilbert space

(e.g.~pair of point-like particle

no action principle
(i.e.~no need for ``quantization'')


only deal with \textbf{local} operators

warning about meaning of locality: related to causality?



generator of translations plays a special role: energy and momentum vector

positivity of energy 



note: assuming symmetry are there at the quantum level, there exists a traceless energy-momentum tensor and generators of conformal transformations can be constructed from them

(this only goes in one direction: given $T$, one can construct generators; but given the symmetry, Noether's theorem only gives a conserved current if there is a Lagrangian, it is unproven for non-lagrangian theories; additional condition of existence of energy-momentum tensor is sometimes called \emph{locality})

these generators are non-local!
but their commutators with local operators is local (see Simmons-Duffin)

discuss connection with path integral formalism (in an exercise? or a parenthesis?)


\subsection{Non-perturbative QFT}

%\subsection{Hilbert space(s)}

take $P^0$ as the Hamiltonian

space-time is foliated by surfaces of equal time

each equal-time surface has a Hilbert space; evolution between different surfaces at times $t_1$ and $t_2$ is given by the unitary evolution operator $U = e^{\pm i (t_1 - t_2) P^0}$

because of time-translation symmetry, the Hilbert space is the same on each surface!

states characterized by energy and momentum: $P^\mu \ket{p} = p^\mu \ket{p}$

there is a unique vacuum state: $\ket{0}$

there are local operators: an infinity of them
example: free scalar theory

%\subsection{Wightman functions}


$\phi(x)$ are not operators, but operator-valued distributions

smeared operator $\phi[f]$ have finite norm; $\phi(x)$ don't

but don't worry for the rest of these lectures


positivity of norm of smeared operators implies positivity of thing multiplying delta function!




good observable: Wightman correlation function
(note: operator need not be time-ordered; time evolution is unitary, so it can go both ways!)

operators define correlation functions, but the opposite is also true (Wightman reconstruction theorem); so use the latter and forget about Hilbert space



but then lose connection with path integral formulation: this gives time-ordered correlation functions!



representation theory: follow Shester's notes
commutation relations for infinitesimal transformations
exponentiated form is an abstract unitary operator

forget about unitary rep of SCT: only act on Lorentz cylinder, infinite cover of Minkowski space-time; we know this rep exists
but keep infinitesimal form




Wigner construction: use reference momentum


generator of Lorentz transformation can be diagonalize at a point: the origin of space-time

get the action of $M^{\mu\nu}$ at arbitrary point



conformal Ward identities



%\subsection{Spectral representation}

operator vs.~field

\subsection{Scale symmetry}

action of generator of translations

no mass (often said)

or better said: all masses!

reference vector, then construct spectral density


remark: only operator that is a scalar with scaling dimension zero can have a one-point function



\subsection{Special conformal symmetry}

note: finite special conformal transformations brings us outside of Minkowski space-time;
but no need for them: only infinitesimal form

special conformal generator implies two things:

primary and descendant operators

special condition on 2-point function of primary operators: that they have identical scaling dimensions


%%%%%%%

note that scale invariance and unitarity generally imply that the trace of the energy-momentum tensor is zero: then Hamiltonian is invariant under change of the metric $\delta g_{\mu\nu} = \sigma(x) \eta_{\mu\nu}$,
\begin{equation}
	\delta H = \int d^dx \, T_{\mu\nu} \delta g^{\mu\nu}
	= \int d^x \, \sigma(x) T^\mu_\mu = 0
\end{equation}
invariance under infinitesimal Weyl transformations 

note that $T_{\mu\nu}$ is only defined in flat space, or rather that it acquires a vacuum expectation value in curved space: the Weyl anomaly
so let us only consider Weyl transformations that do not change the geometry of space-time: those that are equivalent to a coordinate transformation (a diffeomorphism)


exercise: compute conformal generators in momentum-space representation


Mack's classification in 4d


representation: long and short multiplets!



%%%%

compare this with Slava's approach:

organize operators into representation of conformal group:

call some operators \emph{primary}

operators are local, which mean that they do not feel the effect of the transformation of the metric, or only through coordinate dependence:
$\phi(x) \to \phi(x') = \Omega(x)^{-\Delta} \phi(x)$

note: if $\phi$ transforms like this, then $\partial_\mu \phi(x)$ does not! call it a \emph{descendant}


comment on scale vs.~conformal invariance

correlation functions:

no need for action principle


\subsection{Time-ordered products and path integrals}


Simmons-Duffin intro


\subsection{The energy-momentum tensor}


locality

allows to construct conformal charges


\subsection{Exercises}

show that the spectral density in the case $\Delta = \frac{d}{2} - 1$ is that of a free scalar field;

show also that in the case $\Delta = \frac{d}{2}$ it is given by the phase space of two massless particle states (see Rychkov eq. 1.13)


%%%%%%%%%%%%%%%%%%%%%%%%%%%%%%%%%%%%%%%%%%%%%%%%%%%%%%%%%%%%%%%%%%%%%%%

\section{Conformal correlators}
\label{sec:correlators}

correlation functions aka Green's functions

remark about ``physicality''
in ordinary QFT, correlation functions are not physical observables: they depend on renormalization scale; only scattering amplitudes are!
here no scale: they are physical


2-pt function: go from Wightman 2-pt function in momentum space to Euclidean 2-pt function in position space in two ways:

- Fourier transform first, then analytic continuation of Wightman function
- first construct T product (Källen-Lehmann representation), then Wick rotation to Euclidean momentum space, then Fourier transform

%%%%%


T-products vs. Wightman functions

use retarded products and micro-causality


Ward identities

more identities for conserved currents and the energy-momentum tensor
(there cannot be higher-spin conserved currents otherwise the theory must be free; reference?)


generalities:

one-point functions vanish (in flat space; not so on conformally flat manifolds such as $S^d$, where 1-point function might depend on radius)

\subsection{Spectral representation for spinning operators}

unitary bounds

\subsection{From momentum to position}

taking fourier transforms

\subsection{From Minkowski to Eulidean space}

analytic continuation of Wightman functions vs. Wick rotation of T-products

reflection positivity


going in the other direction: Osterwalder-Schrader


do we need to use the cylinder interpretation?
I prefer not: only CFT in flat space here



\subsection{Embedding-space formalism}

conformal group acts linearly in embedding space with $d+2$ dimensions, among which two times

$X'^M = \Lambda^M_{~N} X^N$

get rid of two dimensions by restricting to light-cone $X^2 = 0$, and identifying $X^+ \sim \lambda X^+$


can take $X^+ = 1$, so that
\begin{equation}
	(X^+, X^-, X^\mu) = (1, x^2, x^\mu)
\end{equation}


fields defined on the cone, depending homogeneously on $X$

rewrite 2-point function


see section 2.1 of Rychkov

\subsection{3-point functions}

scalar 3-point function

remarkable fact! fixed up to a single coefficient

if only scale invariance, this could be arbitrarily complicated

historical note (cite Rychkov, around eq. 2.43): birth of conformal field theory


check: 2-point functions involving conserved currents and energy-momentum tensors are automatically conserved; for 3-point functions this gives additional constraint: scaling dimensions of operators must be equal



\subsection{4-point functions}
% and higher?

invariant cross-ratios for 4-point functions

%%%%%%%%%%%%%%%%%%%%%%%%%%%%%%%%%%%%%%%%%%%%%%%%%%%%%%%%%%%%%%%%%%%%%%%

\section{State-operator correspondence and OPE}
\label{sec:OPE}


goal of this section: prove state/operator correspondence heuristically, justifying a posteriori some facts assumed before

not only a curiosity: also useful computationally


\subsection{N-S quantization}

different foliation of space-time than the one discussed above



same Hilbert space on a slice at $t = 0$; use different evolution operator


conjugation in Euclidean space?

takes $x_d \to - x_d$

use Hamiltonian
\begin{equation}
	H = P_d + K_d
\end{equation}
instead of $P_d$

fixed points

operators can be evolved to fixed points by Hamiltonian


\subsection{Radial quantization}

in Euclidean, take the generator of dilatations as Hamiltonian

simple rotation in $\text{SO}(d+1,1)$

apply conformal transformation mapping the two fixed points to 0 and $\infty$

conjugation is inversion $r \to 1/r$

generators $P_\mu$ and $K_\mu$ are conjugate under inversion!

raising and lowering operators


equivalence between state living on a sphere (arbitary radius) and operators inserted at the origin

organize states as eigenstates of scale and Lorentz (commuting): matches the construction we did above!

this shows that every state can be created by local primary operator (and possibly action of translation generators); conversely, every operator defines a state (this part was already clear)

not necessarily so in QFT! (e.g.~gauge theories? do we also need non-local operators? Wilson lines?)



how to define an operator from a state? see Rychkov (3.26)
operator is defined by all of its correlation functions



note: state/operator correspondence validates our interpretation of Wightman 2-point function

note 2: operator inside the unit sphere need not be smeared anymore; they have a well-defined norm!



exercise: unitarity bound in radial quantization (Rychkov section 3.2)



path integral interpretation: state is created by performing path integral inside unit sphere



\subsection{Conformal OPE}

derive OPE from radial quantization

relate this to Lorentzian OPE on vacuum


use in 3-point function: relates OPE coefficient with 3-pt fct coefficient
(structure constants of the operator algebra)

exercise: compute function entering the OPE


\subsection{OPE convergence}

Hilbert-space argument


radius? 

absolute convergence as long as one can find a sphere separating the points:

note that a plane is also a sphere with radius infinity


%%%


concluding remarks:


CFT data: spectrum of operators and OPE coefficients
completely determines the theory!

OPE reduces any higher-point function to 2-point


%%%%%%%%%%%%%%%%%%%%%%%%%%%%%%%%%%%%%%%%%%%%%%%%%%%%%%%%%%%%%%%%%%%%%%%

\section{The conformal bootstrap}
\label{sec:bootstrap}

any CFT data defines a good theory? no!

different quantization surfaces gives different OPEs in the same theory

OPE associativity

often called ``crossing symmetry'' (note: quite different from crossing symmetry of scattering amplitudes)

(in fact this argument shows that Schwinger functions are symmetric!)

focus on 4-point functions:
all 4-point functions of all operators contain all the associativity constraints


simplifications:
only 4 identical operators, all scalars


\subsection{Conformal blocks}

rather conformal partial waves

diagrammatics (not to be confused with Feynman diags)



square configuration and expansion in rho coordinates



\subsection{The numerical boostrap}

discussion about universality: see Poland, Rychkov, Vichi
also Shai Chester's notes



bound on scaling dimensions

also spectrum at the boundary


\subsection{Generalized free field theory}

aka mean free theory, Gaussian theory

double-twist operator

corresponds to large-N limit of gauge theory


\subsection{Analytic bootstrap}

Euclidean $z^* = \bar{z}$
$z = \bar{z} = \frac{1}{2}$ gives bootstrap equations that constrain operators of low scaling dimension $\Delta$

Minkowski: $z \neq \bar{z}$, both real 
light-cone limit $z \to 0$, $\bar{z}$ fixed is dominated by operators of low twist 

rigorous bound in the limit $\ell \to \infty$

Regge trajectories




%%%%%%%%%%%%%%%%%%%%%%%%%%%%%%%%%%%%%%%%%%%%%%%%%%%%%%%%%%%%%%%%%%%%%%%

\section{Selected advanced topics}

light-cone limit and ?

Virasoro symmetry in 2d:


conformal anomalies: anomalies are contact term in the action for the source


superconformal bootstrap? note classification of all superconformal algebras


\subsection{The 2d conformal bootstrap}

historical presentation

see Rychkov section 4.3.1

advantage 1:
full Virasoro symmetry

advantage 2:
easier to deal with spin: left- and right-moving, i.e.~holomorphic and anti-holomorphic

by looking at just the energy-momentum tensor, already get strong constraints on its 2-point function (the central charge)

for $c < 1$, only minimal models possible:
match Ising, \ldots

finitely many primaries in 2d (still infinitely many quasi-primaries)

note: infinitely many primaries for theories with $c > 1$

also can put the theory on torus and study modular properties of partition function


\subsection{How to continue from here}


\begin{itemize}

\item
holography: see Joao's TASI lectures

\item
superconformal: Tajikawa pedestrian lectures

\item
bootstrap: Shai Chester's lectures

\item
condensed matter:
note that unitarity is unnecessary: many non-unitary fixed points, yet they have conformal invariance (why?)

\end{itemize}




\begin{center}	
	
\parbox{11.8cm}{
\emph{%
[Courses] are fantastically good for learning physics. The lecturer learns a lot of physics. After my first few studies, just about everything I learned about physics came from teaching it. I don’t know if the students learned a lot, but I certainly did. So I consider teaching physics very important.}
--- Leonard Susskind \cite{Susskind}}

\end{center}


\bibliography{Bibliography}
\bibliographystyle{utphys}

\end{document}
