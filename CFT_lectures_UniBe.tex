\documentclass[a4paper,12pt]{article}

%\usepackage[left=25mm, right=25mm, top=30mm, bottom=30mm]{geometry}
\usepackage[utf8]{inputenc}
\usepackage[english]{babel}
\usepackage{amsmath}
\usepackage{amsfonts}
\usepackage{mathrsfs}
\usepackage{dsfont}
\usepackage{graphicx}
\usepackage{cite}
\usepackage[colorlinks, allcolors=blue, linktocpage=true]{hyperref}

%\renewcommand{\arraystretch}{1.5}

%\renewcommand{\O}{\mathcal{O}}
\newcommand{\ket}[1]{\left| #1 \right\rangle}
\newcommand{\bra}[1]{\left\langle #1 \right|}
\newcommand{\SO}{\text{SO}}


% double angle-bracket notation:
%\makeatletter
%\newsavebox{\@brx}
%\newcommand{\llangle}[1][]{\savebox{\@brx}{\(\m@th{#1\langle}\)}%
%  \mathopen{\copy\@brx\kern-0.5\wd\@brx\usebox{\@brx}}}
%\newcommand{\rrangle}[1][]{\savebox{\@brx}{\(\m@th{#1\rangle}\)}%
%  \mathclose{\copy\@brx\kern-0.5\wd\@brx\usebox{\@brx}}}
%\makeatother

\numberwithin{equation}{section}


%%%%%%%%%%%%%%%%%%%%%%%%%%%%%%%%%%%%%%%%%%%%%%%%%%%%%%%%%%%%%%%%%%%%%%%

\title{%
Conformal Field Theory
\\[1em]
\Large
Lecture notes}

%\title{Conformal Field Theory for Particle/High-Energy Physicists}

\author{Marc Gillioz}

\date{Spring semester 2022}


\begin{document} 

\maketitle

\begin{center}
	\parbox{11.8cm}{\emph{%
[Courses] are fantastically good for learning physics. The lecturer learns a lot of physics. After my first few studies, just about everything I learned about physics came from teaching it. I don’t know if the students learned a lot, but I certainly did. So I consider teaching physics very important.} --- Leonard Susskind}
\end{center}
%https://www.delta.tudelft.nl/article/leonard-susskind-sleeping-during-lecture-good-thing#

%%%%%%%%%%%%%%%%%%%%%%%%%%%%%%%%%%%%%%%%%%%%%%%%%%%%%%%%%%%%%%%%%%%%%%%

\newpage

\tableofcontents

%%%%%%%%%%%%%%%%%%%%%%%%%%%%%%%%%%%%%%%%%%%%%%%%%%%%%%%%%%%%%%%%%%%%%%%

\newpage
\section{Introduction}

\begin{equation}
	\text{conformal field theory}
	= \text{quantum field theory} + \text{conformal symmetry}
\end{equation}


%\begin{equation}
%	\text{CFT}
%	= \text{QFT} 
%	+ \text{Poincaré}
%	+ \text{Lorentz}
%	+ \text{scale}
%	+ \text{special conformal symmetry}
%\end{equation}
where Poincaré symmetry means translations and Lorentz transformations (rotations and boosts)

\begin{equation}
	\text{conformal field theory}
	= \text{relativistic quantum field theory} 
	+ \text{scale}
	+ \text{special conformal symmetry}
\end{equation}

most reviews focus first on conformal symmetry as a whole 
``geometric'' approach to CFT

for good reasons!
but often makes it difficult to connect with standard understanding of relativistic quantum field theory that we have from particle physics

things that are hard to connect to:
- spectral representation, particles (is CFT a theory of massless particles? no!)
- UV/IR divergences and anomalies
these are precisely subjects that will be covered in these lectures

peculiarity of these lecture notes:
``algebraic'' approach to CFT
review non-perturbative approach of QFT, then add scale and special conformal symmetry;
``anatomical'' approach allows to understand what are the consequences of each of them



\subsection{What is conformal field theory?}

Introduction: What is CFT? Why is it useful. Examples of CFT. Where does it fit into modern theoretical physics.


in the last few years, CFT has been dominated by 

string theory: 2d CFT

holography: geometric approach

condensed matter physics: Euclidean


here focus on the ``old'' quantum field theory approach

links with: lattice, perturbation theory, etc.

fun fact: the conformal bootstrap was invented by particle physicists


\textbf{strongly-coupled QFT}

no need for action principle


\subsection{Examples of conformal field theories}

perturbative examples: $\phi^n$ theory in non-integer $d$

Caswell-Banks-Zaks

superconformal: $\mathcal{N} = 4$


any IR fixed point (maybe trivial)


use beyond CFT:
correlators that have the isometries of the conformal group
gravity in AdS
but also late-time correlators in de Sitter

\subsection{Outline}

originally covered in 14 periods of 45 minutes each

split into 7 chapters?

%%%%%%%%%%%%%%%%%%%%%%%%%%%%%%%%%%%%%%%%%%%%%%%%%%%%%%%%%%%%%%%%%%%%%%%

\section{Classical conformal symmetry}


Poincar\'e symmetry: physics is the same in every coordinate frame

linear transformation of the coordinates
\begin{equation}
	x'^\mu = x^\mu + \omega^{[\mu\nu]} x_\nu + a^\mu
\end{equation}


infinitesimal line element $dx^2 = \eta_{\mu\nu} dx^\mu dx^\nu$ is invariant:
\begin{equation}
	dx'^2 = dx^2
\end{equation}


scale symmetry:
\begin{equation}
	x'^\mu = \lambda x^\mu
\end{equation}

note that this can be seen as a rescaling of the metric, of the form
$\eta'^{\mu\nu} = \# \eta^{\mu\nu}$


can we generalize this? general transformations of the metric $\eta'^{\mu\nu} = \Omega(x) \eta^{\mu\nu}$
called a \emph{Weyl transformation}

in general changes geometry of space-time; unless this transformation is equivalent to a coordinate transformation (diffeomorphism)

in infinitesimal form
\begin{equation}
	x'^\mu = x^\mu + v^\mu(x)
\end{equation}
requiring that this is equal to the Weyl transformation gives the conformal Killing equation:
\begin{equation}
	\partial^\mu v^\nu(x) + \partial^\nu v^\mu(x)
	= 2 \sigma(x) \eta^{\mu\nu}
\end{equation}

work out solution in $d = 1$, $d = 2$ and $d > 2$;


conformal Killing vectors (see Osborn's notes):
translation
Lorentz transformations
dilatation
special conformal symmetry



in $d = 2$, use $z$, $\bar{z}$ notation and Laurent series






finite transformations


use inversions: discrete transformations not connected to identity

but inversion followed by translation followed by inversion gives special conformal transformations


group of conformal transformation is a subgroup of diffeomorphisms; in fact the largest \emph{finite-dimensional} subgroup
% Rychkov (1.53)



defining property
\begin{equation}
	g'_{\mu\nu}(x') = \Omega(x)^2 \eta_{\mu\nu}
\end{equation}
gives Jacobian of conformal transformation
\begin{equation}
	\frac{\partial x^\mu}{\partial x^\nu}
	= \Omega(x) M^\mu_{~\nu}(x)
\end{equation}
$M$: $\SO(d-1, 1)$ matrix

conformal transformations look locally like Lorentz and scale transformations


\subsection{The conformal algebra}

generators corresponding to Killing vectors

when applied to function of $x$, gives derivative action



compute commutators, find Lorentz and Poincaré subalgebras, and new commutations involving $D$ and $K$

isomorphic to $\SO(d, 2)$

(exercise)


note: everything we said is also valid in Euclidean space: replace $\eta^{\mu\nu} \to \delta^{\mu\nu}$

get the Euclidean conformal group



\subsection{Examples}

conformal transformations map circles to circles (including circles with infinite radius, aka straight lines): explanation

picture

think of it as renormalization group transformation that depends on space-time

Escher?




%%%%%%%%%%%%%%%%%%%%%%%%%%%%%%%%%%%%%%%%%%%%%%%%%%%%%%%%%%%%%%%%%%%%%%%

\section{Non-perturbative quantum field theory}

Wightman axioms

operators are not necessarily invariant under conformal symmetry, but they transform unitarily

unitary representations on Hilbert space

(e.g.~pair of point-like particle

no action principle
(i.e.~no need for ``quantization'')


only deal with \textbf{local} operators

warning about meaning of locality: related to causality?



generator of translations plays a special role: energy and momentum vector

positivity of energy 

\subsection{Hilbert space(s)}

take $P^0$ as the Hamiltonian

space-time is foliated by surfaces of equal time

each equal-time surface has a Hilbert space; evolution between different surfaces at times $t_1$ and $t_2$ is given by the unitary evolution operator $U = e^{\pm i (t_1 - t_2) P^0}$

because of time-translation symmetry, the Hilbert space is the same on each surface!

states characterized by energy and momentum: $P^\mu \ket{p} = p^\mu \ket{p}$

there is a unique vacuum state: $\ket{0}$

there are local operators: an infinity of them
example: free scalar theory

\subsection{Wightman functions}

good observable: Wightman correlation function
(note: operator need not be time-ordered; time evolution is unitary, so it can go both ways!)

but then lose connection with path integral formulation: this gives time-ordered correlation functions!




Wigner construction: use reference momentum


generator of Lorentz transformation can be diagonalize at a point: the origin of space-time

get the action of $M^{\mu\nu}$ at arbitrary point



conformal Ward identities



\subsection{Spectral representation}

operator vs.~field

\subsection{Scale symmetry}

action of generator of translations

no mass (often said)

or better said: all masses!

reference vector, then construct spectral density


remark: only operator that is a scalar with scaling dimension zero can have a one-point function



\subsection{Special conformal symmetry}

special conformal generator implies two things:

primary and descendant operators

special condition on 2-point function of primary operators: that they have identical scaling dimensions


%%%%%%%

note that scale invariance and unitarity generally imply that the trace of the energy-momentum tensor is zero: then Hamiltonian is invariant under change of the metric $\delta g_{\mu\nu} = \sigma(x) \eta_{\mu\nu}$,
\begin{equation}
	\delta H = \int d^dx \, T_{\mu\nu} \delta g^{\mu\nu}
	= \int d^x \, \sigma(x) T^\mu_\mu = 0
\end{equation}
invariance under infinitesimal Weyl transformations 

note that $T_{\mu\nu}$ is only defined in flat space, or rather that it acquires a vacuum expectation value in curved space: the Weyl anomaly
so let us only consider Weyl transformations that do not change the geometry of space-time: those that are equivalent to a coordinate transformation (a diffeomorphism)


exercise: compute conformal generators in momentum-space representation


Mack's classification in 4d


representation: long and short multiplets!



%%%%

compare this with Slava's approach:

organize operators into representation of conformal group:

call some operators \emph{primary}

operators are local, which mean that they do not feel the effect of the transformation of the metric, or only through coordinate dependence:
$\phi(x) \to \phi(x') = \Omega(x)^{-\Delta} \phi(x)$

note: if $\phi$ transforms like this, then $\partial_\mu \phi(x)$ does not! call it a \emph{descendant}


comment on scale vs.~conformal invariance

correlation functions:

no need for action principle



\subsection{Exercises}

show that the spectral density in the case $\Delta = \frac{d}{2} - 1$ is that of a free scalar field;

show also that in the case $\Delta = \frac{d}{2}$ it is given by the phase space of two massless particle states (see Rychkov eq. 1.13)


\subsection{Literature}

itzikson zuber

%%%%%%%%%%%%%%%%%%%%%%%%%%%%%%%%%%%%%%%%%%%%%%%%%%%%%%%%%%%%%%%%%%%%%%%

\section{Conformal correlators}


2-pt function: go from Wightman 2-pt function in momentum space to Euclidean 2-pt function in position space in two ways:

- Fourier transform first, then analytic continuation of Wightman function
- first construct T product (Källen-Lehmann representation), then Wick rotation to Euclidean momentum space, then Fourier transform

%%%%%


T-products vs. Wightman functions

use retarded products and micro-causality


Ward identities

more identities for conserved currents and the energy-momentum tensor
(there cannot be higher-spin conserved currents otherwise the theory must be free; reference?)


\subsection{Spectral representation for spinning operators}

unitary bounds

\subsection{From momentum to position}

taking fourier transforms

\subsection{From Minkowski to Eulidean space}

Osterwalder-Schrader

analytic continuation of Wightman functions vs. Wick rotation of T-products

reflection positivity

note that this is usually done in the other direction!

\subsection{Embedding-space formalism}

conformal group acts linearly in embedding space with $d+2$ dimensions, among which two times

$X'^M = \Lambda^M_{~N} X^N$

get rid of two dimensions by restricting to light-cone $X^2 = 0$, and identifying $X^+ \sim \lambda X^+$


can take $X^+ = 1$, so that
\begin{equation}
	(X^+, X^-, X^\mu) = (1, x^2, x^\mu)
\end{equation}


fields defined on the cone, depending homogeneously on $X$

rewrite 2-point function


see section 2.1 of Rychkov

\subsection{3-point functions}

scalar 3-point function

remarkable fact! fixed up to a single coefficient

if only scale invariance, this could be arbitrarily complicated

historical note (cite Rychkov, around eq. 2.43): birth of conformal field theory


check: 2-point functions involving conserved currents and energy-momentum tensors are automatically conserved; for 3-point functions this gives additional constraint: scaling dimensions of operators must be equal



\subsection{4-point functions}
% and higher?

invariant cross-ratios for 4-point functions

%%%%%%%%%%%%%%%%%%%%%%%%%%%%%%%%%%%%%%%%%%%%%%%%%%%%%%%%%%%%%%%%%%%%%%%

\section{State-operator correspondence and OPE}


\subsection{Radial quantization}

different foliation of space-time than the one discussed above

in Euclidean, take the generator of dilatations as Hamiltonian


(note: this is equivalent to taking $N-S$ quantization with Hamiltonian $H = P^0 + K^0$)

same Hilbert space on a slice at $t = 0$; use different evolution operator


equivalence between state living on a sphere (arbitary radius) and operators inserted at the origin

organize states as eigenstates of scale and Lorentz (commuting): matches the construction we did above!

this shows that every state can be created by local primary operator (and possibly action of translation generators); conversely, every operator defines a state (this part was already clear)

not necessarily so in QFT! (e.g.~gauge theories? do we also need non-local operators? Wilson lines?)





important remark on OPE convergence in Euclidean space!

%%%%%%%%%%%%%%%%%%%%%%%%%%%%%%%%%%%%%%%%%%%%%%%%%%%%%%%%%%%%%%%%%%%%%%%

\section{The conformal bootstrap}

\subsection{Conformal blocks}

\subsection{The numerical boostrap}

generalized free fields/Gaussian

\subsection{Results}


discussion about universality: see Poland, Rychkov, Vichi

%%%%%%%%%%%%%%%%%%%%%%%%%%%%%%%%%%%%%%%%%%%%%%%%%%%%%%%%%%%%%%%%%%%%%%%

\section{Selected advanced topics}

light-cone limit and ?

Virasoro symmetry in 2d:


conformal anomalies: anomalies are contact term in the action for the source


superconformal bootstrap? note classification of all superconformal algebras


\subsection{How to continue from here}


\begin{itemize}

\item
holography: see Joao's TASI lectures

\item
superconformal: Tajikawa pedestrian lectures

\item
bootstrap: Shai Chester's lectures

\item
condensed matter:
note that unitarity is unnecessary: many non-unitary fixed points, yet they have conformal invariance (why?)

\end{itemize}



%\bibliography{Bibliography}
%\bibliographystyle{utphys}

\end{document}
