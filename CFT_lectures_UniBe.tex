\documentclass[a4paper,12pt]{article}

%\usepackage[left=25mm, right=25mm, top=30mm, bottom=30mm]{geometry}
\usepackage[utf8]{inputenc}
\usepackage[english]{babel}
\usepackage{amsmath}
\usepackage{amsfonts}
\usepackage{mathrsfs}
\usepackage{dsfont}
\usepackage{graphicx}
\usepackage{cite}
\usepackage[colorlinks, allcolors=blue, linktocpage=true]{hyperref}

%\renewcommand{\arraystretch}{1.5}

%\renewcommand{\O}{\mathcal{O}}
\newcommand{\ket}[1]{\left| #1 \right\rangle}
\newcommand{\bra}[1]{\left\langle #1 \right|}
\newcommand{\SO}{\text{SO}}


% double angle-bracket notation:
%\makeatletter
%\newsavebox{\@brx}
%\newcommand{\llangle}[1][]{\savebox{\@brx}{\(\m@th{#1\langle}\)}%
%  \mathopen{\copy\@brx\kern-0.5\wd\@brx\usebox{\@brx}}}
%\newcommand{\rrangle}[1][]{\savebox{\@brx}{\(\m@th{#1\rangle}\)}%
%  \mathclose{\copy\@brx\kern-0.5\wd\@brx\usebox{\@brx}}}
%\makeatother

\numberwithin{equation}{section}


%%%%%%%%%%%%%%%%%%%%%%%%%%%%%%%%%%%%%%%%%%%%%%%%%%%%%%%%%%%%%%%%%%%%%%%

\title{%
Conformal Field Theory
\\[1em]
\Large
Lecture notes}

\author{Marc Gillioz}

\date{Spring semester 2022}


\begin{document} 

\maketitle

\vspace{10mm}	
	
\begin{center}
	
\parbox{\textwidth}{%
%Conformal field theory (CFT) is an ubiquitous subject in modern theoretical physics. Every local quantum field theory approaches a CFT in the large- and small-distance limits, and even the study of quantum gravity is related to it through the AdS/CFT correspondence. CFT is also one of the rare frameworks in which quantum field theory can be studied outside the realm of perturbation theory.

This is an introductory course in which the students will learn what is conformal field theory, why it is special, and get a glimpse of modern developments. The course will begin with a non-perturbative formulation of quantum field theory (Wightman functions, spectral representation), and then gradually focus on understanding the implications of scale and special conformal symmetry. The study of practical tools (embedding space formalism, radial quantization, state-operator correspondence, conformal blocks) will finally lead to the formulation of the modern conformal bootstrap and a review of its most recent results. Some advanced topics will be discussed depending on the students' interests (Virasoro symmetry in 2 dimensions, UV and IR divergences, Mellin representation, superconformal symmetry).}
	
\vspace{20mm}
	
\parbox{11.8cm}{
\emph{%
[Courses] are fantastically good for learning physics. The lecturer learns a lot of physics. After my first few studies, just about everything I learned about physics came from teaching it. I don’t know if the students learned a lot, but I certainly did. So I consider teaching physics very important.}
--- Leonard Susskind \cite{Susskind}}

\end{center}

%%%%%%%%%%%%%%%%%%%%%%%%%%%%%%%%%%%%%%%%%%%%%%%%%%%%%%%%%%%%%%%%%%%%%%%

\newpage

\tableofcontents

%%%%%%%%%%%%%%%%%%%%%%%%%%%%%%%%%%%%%%%%%%%%%%%%%%%%%%%%%%%%%%%%%%%%%%%

\newpage
\section{Introduction}

\begin{equation}
	\text{conformal field theory}
	= \text{quantum field theory} + \text{conformal symmetry}
\end{equation}


%\begin{equation}
%	\text{CFT}
%	= \text{QFT} 
%	+ \text{Poincaré}
%	+ \text{Lorentz}
%	+ \text{scale}
%	+ \text{special conformal symmetry}
%\end{equation}
where Poincaré symmetry means translations and Lorentz transformations (rotations and boosts)

\begin{equation}
	\text{conformal field theory}
	= \text{relativistic quantum field theory} 
	+ \text{scale}
	+ \text{special conformal symmetry}
\end{equation}

most reviews focus first on conformal symmetry as a whole 
``geometric'' approach to CFT

for good reasons!
but often makes it difficult to connect with standard understanding of relativistic quantum field theory that we have from particle physics

things that are hard to connect to:
- spectral representation, particles (is CFT a theory of massless particles? no!)
- UV/IR divergences and anomalies
these are precisely subjects that will be covered in these lectures

peculiarity of these lecture notes:
``algebraic'' approach to CFT
review non-perturbative approach of QFT, then add scale and special conformal symmetry;
``anatomical'' approach allows to understand what are the consequences of each of them




%%%%

possible motivation: study scale transformation to get Wilsonian RG

why not local scale transformations that respect Poincare symmetry?

Jacobian of conformal symmetry is composition of dilatation and rotation (see Rychkov's Lorentzian)

%%%%

massive $\phi^4$ theory in $d = 3$;
statistical Ising model;
critical point of boiling water (and other liquids);

same critical exponents!

universality of IR behavior is a hint that the space of CFTs is sparse

unlike QFT, where you can always add/remove some operator in your favorite Lagrangian

both Euclidean and Lorentzian!!!


bootstrap philosophy: focus on CFT, not on microscopic details

in the framework of quantum field theory, push mathematical understanding of conformal symmetry to the limit, with amazing results


non-perturbative;
in fact, another strong motivation to study CFT is that it gives a mathematically rigorous definition of a QFT (a special one, but anyway);
no need for ``quantization'' of classical Lagrangian theory (in fact no classical limit!)

%also sometimes the only approach: theory with $(2,0)$ extend supersymmetry in $d = 6$ dimensions



\subsection{What is conformal field theory?}

Introduction: What is CFT? Why is it useful. Examples of CFT. Where does it fit into modern theoretical physics.


in the last few years, CFT has been dominated by 

string theory: 2d CFT

holography: geometric approach

condensed matter physics: Euclidean


here focus on the ``old'' quantum field theory approach

links with: lattice, perturbation theory, etc.

fun fact: the conformal bootstrap was invented by particle physicists


\textbf{strongly-coupled QFT}

no need for action principle


\subsection{Examples of conformal field theories}

perturbative examples: $\phi^n$ theory in non-integer $d$

Caswell-Banks-Zaks

superconformal: $\mathcal{N} = 4$


any IR fixed point (maybe trivial)


use beyond CFT:
correlators that have the isometries of the conformal group
gravity in AdS
but also late-time correlators in de Sitter

\subsection{Outline}

originally covered in 14 periods of 45 minutes each

split into 7 chapters?


\subsection{Literature}


excellent reviews, in order of relevance for the present course:


modern reviews, by conformal bootstrap experts
\begin{itemize}

\item
Slava Rychkov's EPFL lectures
\cite{Rychkov:2016iqz}
also historical references

\item
David Simmons-Duffin's TASI lectures
\cite{Simmons-Duffin:2016gjk}

\item
Shai Chester's Weizmann Lectures
\cite{Chester:2019wfx}


\end{itemize}

%%%%%%%%

other reviews:

\begin{itemize}

\item
state-of-the art
\cite{Poland:2018epd}

\item
Hugh Osborn's course 
\url{https://www.damtp.cam.ac.uk/user/ho/CFTNotes.pdf}

\item
Joshua Qualls \cite{Qualls:2015qjb}

\end{itemize}

%%%%%%%%

most Euclidean CFT, for Lorentzian perspective:
\begin{itemize}

\item
Slava Rychkov's \emph{Lorentzian methods in conformal field theory}
\url{https://courses.ipht.fr/node/226}

%\item
%Peskin Schroeder for spectral rep?

\end{itemize}

%%%%%%%%

focus on CFT in $d = 2$:
\begin{itemize}

\item
Sylvain Ribault \cite{Ribault:2014hia} \\
also available on GitHub~\url{https://github.com/ribault/CFT-Review}

\item
Schellekens ``Conformal Field Theory'' lecture notes~\cite{Schellekens:1996tg} \\
recent version available at
\url{https://www.nikhef.nl/\textasciitilde t58/CFT.pdf}

\end{itemize}

also older literature on the subject  in $d = 2$:
\begin{itemize}

\item
Yellow book
\cite{DiFrancesco:1997nk}

\item
Polchinski String theory vol. 1
\cite{Polchinski:1998rq}

\item
older lectures by Paul Ginsparg
\cite{Ginsparg:1988ui}
emphasis on statistical physics and string theory applications

\end{itemize}


going further:
\begin{itemize}

\item
superconformal symmetry: Lorenz Eberhardt 
\cite{Eberhardt:2020cxo}

\end{itemize}


%%%%%%%%%%%%%%%%%%%%%%%%%%%%%%%%%%%%%%%%%%%%%%%%%%%%%%%%%%%%%%%%%%%%%%%

\section{Classical conformal transformations}


Poincar\'e symmetry: physics is the same in every coordinate frame

linear transformation of the coordinates
\begin{equation}
	x'^\mu = x^\mu + \omega^{[\mu\nu]} x_\nu + a^\mu
\end{equation}


infinitesimal line element $dx^2 = \eta_{\mu\nu} dx^\mu dx^\nu$ is invariant:
\begin{equation}
	dx'^2 = dx^2
\end{equation}


scale symmetry:
\begin{equation}
	x'^\mu = \lambda x^\mu
\end{equation}

note that this can be seen as a rescaling of the metric, of the form
$\eta'^{\mu\nu} = \# \eta^{\mu\nu}$


can we generalize this? general transformations of the metric $\eta'^{\mu\nu} = \Omega(x) \eta^{\mu\nu}$
called a \emph{Weyl transformation}

in general changes geometry of space-time; unless this transformation is equivalent to a coordinate transformation (diffeomorphism)

in infinitesimal form
\begin{equation}
	x'^\mu = x^\mu + v^\mu(x)
\end{equation}
requiring that this is equal to the Weyl transformation gives the conformal Killing equation:
\begin{equation}
	\partial^\mu v^\nu(x) + \partial^\nu v^\mu(x)
	= 2 \sigma(x) \eta^{\mu\nu}
\end{equation}

work out solution in $d = 1$, $d = 2$ and $d > 2$;


conformal Killing vectors (see Osborn's notes):
translation
Lorentz transformations
dilatation
special conformal symmetry



in $d = 2$, use $z$, $\bar{z}$ notation and Laurent series






finite transformations


use inversions: discrete transformations not connected to identity
(no infinitesimal form!)

but inversion followed by translation followed by inversion gives special conformal transformations


group of conformal transformation is a subgroup of diffeomorphisms; in fact the largest \emph{finite-dimensional} subgroup
% Rychkov (1.53)



defining property
\begin{equation}
	g'_{\mu\nu}(x') = \Omega(x)^2 \eta_{\mu\nu}
\end{equation}
gives Jacobian of conformal transformation
\begin{equation}
	\frac{\partial x^\mu}{\partial x^\nu}
	= \Omega(x) M^\mu_{~\nu}(x)
\end{equation}
$M$: $\SO(d-1, 1)$ matrix

conformal transformations look locally like Lorentz and scale transformations



translation move 0, leave $\infty$ fixed

special conformal transformation move $\infty$, leave 0 fixed


finite transformation derived from infinitesimal: for rotations, translations and scale exactly the same;
for sct
\begin{equation}
	x'^\mu = \frac{x^\mu - x^2 b^\mu}
	{1 - 2 b \cdot x + b^2 x^2} 
\end{equation}
or 
\begin{equation}
	\frac{x'^\mu}{x'^2} = \frac{x^\mu}{x^2} - b^\mu
\end{equation}

\subsection{The conformal algebra}

follow Shester

generators corresponding to Killing vectors

when applied to function of $x$, gives derivative action



compute commutators, find Lorentz and Poincaré subalgebras, and new commutations involving $D$ and $K$

isomorphic to $\SO(d, 2)$

(exercise)

compact vs.~non-compact


note: everything we said is also valid in Euclidean space: replace $\eta^{\mu\nu} \to \delta^{\mu\nu}$

get the Euclidean conformal group


\subsection{Classical field theory}



\subsection{Energy-momentum tensor}

conserved current associated with symmetry

use the metric as source

condition for Weyl invariance is tracelessness of energy-momentum tensor


\subsection{Examples}

conformal transformations map circles to circles (including circles with infinite radius, aka straight lines): explanation

picture

think of it as renormalization group transformation that depends on space-time

Escher?




%%%%%%%%%%%%%%%%%%%%%%%%%%%%%%%%%%%%%%%%%%%%%%%%%%%%%%%%%%%%%%%%%%%%%%%

\section{Quantum conformal symmetry}


Wightman axioms

operators are not necessarily invariant under conformal symmetry, but they transform unitarily

unitary representations on Hilbert space

(e.g.~pair of point-like particle

no action principle
(i.e.~no need for ``quantization'')


only deal with \textbf{local} operators

warning about meaning of locality: related to causality?



generator of translations plays a special role: energy and momentum vector

positivity of energy 



note: assuming symmetry are there at the quantum level, there exists a traceless energy-momentum tensor and generators of conformal transformations can be constructed from them

(this only goes in one direction: given $T$, one can construct generators; but given the symmetry, Noether's theorem only gives a conserved current if there is a Lagrangian, it is unproven for non-lagrangian theories; additional condition of existence of energy-momentum tensor is sometimes called \emph{locality})

these generators are non-local!
but their commutators with local operators is local (see Simmons-Duffin)

discuss connection with path integral formalism (in an exercise? or a parenthesis?)


\subsection{Non-perturbative QFT}

%\subsection{Hilbert space(s)}

take $P^0$ as the Hamiltonian

space-time is foliated by surfaces of equal time

each equal-time surface has a Hilbert space; evolution between different surfaces at times $t_1$ and $t_2$ is given by the unitary evolution operator $U = e^{\pm i (t_1 - t_2) P^0}$

because of time-translation symmetry, the Hilbert space is the same on each surface!

states characterized by energy and momentum: $P^\mu \ket{p} = p^\mu \ket{p}$

there is a unique vacuum state: $\ket{0}$

there are local operators: an infinity of them
example: free scalar theory

%\subsection{Wightman functions}


$\phi(x)$ are not operators, but operator-valued distributions

smeared operator $\phi[f]$ have finite norm; $\phi(x)$ don't

but don't worry for the rest of these lectures


positivity of norm of smeared operators implies positivity of thing multiplying delta function!




good observable: Wightman correlation function
(note: operator need not be time-ordered; time evolution is unitary, so it can go both ways!)

operators define correlation functions, but the opposite is also true (Wightman reconstruction theorem); so use the latter and forget about Hilbert space



but then lose connection with path integral formulation: this gives time-ordered correlation functions!



representation theory: follow Shester's notes
commutation relations for infinitesimal transformations
exponentiated form is an abstract unitary operator

forget about unitary rep of SCT: only act on Lorentz cylinder, infinite cover of Minkowski space-time; we know this rep exists
but keep infinitesimal form




Wigner construction: use reference momentum


generator of Lorentz transformation can be diagonalize at a point: the origin of space-time

get the action of $M^{\mu\nu}$ at arbitrary point



conformal Ward identities



%\subsection{Spectral representation}

operator vs.~field

\subsection{Scale symmetry}

action of generator of translations

no mass (often said)

or better said: all masses!

reference vector, then construct spectral density


remark: only operator that is a scalar with scaling dimension zero can have a one-point function



\subsection{Special conformal symmetry}

note: finite special conformal transformations brings us outside of Minkowski space-time;
but no need for them: only infinitesimal form

special conformal generator implies two things:

primary and descendant operators

special condition on 2-point function of primary operators: that they have identical scaling dimensions


%%%%%%%

note that scale invariance and unitarity generally imply that the trace of the energy-momentum tensor is zero: then Hamiltonian is invariant under change of the metric $\delta g_{\mu\nu} = \sigma(x) \eta_{\mu\nu}$,
\begin{equation}
	\delta H = \int d^dx \, T_{\mu\nu} \delta g^{\mu\nu}
	= \int d^x \, \sigma(x) T^\mu_\mu = 0
\end{equation}
invariance under infinitesimal Weyl transformations 

note that $T_{\mu\nu}$ is only defined in flat space, or rather that it acquires a vacuum expectation value in curved space: the Weyl anomaly
so let us only consider Weyl transformations that do not change the geometry of space-time: those that are equivalent to a coordinate transformation (a diffeomorphism)


exercise: compute conformal generators in momentum-space representation


Mack's classification in 4d


representation: long and short multiplets!



%%%%

compare this with Slava's approach:

organize operators into representation of conformal group:

call some operators \emph{primary}

operators are local, which mean that they do not feel the effect of the transformation of the metric, or only through coordinate dependence:
$\phi(x) \to \phi(x') = \Omega(x)^{-\Delta} \phi(x)$

note: if $\phi$ transforms like this, then $\partial_\mu \phi(x)$ does not! call it a \emph{descendant}


comment on scale vs.~conformal invariance

correlation functions:

no need for action principle


\subsection{Time-ordered products and path integrals}


Simmons-Duffin intro


\subsection{The energy-momentum tensor}


locality

allows to construct conformal charges


\subsection{Exercises}

show that the spectral density in the case $\Delta = \frac{d}{2} - 1$ is that of a free scalar field;

show also that in the case $\Delta = \frac{d}{2}$ it is given by the phase space of two massless particle states (see Rychkov eq. 1.13)


%%%%%%%%%%%%%%%%%%%%%%%%%%%%%%%%%%%%%%%%%%%%%%%%%%%%%%%%%%%%%%%%%%%%%%%

\section{Conformal correlators}

correlation functions aka Green's functions

remark about ``physicality''
in ordinary QFT, correlation functions are not physical observables: they depend on renormalization scale; only scattering amplitudes are!
here no scale: they are physical


2-pt function: go from Wightman 2-pt function in momentum space to Euclidean 2-pt function in position space in two ways:

- Fourier transform first, then analytic continuation of Wightman function
- first construct T product (Källen-Lehmann representation), then Wick rotation to Euclidean momentum space, then Fourier transform

%%%%%


T-products vs. Wightman functions

use retarded products and micro-causality


Ward identities

more identities for conserved currents and the energy-momentum tensor
(there cannot be higher-spin conserved currents otherwise the theory must be free; reference?)


generalities:

one-point functions vanish (in flat space; not so on conformally flat manifolds such as $S^d$, where 1-point function might depend on radius)

\subsection{Spectral representation for spinning operators}

unitary bounds

\subsection{From momentum to position}

taking fourier transforms

\subsection{From Minkowski to Eulidean space}

analytic continuation of Wightman functions vs. Wick rotation of T-products

reflection positivity


going in the other direction: Osterwalder-Schrader


do we need to use the cylinder interpretation?
I prefer not: only CFT in flat space here



\subsection{Embedding-space formalism}

conformal group acts linearly in embedding space with $d+2$ dimensions, among which two times

$X'^M = \Lambda^M_{~N} X^N$

get rid of two dimensions by restricting to light-cone $X^2 = 0$, and identifying $X^+ \sim \lambda X^+$


can take $X^+ = 1$, so that
\begin{equation}
	(X^+, X^-, X^\mu) = (1, x^2, x^\mu)
\end{equation}


fields defined on the cone, depending homogeneously on $X$

rewrite 2-point function


see section 2.1 of Rychkov

\subsection{3-point functions}

scalar 3-point function

remarkable fact! fixed up to a single coefficient

if only scale invariance, this could be arbitrarily complicated

historical note (cite Rychkov, around eq. 2.43): birth of conformal field theory


check: 2-point functions involving conserved currents and energy-momentum tensors are automatically conserved; for 3-point functions this gives additional constraint: scaling dimensions of operators must be equal



\subsection{4-point functions}
% and higher?

invariant cross-ratios for 4-point functions

%%%%%%%%%%%%%%%%%%%%%%%%%%%%%%%%%%%%%%%%%%%%%%%%%%%%%%%%%%%%%%%%%%%%%%%

\section{State-operator correspondence and OPE}


goal of this section: prove state/operator correspondence heuristically, justifying a posteriori some facts assumed before

not only a curiosity: also useful computationally


\subsection{N-S quantization}

different foliation of space-time than the one discussed above



same Hilbert space on a slice at $t = 0$; use different evolution operator


conjugation in Euclidean space?

takes $x_d \to - x_d$

use Hamiltonian
\begin{equation}
	H = P_d + K_d
\end{equation}
instead of $P_d$

fixed points

operators can be evolved to fixed points by Hamiltonian


\subsection{Radial quantization}

in Euclidean, take the generator of dilatations as Hamiltonian

simple rotation in $\SO(d+1,1)$

apply conformal transformation mapping the two fixed points to 0 and $\infty$

conjugation is inversion $r \to 1/r$

generators $P_\mu$ and $K_\mu$ are conjugate under inversion!

raising and lowering operators


equivalence between state living on a sphere (arbitary radius) and operators inserted at the origin

organize states as eigenstates of scale and Lorentz (commuting): matches the construction we did above!

this shows that every state can be created by local primary operator (and possibly action of translation generators); conversely, every operator defines a state (this part was already clear)

not necessarily so in QFT! (e.g.~gauge theories? do we also need non-local operators? Wilson lines?)



how to define an operator from a state? see Rychkov (3.26)
operator is defined by all of its correlation functions



note: state/operator correspondence validates our interpretation of Wightman 2-point function

note 2: operator inside the unit sphere need not be smeared anymore; they have a well-defined norm!



exercise: unitarity bound in radial quantization (Rychkov section 3.2)



path integral interpretation: state is created by performing path integral inside unit sphere



\subsection{Conformal OPE}

derive OPE from radial quantization

relate this to Lorentzian OPE on vacuum


use in 3-point function: relates OPE coefficient with 3-pt fct coefficient
(structure constants of the operator algebra)

exercise: compute function entering the OPE


\subsection{OPE convergence}

Hilbert-space argument


radius? 

absolute convergence as long as one can find a sphere separating the points:

note that a plane is also a sphere with radius infinity


%%%


concluding remarks:


CFT data: spectrum of operators and OPE coefficients
completely determines the theory!

OPE reduces any higher-point function to 2-point


%%%%%%%%%%%%%%%%%%%%%%%%%%%%%%%%%%%%%%%%%%%%%%%%%%%%%%%%%%%%%%%%%%%%%%%

\section{The conformal bootstrap}

any CFT data defines a good theory? no!

different quantization surfaces gives different OPEs in the same theory

OPE associativity

often called ``crossing symmetry'' (note: quite different from crossing symmetry of scattering amplitudes)

(in fact this argument shows that Schwinger functions are symmetric!)

focus on 4-point functions:
all 4-point functions of all operators contain all the associativity constraints


simplifications:
only 4 identical operators, all scalars


\subsection{Conformal blocks}

rather conformal partial waves

diagrammatics (not to be confused with Feynman diags)



square configuration and expansion in rho coordinates



\subsection{The numerical boostrap}

discussion about universality: see Poland, Rychkov, Vichi
also Shai Chester's notes



bound on scaling dimensions

also spectrum at the boundary


\subsection{Generalized free field theory}

aka mean free theory, Gaussian theory

double-twist operator

corresponds to large-N limit of gauge theory


\subsection{Analytic bootstrap}

Euclidean $z^* = \bar{z}$
$z = \bar{z} = \frac{1}{2}$ gives bootstrap equations that constrain operators of low scaling dimension $\Delta$

Minkowski: $z \neq \bar{z}$, both real 
light-cone limit $z \to 0$, $\bar{z}$ fixed is dominated by operators of low twist 

rigorous bound in the limit $\ell \to \infty$

Regge trajectories




%%%%%%%%%%%%%%%%%%%%%%%%%%%%%%%%%%%%%%%%%%%%%%%%%%%%%%%%%%%%%%%%%%%%%%%

\section{Selected advanced topics}

light-cone limit and ?

Virasoro symmetry in 2d:


conformal anomalies: anomalies are contact term in the action for the source


superconformal bootstrap? note classification of all superconformal algebras


\subsection{The 2d conformal bootstrap}

historical presentation

see Rychkov section 4.3.1

advantage 1:
full Virasoro symmetry

advantage 2:
easier to deal with spin: left- and right-moving, i.e.~holomorphic and anti-holomorphic

by looking at just the energy-momentum tensor, already get strong constraints on its 2-point function (the central charge)

for $c < 1$, only minimal models possible:
match Ising, \ldots

finitely many primaries in 2d (still infinitely many quasi-primaries)

note: infinitely many primaries for theories with $c > 1$

also can put the theory on torus and study modular properties of partition function


\subsection{How to continue from here}


\begin{itemize}

\item
holography: see Joao's TASI lectures

\item
superconformal: Tajikawa pedestrian lectures

\item
bootstrap: Shai Chester's lectures

\item
condensed matter:
note that unitarity is unnecessary: many non-unitary fixed points, yet they have conformal invariance (why?)

\end{itemize}



\bibliography{Bibliography}
\bibliographystyle{utphys}

\end{document}
