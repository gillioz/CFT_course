\documentclass[a4paper,12pt]{article}

%\usepackage[left=25mm, right=25mm, top=30mm, bottom=30mm]{geometry}
\usepackage[utf8]{inputenc}
\usepackage[english]{babel}
\usepackage{amsmath}
\usepackage{amsfonts}
\usepackage{mathrsfs}
\usepackage{dsfont}
\usepackage{graphicx}
\usepackage{cite}
\usepackage[colorlinks, allcolors=blue, linktocpage=true]{hyperref}

%\renewcommand{\arraystretch}{1.5}

%\renewcommand{\O}{\mathcal{O}}
%\newcommand{\ket}[1]{\left| #1 \right\rangle}
%\newcommand{\bra}[1]{\left\langle #1 \right|}
%\newcommand{\SO}{\text{SO}}


% double angle-bracket notation:
%\makeatletter
%\newsavebox{\@brx}
%\newcommand{\llangle}[1][]{\savebox{\@brx}{\(\m@th{#1\langle}\)}%
%  \mathopen{\copy\@brx\kern-0.5\wd\@brx\usebox{\@brx}}}
%\newcommand{\rrangle}[1][]{\savebox{\@brx}{\(\m@th{#1\rangle}\)}%
%  \mathclose{\copy\@brx\kern-0.5\wd\@brx\usebox{\@brx}}}
%\makeatother

\numberwithin{equation}{section}


%%%%%%%%%%%%%%%%%%%%%%%%%%%%%%%%%%%%%%%%%%%%%%%%%%%%%%%%%%%%%%%%%%%%%%%

\title{%
Conformal Field Theory
\\[1em]
\Large
Lecture notes}

%\title{Conformal Field Theory for Particle/High-Energy Physicists}

\author{Marc Gillioz}

\date{Spring semester 2022}


\begin{document} 

\maketitle

\begin{center}
	\parbox{11.5cm}{\emph{%
[Lectures] are fantastically good for learning physics. The lecturer learns a lot of physics. After my first few studies, just about everything I learned about physics came from teaching it. I don’t know if the students learned a lot, but I certainly did. So I consider teaching physics very important.} --- Leonard Susskind}
\end{center}
%https://www.delta.tudelft.nl/article/leonard-susskind-sleeping-during-lecture-good-thing#

%%%%%%%%%%%%%%%%%%%%%%%%%%%%%%%%%%%%%%%%%%%%%%%%%%%%%%%%%%%%%%%%%%%%%%%

\newpage

\tableofcontents

%%%%%%%%%%%%%%%%%%%%%%%%%%%%%%%%%%%%%%%%%%%%%%%%%%%%%%%%%%%%%%%%%%%%%%%

\newpage
\section{Introduction}

\subsection{What is conformal field theory?}

Introduction: What is CFT? Why is it useful. Examples of CFT. Where does it fit into modern theoretical physics.


in the last few years, CFT has been dominated by 

string theory: 2d CFT

holography: geometric approach

condensed matter physics: Euclidean


here focus on the ``old'' quantum field theory approach

links with: lattice, perturbation theory, etc.

fun fact: the conformal bootstrap was invented by particle physicists


\textbf{strongly-coupled QFT}

no need for action principle


\subsection{Examples of conformal field theories}

perturbative examples: $\phi^n$ theory in non-integer $d$

Caswell-Banks-Zaks

superconformal: $\mathcal{N} = 4$


any IR fixed point (maybe trivial)


use beyond CFT:
correlators that have the isometries of the conformal group
gravity in AdS
but also late-time correlators in de Sitter

\subsection{Outline}

originally covered in 14 periods of 45 minutes each

split into 7 chapters?

%%%%%%%%%%%%%%%%%%%%%%%%%%%%%%%%%%%%%%%%%%%%%%%%%%%%%%%%%%%%%%%%%%%%%%%

\section{Classical conformal symmetry}


Poincar\'e symmetry: physics is the same in every coordinate frame

linear transformation of the coordinates
\begin{equation}
	x'^\mu = x^\mu + \omega^\mu_{~\nu} x^\nu + a^\mu
\end{equation}


infinitesimal line element $dx^2 = \eta_{\mu\nu} dx^\mu dx^\nu$ is invariant:
\begin{equation}
	dx'^2 = dx^2
\end{equation}


scale symmetry:
\begin{equation}
	x'^\mu = \lambda x^\mu
\end{equation}




special conformal symmetry:


conformal Killing vectors:
(see Osborn's notes)
infinitesimal transformation
\begin{equation}
	x'^\mu = x^\mu + v^\mu(x)
\end{equation}

\begin{equation}
	\partial^\mu v^\nu + \partial^\nu v^\mu = 2 \sigma \eta^{\mu\nu}
\end{equation}

solution in $d = 1$, $d = 2$ and $d > 2$



finite transformations


inversions: discrete transformations not connected to identity

but inversion followed by translation followed by inversion

\subsection{The conformal algebra}


embedding space?

\subsection{Examples}

Escher?




%%%%%%%%%%%%%%%%%%%%%%%%%%%%%%%%%%%%%%%%%%%%%%%%%%%%%%%%%%%%%%%%%%%%%%%

\section{Non-perturbative quantum field theory}

Wightman axioms

operators are not necessarily invariant under conformal symmetry, but they transform unitarily

unitary representations on Hilbert space

(e.g.~pair of point-like particle

no action principle
(i.e.~no need for ``quantization'')


only deal with \textbf{local} operators


\subsection{Spectral representation}

operator vs.~field

\subsection{Scale symmetry}


comment on scale vs.~conformal invariance

correlation functions:

no need for action principle


\subsection{Special conformal symmetry}

Mack's classification in 4d


representation: long and short multiplets!


\subsection{Exercises}


\subsection{Bibliography}

itzikson zuber

%%%%%%%%%%%%%%%%%%%%%%%%%%%%%%%%%%%%%%%%%%%%%%%%%%%%%%%%%%%%%%%%%%%%%%%

\section{Conformal correlation functions}

T-products vs. Wightman functions

use retarded products and micro-causality


Ward identities


\subsection{Spectral representation for spinning operators}

unitary bounds

\subsection{From momentum to position}

taking fourier transforms

\subsection{From Minkowski to Eulidean space}

Osterwalder-Schrader

analytic continuation of Wightman functions vs. Wick rotation of T-products

reflection positivity

note that this is usually done in the other direction!

\subsection{Embedding-space formalism}

in position space

2- and 3-point functions

invariant cross-ratios for 4-point functions

%%%%%%%%%%%%%%%%%%%%%%%%%%%%%%%%%%%%%%%%%%%%%%%%%%%%%%%%%%%%%%%%%%%%%%%

\section{State-operator correspondence and OPE}

important remark on OPE convergence in Euclidean space!

%%%%%%%%%%%%%%%%%%%%%%%%%%%%%%%%%%%%%%%%%%%%%%%%%%%%%%%%%%%%%%%%%%%%%%%

\section{The conformal bootstrap}

\subsection{Conformal blocks}

\subsection{The numerical boostrap}

generalized free fields/Gaussian

\subsection{Results}


%%%%%%%%%%%%%%%%%%%%%%%%%%%%%%%%%%%%%%%%%%%%%%%%%%%%%%%%%%%%%%%%%%%%%%%

\section{Selected advanced topics}

light-cone limit and ?

Virasoro symmetry in 2d:


conformal anomalies: anomalies are contact term in the action for the source


%\bibliography{Bibliography}
%\bibliographystyle{utphys}

\end{document}
