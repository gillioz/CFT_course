\documentclass[a4paper,12pt]{article}

%\usepackage[left=25mm, right=25mm, top=30mm, bottom=30mm]{geometry}
\usepackage[utf8]{inputenc}
\usepackage[english]{babel}
\usepackage{amsmath}
\usepackage{amsfonts}
\usepackage{amssymb}
\usepackage{mathrsfs}
\usepackage{dsfont}
\usepackage{graphicx}
\usepackage{cite}
\usepackage{xcolor}
\usepackage{mdframed}
\usepackage[colorlinks, allcolors=blue, linktocpage=true]{hyperref}

\newcommand{\ket}[1]{\left| #1 \right\rangle}
\newcommand{\bra}[1]{\left\langle #1 \right|}
\newcommand{\Lagr}{\mathscr{L}}
\renewcommand{\O}{\mathcal{O}}
\DeclareMathOperator{\im}{\text{im}}
\newcommand{\SO}{\text{SO}}

% double angle-bracket notation:
%\makeatletter
%\newsavebox{\@brx}
%\newcommand{\llangle}[1][]{\savebox{\@brx}{\(\m@th{#1\langle}\)}%
%  \mathopen{\copy\@brx\kern-0.5\wd\@brx\usebox{\@brx}}}
%\newcommand{\rrangle}[1][]{\savebox{\@brx}{\(\m@th{#1\rangle}\)}%
%  \mathclose{\copy\@brx\kern-0.5\wd\@brx\usebox{\@brx}}}
%\makeatother

\numberwithin{equation}{section}

\newcounter{exercise}[section]
\newenvironment{exercise}[1][]%
	{\refstepcounter{exercise}\bigskip
	\begin{mdframed}[backgroundcolor=gray!20, linewidth=0]
	\noindent\textbf{Exercise~\thesection.\theexercise #1} \rmfamily}
  	{\end{mdframed}\bigskip}
\newcommand\hint[1]{\emph{Hint: #1}}


%%%%%%%%%%%%%%%%%%%%%%%%%%%%%%%%%%%%%%%%%%%%%%%%%%%%%%%%%%%%%%%%%%%%%%%

\title{%
Conformal Field Theory
\\[1em]
\Large
Lecture notes}

\author{Marc Gillioz}

\date{Spring semester 2022}


\begin{document} 

\maketitle

\vspace{10mm}	
	
This is the transcript of my notes for the class 
``\href{https://www.itp.unibe.ch/studies/graduate_program/f22_conformal_field_theory/index_eng.html}{Conformal Field Theory}'' at the Institute for Theoretical Physics of the University of Bern.
These notes will be updated as the class advances, with the latest version always available at:
\begin{center}
	\url{https://github.com/gillioz/CFT_course}
\end{center}
Suggestions for improvements are welcome and can be directly submitted on the GitHub page, or sent to me by email at \href{mailto:marc.gillioz@sissa.it}{marc.gillioz@sissa.it}.
	
%\parbox{\textwidth}{%
%Conformal field theory (CFT) is an ubiquitous subject in modern theoretical physics. Every local quantum field theory approaches a CFT in the large- and small-distance limits, and even the study of quantum gravity is related to it through the AdS/CFT correspondence. CFT is also one of the rare frameworks in which quantum field theory can be studied outside the realm of perturbation theory.

%This is an introductory course in which the students will learn what is conformal field theory, why it is special, and get a glimpse of modern developments. The course will begin with a non-perturbative formulation of quantum field theory (Wightman functions, spectral representation), and then gradually focus on understanding the implications of scale and special conformal symmetry. The study of practical tools (embedding space formalism, radial quantization, state-operator correspondence, conformal blocks) will finally lead to the formulation of the modern conformal bootstrap and a review of its most recent results. Some advanced topics will be discussed depending on the students' interests (Virasoro symmetry in 2 dimensions, UV and IR divergences, Mellin representation, superconformal symmetry).}



%%%%%%%%%%%%%%%%%%%%%%%%%%%%%%%%%%%%%%%%%%%%%%%%%%%%%%%%%%%%%%%%%%%%%%%

\newpage

\tableofcontents

%%%%%%%%%%%%%%%%%%%%%%%%%%%%%%%%%%%%%%%%%%%%%%%%%%%%%%%%%%%%%%%%%%%%%%%

\newpage
\section{Introduction}


\subsection{What is conformal field theory?}

Conformal field theory is a quantum field theory with conformal symmetry:
\begin{equation}
	\text{conformal field theory}
	= \text{quantum field theory} + \text{conformal symmetry},
\end{equation}
where under conformal symmetry we understand
\begin{itemize}

\item
translations in time and space,

\item
rotations and Lorentz boosts,

\item
scale transformations, also known as dilatations,

\item
special conformal transformations.

\end{itemize}
The course will begin in section~\ref{sec:classical} with a precise definition of all these symmetries, even though the first two (the Poincaré symmetry) should be familiar from an elementary quantum field theory class, and you should even have heard of the importance of dilatations in the context of the \emph{renormalization group}.

Most courses on CFT put the emphasis on conformal symmetry as a whole, which makes a lot of sense because it forms a group (more on this latter). However, this approach often requires to work in Euclidean space (we will see why), and the connection with quantum field theory as we know it appears very late in the course, if at all.

Instead, in this lecture we would like to stay as close as possible to quantum field theory; in some sense, we will defined conformal field theory as
\begin{equation}
\begin{aligned}
	\text{conformal field theory}
	&= \text{relativistic quantum field theory} 
	\\
	& \quad
	+ \text{scale symmetry}
	\\
	& \quad
	+ \text{special conformal symmetry},
\end{aligned}
\end{equation}
and go through this definition carefully.
The first element is quantum field theory in flat Minkowski space-time. We have a good intuition of this part from particle physics. However, we do not want to restrict our attention to theories that have a nice classical limit and an understanding in terms of perturbation theory!
Therefore, we will discuss in section~\ref{sec:quantum} the basics of a non-perturbative quantum field theory.
Even though we care about unitary quantum field theory in Minkowski space-time, it will be useful to establish a connection with the same theory defined in flat Euclidean space. For this reason, we will work with the Minkowski metric convention
\begin{equation}
	\eta_{\mu\nu} = \left( \begin{array}{cccc}
		-1 &&& \\ & 1 && \\ && \ddots & \\ &&& 1
	\end{array} \right)_{\mu\nu},
	\qquad\quad
	\mu, \nu = 0, 1, \ldots, d-1,
\end{equation}
so that going from Minkowski space-time to Euclidean space will be achieved through a rotation of the time coordinate in the complex plane, $x^0 = t \to i \tau$.
Since for us the space-time will always be flat (no matter whether Minkowski or Euclidean), we will raise and lower the indices at will.

After giving a non-perturbative definition of relativistic quantum field theory, we will examine the role of scale symmetry. The main novelty here is that scale invariance forbids the presence of particles with a definite mass; instead, in a scale-invariant theory there are excitations of \emph{any} energy (unless we are in the very special case of a free theory).

Finally, special conformal symmetry will be discussed last. The point is that it (nearly always) come along with scale symmetry. Nevertheless, it provides extremely powerful tools to understand the theory from a non-perturbative point of view. This will lead us to a comprehensive study of conformally-invariant correlation functions in section~\ref{sec:correlators}, and to the formulation of an \emph{operator product expansion} (OPE) that has very nice features in section~\ref{sec:OPE}.

%- UV/IR divergences and anomalies 
%these are precisely subjects that will be covered in these lectures


\subsection{Why is conformal field theory interesting?}

Nature is certainly not scale invariant, so why do we care so much about conformal field theory?

First of all, scale transformations play an essential role in quantum field theory, even when this is not a symmetry of the theory. The Callan-Symanzik equation in renormalization is essentially a Ward identity associated with the (broken) scale symmetry. 
We will see that conformal transformations can be viewed as a \emph{local} generalization of scale transformations, in which the rescaling factor is different at every point in space-time, while still being consistent with Poincaré symmetry.

We also know that all end points of renormalization group trajectories are described by scale-invariant fixed points, and it turns out that \emph{all} known fixed points are in fact conformally invariant. Therefore, conformal field theory is automatically relevant for the description of the low- and high-energy limits of any quantum field theory.

But one of the main reasons why it is important to study conformal field theory now (or at least have a basic understanding of it) is that it has been yielding exciting results in the last 10-15 years. In section~\ref{sec:bootstrap}, we will give a short review of the  \emph{conformal bootstrap}, a technique that allows to completely solve conformal field theory in certain cases.
Even though the modern conformal bootstrap was invented to solve a particle physics problem, its results so far have had more impact in condensed matter physics. There, the concept of \emph{universality} is playing a central role: it turns out that the critical exponents of very different systems coincide. This has a simple explanation from the fact that there are only ``few'' conformal field theories that can exist at all (at least once global symmetries are specified together with some information about what are the relevant operators). In other words, the space of all conformal field theories is sparse, unlike in QFT where it is always possible to add some new operator in a Lagrangian to slightly deform the theory.
This concept is extremely successful in $d = 2$ and 3 dimensions, but there is also a big hope to classify all possible conformal field theories in $d = 4$ (or higher) dimensions. 
This goal would have a huge impact also for QFT with scale dependence, because it would strongly constrain the possible renormalization group flow.
But this is still a young field, and much more work needs to be done before anything like this can be claimed.

Besides its importance in quantum field theory and condensed matter physics, conformal field theory has also given very nice insights into:
\begin{itemize}

\item
Quantum gravity, through the AdS/CFT correspondence.

\item
String theory, as the worldsheet action is a CFT in $d = 2$.

\item
Foundations of quantum field theory beyond the perturbative approach.
CFT admits a mathematically rigorous definition, which does not require to ``quantize'' a classical Lagrangian theory.%
%
\footnote{In fact there needs not be a classical limit; for instance, in $d = 6$ dimensions there exists a theory with $(2,0)$ extended supersymmetry for which no Lagrangian can be written down. The theory is known to exists from string theory arguments, and it has also been ``observed'' in a conformal bootstrap setup.}
%

\end{itemize}


\subsection{Examples of conformal field theories}

Conformal field theories are everywhere in physics, and not necessarily limited to very special situations. Here are some examples, some of which you may already have encountered:%
%
\begin{itemize}

\item
Free massless theories are conformal. This is the case of the free boson and free fermion in any number of dimensions, but also of the $n$-form gauge theory in $d = 2n + 2$ dimensions (e.g.~the free vector theory in $d = 4$). The same is true of theories with any number of free fields.

\item
Theories in which the $\beta$-function admits a perturbative fixed points. Typical examples are deformations of the free scalar theory with a potential of the type $\phi^n$. For instance, consider
\begin{equation}
	S = \int d^dx \left[
	- \frac{1}{2} \partial_\mu \phi \partial^\mu \phi
	- \frac{g}{4!} \phi^4 \right].
\end{equation}
$g$ is dimensionless in $d = 4$ dimensions. If one considers the theory defined in $d = 4 - \varepsilon$ dimensions, then the beta function for $g$ is
\begin{equation}
	\beta_g = \mu \frac{dg}{d\mu}
	= - \varepsilon + \frac{3g}{(4\pi)^2}
	+ \O(g^2).
\end{equation}
It vanishes when
\begin{equation}
	\frac{g_*}{(4\pi)^2} = \frac{\varepsilon}{3}.
\end{equation}
Corrections of higher order in $g$ can be neglected in the limit $\varepsilon \ll 1$.
There are similar fixed points for a $\phi^3$ interaction in $d = 6 + \varepsilon$ dimensions, of for a $\phi^6$ interaction around $d = 3$.


\item
A similar type of fixed point can be found in the beta function of a $SU(N_c)$ gauge theory with $N_f$ fermions (in the fundamental representation), which is given at leading order in the gauge coupling $\alpha = g^2/(4\pi)^2$ by
\begin{equation}
	\beta_\alpha = \mu \frac{d\alpha}{d\mu}
	= -\frac{2}{3} \alpha^2 \left(11 N_c - 2 N_f \right)
	+ \O(\alpha^3).
\end{equation}
When $N_f = \frac{11}{2} N_c$, the leading order term in this beta function vanishes, so the next-to-leading term becomes important. Around that value, the first two terms are of similar importance, and one finds a perturbative fixed point of when $N_f \lesssim \frac{11}{2} N_c$. This is called the (Caswell-)Banks-Zaks fixed point. A theory in this situation is asymptotically free like QCD, but in the low-energy limit it approaches an interacting conformal field theory.
For an $SU(3)$ gauge theory like QCD this critical value is at $N_f = 16.5$. There is strong evidence from lattice simulation that a theory with $N_f = 16$ is conformal. On the other hand a theory with low $N_f$ (QCD has only 3 light quarks) is clearly confining, meaning that its low-energy limit is a theory massless Goldstone bosons (if the quarks are massless), or of massive pions. There is a critical value $N_f^* \approx 12$ above which we expect an interacting conformal field theory in the low-energy limit. The domain $N_f^* \leq N_f \leq \frac{11}{2} N_c$ is called the \emph{conformal window}. Note that gauge theories with different gauge groups and/or fermions coupling differently to the gauge fields (i.e.~in different representations) can also have a conformal window. It is even possible to engineer gauge theories with perturbative UV fixed points using not only fermions but also scalars.

\item
There are theories with extended supersymmetry in which the beta function is exactly zero at all orders in perturbation theory, for instance $\mathcal{N} = 4$ supersymmetric Yang-Mills. More generically, in theories with sufficiently many supersymmetries, it is often sufficient to engineer the matter content to make the beta function zero at leading order, and then non-renormalization theorems ensure their vanishing at all orders. Theories with conformal symmetry and supersymmetry are called \emph{superconformal}.

\item
An example of truly non-perturbative fixed point in $d = 3$ dimensions is given by the following action:
\begin{equation}
	S = \int d^3x \left[
	-\frac{1}{2} \partial_\mu \phi \partial^\mu \phi
	- \frac{1}{2} m^2 \phi^2 - \frac{g}{4!} \phi^4 \right],
\end{equation}
in which $g > 0$ for the potential to be bounded below.
Note that the free scalar field has mass dimension $\left[ \phi \right] = \frac{1}{2}$ in $d = 3$, and therefore
\begin{equation}
	\left[ m^2 \right] = 2,
	\qquad\qquad
	\left[ g \right] = 1.
\end{equation}
Since both $m^2$ and $g$ have positive mass dimensions, they are \emph{relevant} operators: they determine the dynamics in the low-energy limit (IR), but their importance decreases at higher and higher energies (UV): at energies $E \gg g, |m|$, this is approaches the theory of a massless free scalar.
On the other hand, at low energies the physics depends obvioulsy on $g$ and $m$: when $m^2 \gg g^2$, it is a theory of a massive scalar of mass $m$, whereas when $m^2 \ll -g^2$, then the potential has two minima at
\begin{equation}
	\langle \phi \rangle = \pm \sqrt{- \frac{6m^2}{g}},
\end{equation}
with excitations fo mass $\sqrt{2} |m|$ around it. Clearly this theory has two phases, and there must therefore be an intermediate value of $m^2$ where the phase transition happens (or working in units set by $g$, a specific value of the dimensionless ratio $m^2/g^2$). Note that this theory has a $\mathds{Z}_2$ symmetry in the UV corresponding to $\phi \to -\phi$, which is spontaneously broken in one phase and not in the other.

It turns out that the theory describing the IR physics exactly at the phase transition is a conformal field theory. Unlike the two phases surrounding it, it admits excitations of arbitrarily small energies (but they are not particles). How do we know that? Feynman diagram computations cannot be trusted in the IR: the approximation given by the (asymptotic) perturbative series is valid in the UV, but it breaks down in the IR, as in QCD. One way of understanding the phase transition is examining the theory in $d \neq 3$ dimensions: the same theory in $d = 4 - \varepsilon$ expansion has a perturbative fixed point, and computations at the fixed point can be performed in an expansion in $\varepsilon$. The value of the fixed point depends on the renormalization scheme, but there are other quantities that are scheme-independent. This is for instance the case of the anomalous dimension of the operator $\phi$.
Using a 6-loop beta-function computation, one finds
\begin{equation}
	\gamma_\phi \approx 0.0182.
\end{equation}
This is more conveniently expressed as a \emph{scaling dimension} of $\phi$,
\begin{equation}
	\Delta_\phi = \frac{d-2}{2} + \gamma_\phi \approx 0.5182,
\end{equation}
which in position space describes how the correlation between two points decays with the distance
\begin{equation}
	\langle \phi(x) \phi(0) \rangle \propto |x|^{-2\Delta_\phi}.
\end{equation}

A surprising thing about this scaling dimension is that it coincides preciesely with that of a statistical physics model. The \emph{Ising model} is a theory of classical spins $s_i = \pm 1$ on a lattice with nearest-neighbor interactions, characterized by the Hamiltonian
\begin{equation}
	H = - J \sum_{\langle ij \rangle} s_i s_j.
\end{equation}
This model has a critical value of $J$ whose continuum limit is described by a CFT. At this value, Monte-Carlo simulations show that the correlation between two spins decay with the distance with power given by
\begin{equation}
	\Delta_s \approx 0.5181.
\end{equation}
The two theories have completely different microscopic descriptions: one is a quantum field theory describing particles in Minkowski space-time, the other a simple theory on a Euclidean lattice. 
Even more surprisingly, the same critical exponent is found in other systems, such as the critical point of water and other liquids. This is an example of \emph{universality}.

The explanation for this coincidence is that there are not many candidate conformal field theories to describe the phase transition of the $\phi^4$ theory and the Ising model. In fact, these two theories have in common:
\begin{itemize}

\item
A global $\mathbb{Z}_2$ symmetry ($s_i \to - s_i$) that is broken in one phase and unbroken in the other;

\item
Exactly 2 relevant operators 
(in addition to $J$ the Ising model can be parametrized in terms of a coupling to an external magnetic field $\delta H = - \mu \sum s_i$).

\end{itemize}
%
The conformal bootstrap philosophy is quite different from the two examples above, in the sense that it does not care about the microscopic details of the theory: instead, it attempts to solve the conformal field theory that is expected to be there at the fixed point, based on symmetry arguments only. 
We shall see that the conformal bootstrap can be used to show that there exists a unique conformal field theory with 2 relevant operators and a $\mathds{Z}_2$ symmetry, and that the scaling dimension of its leading operator is equal to
\begin{equation}
	\Delta_s \approx 0.5181489.
\end{equation}
In fact, the conformal bootstrap does not only establish that, but it gives access to the whole spectrum of operators in that conformal field theory.
This is truly a success story of the bootstrap.

\item
Finally, another example of CFT is the family of theories obtained by considering a quantum field theory in $d+1$-dimensional anti de Sitter (AdS) space-time. AdS admits a compactification to a sphere with a boundary, and correlation functions on that boundary are isomorphic with that of a $d$-dimensional conformal field theory.
A generic (e.g.~perturbative) theory in AdS gives rise to a CFT on the boundary that does not satisfy all the ordinary assumptions of a quantum field theory (for instance it does not have an energy-momentum tensor), but there is evidence that a theory that includes gravity does.
Note that the same type of correspondence applies to late-time correlators in de Sitter space-time.

\end{itemize}


\subsection{Literature}

There a several excellent modern reviews on conformal field theory, and large parts of these lecture notes are directly inspired by them.
The essentials are covered in the works of conformal bootstrap experts:
\begin{itemize}

\item
Slava Rychkov's EPFL lectures
\cite{Rychkov:2016iqz};
this review also contains interesting historical comments.

\item
David Simmons-Duffin's TASI lectures
\cite{Simmons-Duffin:2016gjk}.

\item
Shai Chester's Weizmann Lectures
\cite{Chester:2019wfx}.

\end{itemize}
%
Some other useful material can be found in:
%
\begin{itemize}

\item
A review article on the state-of-the-art bootstrap methods~\cite{Poland:2018epd}.

\item
Hugh Osborn's lecture notes at Cambridge, available at: \\ 
\url{https://www.damtp.cam.ac.uk/user/ho/CFTNotes.pdf}

\item
Joshua Qualls' lectures \cite{Qualls:2015qjb}.

\end{itemize}
%
The reviews mentioned above are all mostly focused on the Euclidean approach to CFT. For a Lorentzian perspective, see:
\begin{itemize}

\item
Slava Rychkov's \emph{Lorentzian methods in conformal field theory}, available at \url{https://courses.ipht.fr/node/226}
(\href{https://www.ipht.fr/Docspht/articles/t19/229/public/Lecture-notes-Rychkov-IPHT.pdf}{link} to the lecture notes)

\end{itemize}
%
There are also modern reviews specifically on the topic of CFT in $d = 2$ dimensions:
\begin{itemize}

\item
Sylvain Ribault's notes~\cite{Ribault:2014hia}, 
also accepting suggestions on GitHub: \\ 
\url{https://github.com/ribault/CFT-Review}

\item
Schellekens ``Conformal Field Theory'' lecture notes~\cite{Schellekens:1996tg}, and a recent version available at:
\url{https://www.nikhef.nl/\textasciitilde t58/CFT.pdf}.

\end{itemize}
More classical literature on the subject can be found in:
\begin{itemize}

\item
The ``yellow book'' by Di Francesco, Mathieu, and Sénéchal
\cite{DiFrancesco:1997nk}.

\item
Polchinski's \emph{String theory} vol.~1
\cite{Polchinski:1998rq}.

\item
Lectures by Paul Ginsparg~\cite{Ginsparg:1988ui}.
%emphasis on statistical physics and string theory applications

\item
A review by Matthias Gaberdiel~\cite{Gaberdiel:1999mc}.
%emphasis on statistical physics and string theory applications

\end{itemize}
%
For theories combining supersymmetry and conformal symmetry:
\begin{itemize}

\item
Lorenz Eberhardt's lecture notes 
\cite{Eberhardt:2020cxo}.

\end{itemize}

%%%%%%%%%%%%%%%%%%%%%%%%%%%%%%%%%%%%%%%%%%%%%%%%%%%%%%%%%%%%%%%%%%%%%%

\section{Classical conformal transformations}
\label{sec:classical}

One of the most fundamental principles of physics is independence of the reference frame: observers living at different points might have different perspectives, but the underlying physical laws are the same.
This is true in space (invariance under translations and rotations), but also in space-time (e.g.~invariance under Lorentz boosts).

\subsection{Infinitesimal transformations}

In mathematical language, this means that if we have a coordinate system $x^\mu$, the laws of physics do not change under a transformations 
\begin{equation}
	x^\mu \to x'^\mu.
	\label{eq:diffeo}
\end{equation}
This principle applies to all maps that are invertible (isomorphisms) and differentiable (smooth transformations), hence it is usually called \emph{diffeomorphism} invariance.
Being differentiable, the transformation \eqref{eq:diffeo} can be Taylor-expanded to write
\begin{equation}
	x^\mu \to x'^\mu = x^\mu + \varepsilon^\mu(x),
	\label{eq:diffeo:infinitesimal}
\end{equation}
in terms of an infinitesimal vector $\varepsilon^\mu$ (meaning that we will always ignore terms of order $\varepsilon^2$).

In addition to the coordinate system, the description of a physical system requires a way of measuring distances that is provided by a metric $g_{\mu\nu}(x)$. Distances are measured integrating the line element
\begin{equation}
	ds^2 = g_{\mu\nu}(x) dx^\mu dx^\nu.
\end{equation}
Since all observers should agree on the measure of distances, we must have
\begin{equation}
	g'_{\mu\nu}(x') dx'^\mu dx'^\nu = g_{\mu\nu}(x) dx^\mu dx^\nu,
\end{equation}
Here $g_{\mu\nu}$ could be the Euclidean metric $\delta_{\mu\nu}$ or the Minkowski metric $\eta_{\mu\nu}$; for simplicity we only consider the case in which $g_{\mu\nu}$ is flat, i.e.~$\partial_\alpha g_{\mu\nu} = 0$.
In this case, we can write
\begin{equation}
\begin{aligned}
	g'_{\mu\nu} &= g_{\alpha\beta}
	\frac{\partial x^\alpha}{\partial x'^\mu}
	\frac{\partial x^\beta}{\partial x'^\nu}
	\\
	&= g_{\alpha\beta}
	\left( \delta^\alpha_\mu - \partial_\mu \varepsilon^\alpha \right)
	\left( \delta^\beta_\nu - \partial_\nu \varepsilon^\beta \right)
	\\
	&= g_{\mu\nu}
	- \left( \partial_\mu \varepsilon_\nu
	+ \partial_\nu \varepsilon_\mu \right).
\end{aligned}
\end{equation}
If we require the different observers to also agree on the metric, then  we must have $g'_{\mu\nu} = g_{\mu\nu}$, which gives a constraint on what kind of coordinate transformations are possible: we must have
\begin{equation}
	\partial_\mu \varepsilon_\nu
	+ \partial_\nu \varepsilon_\mu = 0.
\end{equation}
This condition admits as a most general solution
\begin{equation}
	\varepsilon^\mu = a^\mu + \omega^\mu_{~\nu} x^\nu,
\end{equation}
where $a^\mu$ is a constant vector and $g_{\mu\rho} \omega^\rho_{~\nu} = \omega_{\mu\nu}$ an antisymmetric tensor.
The transformation
\begin{equation}
	x^\mu \xrightarrow{P} x^\mu + a^\mu
\end{equation}
is obviously a translation and
\begin{equation}
	x^\mu \xrightarrow{M}
	\left( \delta^\mu_\nu + \omega^\mu_{~\nu} \right) x^\nu
\end{equation}
a rotation/Lorentz transformation around the origin $x = 0$: the matrix $R^\mu_{~\nu} = \delta^\mu_\nu + \omega^\mu_{~\nu}$ satisfies $R^\mu_{~\alpha} g^{\alpha\beta} R^{T\nu}_\beta = g^{\mu\nu}$. The composition of these two operations generates the Poincaré group.
This is the fundamental symmetry of space-time underlying all relativistic quantum field theory. It is a symmetry of nature to a very good approximation, at least up to energy scales in which quantum gravity becomes important.

However, one can also consider the situation in which the two observers use different systems of units, i.e.~they disagree on the overall definition of scale, but agree otherwise on the metric being flat.
In this case we must have $g'_{\mu\nu} \propto g_{\mu\nu}$, and therefore the constraint becomes
\begin{equation}
	\partial_\mu \varepsilon_\nu
	+ \partial_\nu \varepsilon_\mu = 2 \lambda g_{\mu\nu},
\end{equation}
for some real number $\lambda$,
with the most general solution
\begin{equation}
	\varepsilon^\mu = a^\mu + \omega^\mu_{~\nu} x^\nu
	+ \lambda x^\mu.
\end{equation}
The new infinitesimal transformation is
\begin{equation}
	x^\mu \xrightarrow{D} (1 + \lambda) x^\nu.
\end{equation}
It is a scale transformation. Note that scale symmetry is not a good symmetry of nature: there is in fact a fundamental energy scale on which all observer must agree (this can be for instance chosen to be the mass of the electron).
Nevertheless, there are some systems in which this is a very good approximate symmetry. One can also make very interesting \emph{Gedankenexperimente} that have scale symmetry built in. These are reasons that make it worth studying.

If one pushes this logic further, in a scale-invariant world in which observers have no physical mean of agreeing on a fundamental scale, they might even decide to change their definition of scale as they walk around, or as time passes. This would correspond to the case in which the metric $g'_{\mu\nu}$ of one observer can differ from the original metric $g_{\mu\nu}$ by a function of space(-time):
\begin{equation}
	g'_{\mu\nu}(x) = \Omega(x) g_{\mu\nu}.
\end{equation}
Note that we are not saying that $g'_{\mu\nu}$ is completely arbitrary: at every point in space time it is related to the flat metric by a scale transformation. But the scale factor is different at every point.
The condition on $\varepsilon^\mu$ becomes in this case
\begin{equation}
	\partial_\mu \varepsilon_\nu
	+ \partial_\nu \varepsilon_\mu = 2 \sigma g_{\mu\nu},
	\label{eq:conformalkillingeq}
\end{equation}
where $\Omega(x) = e^{-2 \sigma(x)} \cong 1 - 2\sigma(x)$. To find the most general solution to this equation, note that contracting the indices with $g^{\mu\nu}$ gives
\begin{equation}
	\partial_\mu \varepsilon^\mu  = d \sigma,
\end{equation}
where $d$ is the space(-time) dimension,
while acting with $\partial^\nu$ gives
\begin{equation}
	\partial_\mu \partial_\nu \varepsilon^\nu
	+ \partial^2 \varepsilon_\mu
	= 2 \partial_\mu\sigma,
\end{equation}
so that we get
\begin{equation}
	\partial^2 \varepsilon_\mu
	= (2 - d) \partial_\mu\sigma.
\end{equation}
Acting once again with $\partial^\mu$, we arrive at 
\begin{equation}
	(d - 1) \partial^2 \sigma = 0,
\end{equation}
while acting with $\partial^\nu$ and symmetrizing the indices, we find
\begin{equation}
	(2 - d) \partial_\mu\partial_\nu \sigma
	= g_{\mu\nu} \partial^2 \sigma
	= 0.
\end{equation}
In $d > 2$, we obtain therefore the condition $\partial_\mu \partial_\nu \sigma = 0$, which is solved by 
\begin{equation}
	\sigma(x) = \lambda + 2 b \cdot x.
\end{equation}
We have therefore
\begin{equation}
	\varepsilon^\mu = a^\mu + \omega^\mu_{~\nu} x^\nu
	+ \lambda x^\mu
	+ 2 (b \cdot x) x^\mu - x^2 b^\mu.
	\label{eq:conformalkillingvec}
\end{equation}
In addition to the transformations found before, we also find
\begin{equation}
	x^\mu \xrightarrow{K} x^\mu +  2 (b \cdot x) x^\mu - x^2 b^\mu,
\end{equation}
which is a \emph{special conformal transformation}.
If we examine the Jacobian for this transformation, we find
\begin{equation}
	\frac{\partial x'^\mu}{\partial x^\nu}
	= \left(1 + 2 b \cdot x \right) \delta^\mu_\nu
	+ 2 \left( b_\nu x^\mu - x_\nu b^\mu \right)
	\approx \left(1 + 2 b \cdot x \right)
	R^\mu_{~\nu}(x).
\end{equation}
We have written this as a position-dependent scale factor $(1 + 2 b \cdot x)$, multiplying an orthogonal matrix
\begin{equation}
	R^\mu_{~\nu}(x) = \delta^\mu_\nu
	+ 2 \left( b_\nu x^\mu - x_\nu b^\mu \right).
\end{equation}
This shows that special conformal transformation act locally as the composition of a scale transformation and a rotation (or Lorentz transformation). This also shows that conformal transformations preserve angles, which is the origin of their name.
Eq.~\eqref{eq:conformalkillingeq} is sometimes called the (conformal) Killing equation and its solutions \eqref{eq:conformalkillingvec} the Killing vectors.



Note that in our derivation the original metric $g_{\mu\nu}$ was flat, but the new metric $g'_{\mu\nu}$ is not. It is however conformally flat: it is always possible to make a change of coordinate after which it is flat. In general, transformations
\begin{equation}
	g_{\mu\nu}(x) \to \Omega(x) g_{\mu\nu}(x)
\end{equation}
are called \emph{Weyl transformations}. They change the geometry of space-time. We found that any Weyl transformation which is at most quadratic in $x$ can be compensated by a change of coordinates to go back to flat space. The corresponding flat-space transformation is called conformal transformation.%
%
\footnote{This implies that the group of conformal transformation is a subgroup of diffeomorphisms. It is in fact the largest \emph{finite-dimensional} subgroup.}% Rychkov (1.53)



Note that in $d = 2$ the situation is a bit different: the conditions $\partial^2 \sigma = 0$ is sufficient to ensure that the Killing equation has a solution. This is most easily seen in light-cone coordinates,
\begin{equation}
	x^+ = \frac{x^0 + x^1}{2},
	\qquad
	x^- = \frac{x^0 - x^1}{2},
\end{equation}
in terms of which
\begin{equation}
	\partial^2 \sigma = \partial_+ \partial_- \sigma.
\end{equation}
This is satisfied by taking for $\sigma$ the sum of an arbitrary function of the left-moving variable $x^+$ and of another function of the right-moving variable $x^-$. In fact, if we write $\varepsilon^\pm = \varepsilon^0 \pm \varepsilon^1$, we can 	take arbitrary functions $\varepsilon^+(x^+)$ and $\varepsilon^-(x^-)$, and verify that eq.~\eqref{eq:conformalkillingeq} is satisfied with $\sigma = \frac{1}{2} \left( \partial_+ \varepsilon_+ + \partial_- \varepsilon_- \right)$.
In the Euclidean case we take
\begin{equation}
	z = \frac{x^1 + i x^2}{2},
	\qquad
	\bar{z} = \frac{x^1 - i x^2}{2},
\end{equation}
complex-conjugate to each other, and the same logic follows: we can apply arbitrary holomorphic and anti-holomorphic transformations on $z$ and $\bar{z}$, and the conformal Killing equation is always satisfied. This shows that there are infinitely more conformal transformations in $d = 2$ than in $d > 2$, and also that there is no significant difference between Euclidean and Minkowski conformal transformation in $d = 2$, as the transformation acts essentially on the two light-cone/holomorphic coordinates independently.


\subsection{The conformal algebra}

The conformal Killing equation \eqref{eq:conformalkillingvec} determines the most general form of \emph{infinitesimal} conformal transformations. \emph{Finite} conformal transformations follow from a sequence of infinitesimal transformations.
However, one has to bear in mind that the infinitesimal conformal transformations do not all commute: for instance, a translation followed by a rotation is not the same as the opposite.
In fact, the conformal transformations form a \emph{group}: the composition of conformal transformations is again a conformal transformations.

As we all know from quantum field theory, a group is characterized by its \emph{generators} and their commutation relations (the \emph{algebra}). A generator $G$ describes an infinitesimal transformation in some direction, and finite transformation are obtained using exponentiation, $e^{i \theta G}$, with parameter $\theta$ (the factor of $i$ is the physicist's convention that make the generators Hermitian).
A representation of the conformal group is given on the functions of the coordinates, $f(x)$. For instance, under an infinitesimal translation, we have
\begin{equation}
	f(x) \xrightarrow{P} f(x') = f(x + a) \approx f(x)
	+ a^\mu \partial_\mu f(x)
\end{equation}
and we require this to be equal to $e^{-i a_\mu P^\mu} f(x)$, which means
\begin{equation}
	P_\mu = i \partial_\mu.
	\label{eq:P:fcts}
\end{equation}
Performing the same analysis for the other infinitesimal transformations given in eq.~\eqref{eq:conformalkillingvec}, we obtain for the other generators%
%
\footnote{The sign of these generators is an arbitrary convention. It defines once and for all the commutations relations that we will derive next. After that, we will always refer to the commutation relations as the definition of the generators.}
\begin{align}
	& \text{rotations/Lorentz transformations:} \quad &
	M^{\mu\nu} &= i \left( x^\mu \partial^\nu
	- x^\nu \partial^\mu \right)
	\label{eq:M:fcts}
	\\
	& \text{dilatations:} & 
	D &= i x^\mu \partial_\mu,
	\label{eq:D:fcts}
	\\
	& \text{special conformal transformations:} &
	K^\mu &= i \left( 2 x^\mu x^\nu \partial_\nu 
	- x^2 \partial^\mu \right).
	\label{eq:K:fcts}
\end{align}
The number of generators matches that of the Killing vectors: there are $d$ translations, $d$ special conformal transformations, $d (d-1)/2$ rotations/Lorentz transformations ($M^{\mu\nu}$ is a $d \times d$ antisymmetric matrix), and one scale transformation.
Therefore the total number of generators, i.e.~the dimension of this group, is $(d + 1)(d + 2)/2$. In $d = 4$ space-time dimension, the conformal group has 15 generators.

Using the above definition, one can verify that the following commutation relations are satisfied,
\begin{equation}
\begin{aligned}
	\left[ M^{\mu\nu}, M^{\rho\sigma} \right]
	&= -i \left( g^{\mu\rho} M^{\nu\sigma} - g^{\mu\sigma} M^{\nu\rho}
	- g^{\nu\rho} M^{\mu\sigma} + g^{\nu\sigma} M^{\mu\rho} \right]
	\\
	\left[ M^{\mu\nu}, P^\rho \right]
	&= -i \left( g^{\mu\rho} P^\nu - g^{\nu\rho} P^\mu \right)
	\\
	\left[ M^{\mu\nu}, K^\rho \right]
	&= -i \left( g^{\mu\rho} K^\nu - g^{\nu\rho} K^\mu \right)
	\\
	\left[ D, P^\mu \right] &= -i P^\mu 
	\\
	\left[ D, K^\mu \right] &= i K^\mu 
	\\
	\left[ P^\mu, K^\nu \right]
	&= 2i \left( g^{\mu\nu} D - M^{\mu\nu} \right)
\end{aligned}
\label{eq:conformalalgebra}
\end{equation}
while all other commutators vanish:
\begin{equation}
	\left[ M^{\mu\nu}, D \right]
	= \left[ P^\mu, P^\nu \right] 
	= \left[ K^\mu, K^\nu \right] 	
	= 0.
\end{equation}
The first two relations in eq.~\eqref{eq:conformalalgebra} are the familiar Poincaré algebra. 
The next one states that $K^\mu$ transforms like a vector (as $P^\mu$ does), whereas $D$ is obviously a scalar. The next two relations remind us that $K^\mu$ and $P^\mu$ have respectively the dimension of length and inverse length. 

\begin{exercise}
	Derive the commutation relations from the action \eqref{eq:P:fcts}--\eqref{eq:K:fcts} of the generators on functions of $x$.
\end{exercise}

Even though this is not immediately obvious, this algebra is isomorphic to that of the group $\SO(d+1, 1)$ (if $g^{\mu\nu}$ is the Euclidean metric) or $\SO(d, 2)$ (if it is the Minkowski metric).
To see that it is the case, let us introduce a $(d + 2)$-dimensional space with coordinates
\begin{equation}
	X^\mu, \quad X^{d+1}, \quad X^{d+2},
\end{equation}
and a metric defined by the line element
\begin{equation}
	ds^2 = g_{\mu\nu} dX^\mu dX^\nu + dX^{d+1} dX^{d+1}
	- dX^{d+2} dX^{d+2}
	\equiv \eta_{MN} dX^M dX^N.
\end{equation}
Then we write all conformal commutation relations as being defined by the Lorentzian algebra
\begin{equation}
	\left[ J^{MN}, J^{RS} \right]
	= -i \left( \eta^{MR} J^{NS} - \eta^{MS} J^{NR}
	- \eta^{NR} J^{MS} + \eta^{NS} J^{MR} \right],
\end{equation}
provided that we identify the antisymmetric generators $J^{MN}$ with the conformal generators as follows:
\begin{equation}
\begin{aligned}
	M^{\mu\nu} &= J^{\mu\nu},
	\\
	P^\mu &= J^{\mu, d+1} + J^{\mu, d+2},
	\\
	K^\mu &= J^{\mu, d+1} - J^{\mu, d+2},
	\\
	D &= J^{d+1, d+2}.
\end{aligned}
\label{eq:embeddingspacealgebra}
\end{equation}


\subsection{Finite transformations}

We just saw that the infinitesimal conformal transformations generate a group. But how can we describe finite conformal transformations? Let us see how each generator exponentiates into an element of the group; the most general conformal transformation can then be obtained as a composition of such finite transformations.

In some cases the exponentiation is trivial. For instance, with translations we obtain immediately
\begin{equation}
	x^\mu \xrightarrow{P} x^\mu + a^\mu,
\end{equation}
where $a$ is now any $d$-dimensional vector, not necessarily small.
The same is true of scale transformations,
\begin{equation}
	x^\mu \xrightarrow{D} \lambda x^\mu
\end{equation}
with finite $\lambda$.
Rotations or Lorentz transformations exponentiate as
\begin{equation}
	x^\mu \xrightarrow{M} \Lambda^\mu_{~\nu} x^\nu
\end{equation}
where $\Lambda^\mu_{~\nu}$ is a $\SO(d)$ or $\SO(1, d-1)$ matrix, depending whether the metric is Euclidean or Minkowski.
All of this is well-known and not surprising.

On the contrary, special conformal transformations do not exponentiate trivially. The easiest way to derive their finite form is to make the following observation: recall that in infinitesimal form we have
\begin{equation}
	x'^\mu = x^\mu + 2 (b \cdot x) x^\mu - x^2 b^\mu,
\end{equation}
which implies $x'^2 = \left( 1 + 2 b \cdot x \right) x^2$, and therefore (as always neglecting terms of order $b^2$)
\begin{equation}
	\frac{x'^\mu}{x'^2}
	= \frac{x^\mu}{x^2} - b^\mu.
	\label{eq:K:inversion}
\end{equation}
The ratio $x^\mu/x^2$ appearing on both side of the equation is the \emph{inverse} of the coordinate $x^\mu$, respectively $x'^\mu$: let us define the inversion as
\begin{equation}
	x^\mu \xrightarrow{I} \frac{x^\mu}{x^2}.
\end{equation}
This transformation does not have an infinitesimal form, but otherwise it shares the essential properties of a conformal transformation: its Jacobian is
\begin{equation}
	\frac{\partial x'^\mu}{\partial x^\nu}
	= \frac{1}{x^2}
	\left[ \delta^\mu_\nu - 2 \frac{x^\mu x_\nu}{x^2} \right],
\end{equation}
which is the product of a position-dependent scale factor $x^{-2}$ with an orthogonal matrix. To understand what this transformation does globally, let us consider a point $\vec{x} = (x, 0, \ldots 0) \in \mathds{R}^d$. Then the matrix is square bracket is diagonal, and equates $\text{diag}(-1, 1, \ldots, 1)$. This is an orthogonal matrix with determinant $-1$, which is part of $\text{O}(d)$ but not $\SO(d)$. This shows that the inversion is a discrete transformations not connected to identity.
A conformally invariant theory might be invariant under inversions, but it needs not be.

Eq.~\eqref{eq:K:inversion} shows that infinitesimal special conformal transformations are obtained taking an inversion followed by a translation, followed by an inversion again. Since this process involves the inversion twice, and since inversion is its own inverse, it does not matter whether inversion is a true symmetry of the system or not.
The advantage of this representation is that it can easily be exponentiated: the composition of (infinitely) many infinitesimal special conformal transformation can be written as an inversion followed by a finite translation, followed by an inversion again. In other words, eq.~\eqref{eq:K:inversion} holds for finite $b^\mu$.
This can be used to show that
\begin{equation}
	x^\mu \xrightarrow{K}
	x'^\mu = \frac{x^\mu - x^2 b^\mu}
	{1 - 2 b \cdot x + b^2 x^2}.
	\label{eq:K:finite}
\end{equation}

\begin{exercise}
	Use eq.~\eqref{eq:K:inversion} to show \eqref{eq:K:finite}.
\end{exercise}

What do special transformation do globally?
Let us look specifically in Euclidean space. There are some special points:
\begin{itemize}

\item
The origin of the coordinate system $x = 0$ is mapped onto itself.

\item
The point $b^\mu / b^2$ is mapped to $\infty$. 

\item
Conversely, the ``point'' $x \to \infty$ is mapped to the finite vector $-b^\mu/b^2$.

\end{itemize}
%
These properties can be understood from the fact that special conformal transformations and translations are related by inversion:
special conformal transformations keep the origin fixed but move every other points, including $\infty$; translations move every point \emph{except} $\infty$. The other two transformations, rotations and scale transformations, keep both $0$ and $\infty$ fixed.


An essential property of conformal transformations is that they allow to map any 3 points $(x_1, x_2, x_3)$ onto another set $( x'_1, x'_2, x'_3)$. This can be seen as follows: first, apply a translation to place $x_1$ at the origin, followed by a special conformal transformation that takes $x_3$ to $\infty$, after which the image of the original triplet is $(0, x_2'', \infty)$; then use rotations and scale transformations to move $x''_2$ to another point $x'''_2$, while keeping $0$ and $\infty$ fixed; finally apply again a special conformal transformation that takes $\infty$ to $x'_3 - x'_1$, and a translation by $x'^1$ to reach the configuration $( x'_1, x'_2, x'_3)$.
This property has an immediate physical consequence: in correlation functions of 2 or 3-points (see next sections for a definition), all kinematics is fixed by conformal symmetry. The only freedom encodes information about the operators themselves, not about their position in space.

Another interesting property of conformal transformations is that they map spheres to spheres: this is is an obvious property of translations, rotations and scale transformations, but it is also true of special conformal transformations. 

\begin{exercise}
	Show that under the special conformal transformation \eqref{eq:K:finite}, a sphere centered at the point $a^\mu$ and with radius $R$ gets mapped to a sphere centered at the point 
	$$
	a'^\mu = \frac{a^\mu - (a^2 - R^2) b^\mu}
	{1 - 2 a \cdot b + (a^2 - R^2) b^2}
	$$
	and with radius
	$$
	R' = \frac{R}{\left| 1 - 2 a \cdot b + (a^2 - R^2) b^2 \right|}.
	$$
	In the special case in which $b^\mu / b^2$ is on the original sphere, show that the sphere gets mapped to a plane orthogonal to the vector $a^\mu + (R^2 - a^2) b^\mu$.
	
\end{exercise}

Note that in $d = 2$, the additional conformal transformations (infinitely many of them!) mean that (nearly) any shape can be mapped onto another. This is known as the Riemann mapping theorem.%
%
\footnote{A neat example of how a disk is be mapped onto a polygon is available at \\
\url{https://herbert-mueller.info/uploads/3/5/2/3/35235984/circletopolygon.pdf}.}


\subsection{Compactifications}

It is often said that conformal symmetry is a symmetry of flat space(-time).
This is true, as we have just seen, provided that we treat the point $\infty$ as being part of the space.
This is quite straightforward in Euclidean space, but much more subtle in Minkowski space-time, as there are different ways of reaching $\infty$ there.

To gain a better understanding of this, it can be useful to map the flat Euclidean space $\mathds{R}^d$ or the Minkowski space-time $\mathds{R}^{1,d-1}$ onto a curved (but hopefully compact) manifold.
In Euclidean space this is for instance achieved by the (inverse) stereographic projection that maps $\mathds{R}^d \cup \{ \infty \}$ to the unit sphere $S^d$.
Geometrically, the stereographic projection is constructed as follows: embed $\mathds{R}^d$ as a (hyper-)plane in $\mathds{R}^{d+1}$, together with a sphere of unit radius centered at the origin. Every point on the plane has an image on the sphere obtained drawing a segment between the original point and the north pole of the sphere, and noting where it intersects the sphere. The origin is mapped to the south pole, $\infty$ to the north pole, and the sphere $S^{d-1}$ of unit radius to the equator.
Algebraically, this is achieved as follows: first write the Euclidean metric in spherical coordinates,
\begin{equation}
	ds^2 = dr^2 + r^2 d\Omega_{d-1}^2,
\end{equation}
where the solid angle is given in $d = 2$ by $d\Omega_1^2 = d\phi^2$, in $d = 3$ by $d\Omega_2^2 = d\theta^2 + \sin\theta^2 d\phi^2$, and more generically by the recursion relation $d\Omega_n^2 = d\theta^2 + \sin\theta^2 d\Omega_{n-1}^2$.
Let us perform the change of variable
\begin{equation}
	r = \frac{\sin\varphi}{1 - \cos\varphi},
\end{equation}
and interpret $\varphi \in [0, \pi]$ as the zenith angle on the sphere:
$\varphi = 0$ is the north pole, corresponding to $r \to \infty$, and $\varphi = \pi$ the south pole, corresponding to $r = 0$. In these coordinates we have
\begin{equation}
	ds^2 = \frac{1}{(1 - \cos\varphi)^2}
	\left( d\varphi^2 + \sin\varphi^2 d\Omega_{d-1}^2 \right)
	= \frac{1}{(1 - \cos\varphi)^2} d\Omega_d^2.
\end{equation}
This shows that the metric is flat up to an overall Weyl factor that depends on $\varphi$.
In these new coordinates, conformal transformations are always non-singular. 
For this reason, it is often very convenient to study \emph{classical} conformal transformations on the sphere $S^d$ instead of Euclidean space.

However, this compactification is not as nice in a \emph{quantum} theory in which one wants to foliate the space along some direction: if one chooses $\varphi$ as the ``Euclidean time'', then the ``space'' direction is a sphere $S^{d-1}$ whose volume depends on $\varphi$. In other words, the generator of ``time'' translations is not a symmetry of the system. Being on a perfectly symmetric sphere, this is also true of any other choice of time direction.
This is not forbidden, but it complicates vastly the analysis.

Instead, another compactification is often preferred to the sphere: from the Euclidean metric in spherical coordinates, one can make the change of variable
\begin{equation}
	\tau = \log(r)
	\qquad\Rightarrow\qquad
	r = e^\tau,
\end{equation}
after which
\begin{equation}
	ds^2 = r^2 \left( d\tau^2 + d\Omega_{d-1}^2 \right).
\end{equation}
This is again a flat metric, up to a Weyl factor $r^2 = e^{2\tau}$. In this case however the metric is completely independent of $\tau$.
This space has the geometry of the \emph{cylinder}, $\mathds{R} \times S^{d-1}$. It is not fully compact: $\tau$ goes from $-\infty$ to $+\infty$. But it has a important advantage: translations in $\tau$ are generated by dilatations $D$, which will be taken to be a symmetry of quantum field theory. Foliating the space into surfaces of constant $\tau$ will later lead us to the famous radial quantization in conformal field theory.

Note that in $\SO(d + 1, 1)$ language, the generator $D = J^{d+1, d+2}$ is completely equivalent to any other generator $J^{\mu, d+2} = \frac{1}{2} \left( P^\mu - K^\mu \right)$. So for instance we might as well look for a cylinder compactification in which the non-compact direction corresponds to translations in the direction of
$\frac{1}{2} \left( P^0 - K^0 \right)$. This generator obeys
\begin{equation}
	\frac{1}{2} \left( P^0 - K^0 \right)
	= i \left( \frac{1 - (x^0)^2 + \vec{x}^2}{2} \partial_0
	- x^0 \, \vec{x} \cdot \vec{\partial} \right),
\end{equation}
with two fixed points at $x^0 = \pm 1$ with $\vec{x} = 0$.

\begin{exercise}
	Find the change of coordinate that make the Euclidean metric Weyl-equivalent to a cylinder in which translations in the non-compact direction are generated by $\frac{1}{2} \left( P^0 - K^0 \right)$. \\
	\hint{Find a special conformal transformation followed by a translation that take $(0, \infty)$ to $(-1, 1)$, and apply it to the cylinder coordinates.}
\end{exercise}

The cylinder compactifications of Euclidean space are interesting by themselves, but they are also extremely convenient to understand the connection between Euclidean and Minkowski space-times: in this last form, performing a Wick rotation $\tau \to \pm i t$ defines a Lorentzian cylinder on which the conformal group with a simple action of the Lorentzian conformal group $\SO(d, 2)$. 
But before going there, let us go back to flat Minkowski space-time and make some general remarks.


\subsection{Minkowski space-time}

Everybody is familiar with translations and Lorentz transformations in Minkowski space-time, and even with dilatations as this is a standard tool in the renormalization group analysis. But what do conformal transformations do?

To understand this, let us place an observer at the origin of Minkowski space-time. The presence of this observer breaks translations, but not Lorentz transformations (the observer is point-like), nor dilatations and special conformal transformations. For the observer, space-time is split into three regions: a future light cone, a past light cone, and a   space-like region from which they know nothing.
Lorentz and scale transformation preserve this causal structure: the future and past light-cones are mapped onto themselves. In other words, if a point $x$ is space-like separated from the observed, it will remain space-like separated no matter the choice of scale and Lorentz frame. Without loss of generality, let us choose this point to be at position $x = (0, \vec{n})$, where $\vec{n}$ is a unit vector (units can be chosen so that this is the case).
Now apply a special conformal transformation with parameter $b^\mu = (-\alpha, \alpha \vec{n})$, with $\alpha$ varying between 0 and 1.
This draws a curve $y^\mu $ in space-time, parameterized by $\alpha$,
with
\begin{equation}
	y^0(\alpha) = \frac{\alpha}{1 - 2 \alpha},
	\qquad\qquad
	\vec{y}(\alpha) = \frac{1 - \alpha}{1 - 2 \alpha} \vec{n}.
\end{equation}
This curves begins at the space-like point $y = (0, \vec{n})$, and ends in the past light cone at $y = (-1, \vec{0})$!
Note that $y$ never crosses a light cone: the image of $x$ is never null if $x$ is not itself null, since $x'^2 = x^2 / (1 - 2 x \cdot b + x^2 b^2)$.
Instead, we have
\begin{equation}
	y^2(\alpha) = \frac{1}{1 - 2\alpha} \neq 0.
\end{equation}
The point goes all the way to space-like infinity when $\alpha = \frac{1}{2}$, and comes back from past infinity.
Clearly, special conformal transformation break causality.

The resolution of this paradox is that conformal transformation do not act directly on Minkowski space, but rather on its universal cover that is isomorphic to the Lorentzian cylinder described in the previous section.
Time evolution on that cylinder is given by the Hamiltonian $H = \frac{1}{2} (P^0 - K^0)$. At any given time $t_0$, space is compactified in such a way that the notion of infinite distance is unequivocal: space-like infinity corresponds to a point on the sphere. If one takes any other point of that sphere and apply finite translations using the generator $P^\mu$, then this defines a compact \emph{Poincaré patch}. The full cylinder is a patchwork of Minkowski space-times, but every observer only has access to one.

The lesson that we need to learn from this is that only the infinitesimal form of special conformal transformation can be used in Minkowski space-time: \emph{any} finite special conformal transformation would bring part of space-time into another patch on the cylinder.
This is sometimes called \emph{weak conformal invariance}.

A quantum field theory defined on Minkowski space-time with weak conformal invariance uniquely defines a quantum field theory on the Lorentzian cylinder with global conformal invariance. This non-trivial fact was proven by Martin Lüscher and Gerhard Mack in 1974 \cite{Luscher:1974ez}, both employed at the University of Bern at the time. In their own words:

\begin{center}

\parbox{11cm}{%
\emph{In picturesque language, [the superworld] consists of Minkowski space, infinitely many ``spheres of heaven'' stacked above it and infinitely many ``circles of hell'' below it.}}

\end{center}


\subsection{Conformal symmetry in classical field theory}

So far we have been discussing conformal transformations of the coordinates. The next step is to consider a field theory (for the moment a classical one) that has conformal symmetry built in. For this, consider a field theory defined by an action. The simplest example is the free scalar field theory
\begin{equation}
	S = \int d^dx
	\left[ - \frac{1}{2} \partial_\mu \phi \partial^\mu \phi \right].
	\label{eq:freescalaraction}
\end{equation}
We shall see in the next section that the action principle can in fact be dropped in the conformal field theory, but for now it is a convenient starting point.

In this context, a conformal transformation is a transformation \emph{of the fields}.
There are two different and complementary perspectives one can adopt. It is often convenient to think of the metric tensor as a field in its own right, and to define a conformal transformation as a (position-dependent) scale transformation of the field
\begin{equation}
	\phi(x) \to e^{\Delta \sigma(x)} \phi(x),
	\label{eq:freescalar:conformaltransformation:1}
\end{equation}
combined with a Weyl transformation of the metric
\begin{equation}
	g_{\mu\nu}(x) \to e^{2 \sigma(x)} g_{\mu\nu}(x).
	\label{eq:freescalar:conformaltransformation:2}
\end{equation}
$\Delta$ is the \emph{scaling dimension} of the field $\phi$. In a free theory it coincides with the engineering dimension of the field in units of energy (inverse units of length), namely
\begin{equation}
	\Delta = \frac{d-2}{2}.
\end{equation}
$\sigma(x)$ is an infinitesimal scale factor that satisfies $\partial_\mu \partial_\nu \sigma = 0$.
The advantage of this perspective is that the conformal transformations are simple, multiplicative transformations of the fields. The disadvantage is that it requires thinking of theory in curved space-time. This means that the metric that is implicit in the action \eqref{eq:freescalaraction} must be made explicit. 
%
\begin{exercise}
	Verify that the action \eqref{eq:freescalaraction}
	is invariant under the infinitesimal conformal transformations
	\eqref{eq:freescalar:conformaltransformation:1} and
	\eqref{eq:freescalar:conformaltransformation:2},
	after the explicit dependence on the curved-space metric 
	has been taken into account in the contraction of Lorentz
	indices and in the measure of integration.
\end{exercise}
%
Moreover, the action could be augmented with a term depending on the scalar curvature tensor $R$ as
\begin{equation}
	S = \int d^dx
	\left[ - \frac{1}{2} \partial_\mu \phi \partial^\mu \phi
	+ \alpha R \phi^2 \right].
\end{equation}
Since $R$ vanishes in flat space, it looks like this additional term could appear with an arbitrary coefficient $\alpha$ without modifying the original flat-space action.

For this reason, it is also convenient to consider the opposite perspective in which conformal transformations are transformations of the (dynamical) fields and of the coordinates, but not of the metric. In this case the conformal transformation can be defined as
\begin{equation}
	\phi(x) \to e^{\Delta \sigma(x)} \phi(x + \varepsilon),
\end{equation}
where the parameters $\sigma$ and $\epsilon$ are related by the conformal Killing equation \eqref{eq:conformalkillingeq}, e.g.~$d \sigma = \partial_\mu \varepsilon^\mu$. 
In infinitesimal form, this transformation becomes
\begin{equation}
	\phi(x) \to \left[ 1
	+ \frac{d-2}{2d} (\partial_\nu \varepsilon^\nu)
	+ \varepsilon^\nu \partial_\nu \right] \phi(x).
	\label{eq:freescalar:conformaltransformation:3}
\end{equation}
%
\begin{exercise}
	Show that under the transformation
	\eqref{eq:freescalar:conformaltransformation:3},
	the free scalar field Lagrangian
	$$
	\Lagr = - \frac{1}{2} \partial_\mu \phi \partial^\mu \phi,
	$$
	is shifted by a total derivative term
	$$
	\delta\Lagr = \partial_\mu \left( 
	- \frac{1}{2} \varepsilon^\mu \partial_\nu \phi \partial^\nu \phi
	- \frac{d-2}{2d} \partial_\nu \varepsilon^\nu
	\phi \partial^\mu \phi \right),
	$$
	hence proving that this is a symmetry of the action.
	You will need the fact that $\varepsilon$ is at most quadratic
	in $x$.
\end{exercise}
Note that Poincaré symmetry is a special case of this transformation, corresponding to constant $\varepsilon$ (and thus $\sigma = 0$).

By Noether's theorem, whenever an action is invariant under some transformation of the field
\begin{equation}
	\phi \to \phi + \delta_\varepsilon \phi,
\end{equation}
i.e.~whenever the Lagrangian varies by a total derivative term,
\begin{equation}
	\Lagr \to \Lagr + \partial_\mu \Lambda_\varepsilon^\mu,
\end{equation}
then there exists a conserved current
\begin{equation}
	J_\varepsilon^\mu = \Lambda^\mu_\varepsilon
	- \frac{\partial \Lagr}{\partial (\partial_\mu \phi)}
	\delta_\varepsilon \phi.
\end{equation}
In our example, this conserved current is therefore
\begin{equation}
	J_\varepsilon^\mu = \varepsilon_\nu
	\left( \partial^\mu \phi \partial^\nu \phi
	- \frac{1}{2} g^{\mu\nu} \partial_\rho \phi \partial^\rho \phi
	\right)
	\equiv \varepsilon_\nu T_c^{\mu\nu}.
	\label{eq:Noethercurrent}
\end{equation}
The 2-index tensor on the right-hand side is called the \emph{canonical energy-momentum tensor}.
Its divergence satisfies
\begin{equation}
	\partial_\nu T_c^{\mu\nu}
	= \partial^\mu \phi \partial^2 \phi,
	\label{eq:Tcanonical:conservation}
\end{equation}
therefore vanishing by the equation of motion for the free field, $\partial^2 \phi = 0$.
This implies in turn that the Noether current \eqref{eq:Noethercurrent} is conserved \emph{for constant $\varepsilon$}.
If on the contrary $\varepsilon$ depends on space(-time), then we have
\begin{equation}
	\partial_\mu J_\varepsilon^\mu
	= (\partial_\nu \varepsilon_\mu) T_c^{\mu\nu}.
\end{equation}
In our example the canonical energy-momentum tensor is symmetric in its indices $\mu$ and $\nu$, and therefore we can write
\begin{equation}
	\partial_\mu J_\varepsilon^\mu
	= \frac{1}{2} (\partial_\mu \varepsilon_\nu
	+ \partial_\nu \varepsilon_\mu) T_c^{\mu\nu}
	= \sigma g_{\mu\nu} T_c^{\mu\nu}
	= -\frac{d-2}{2} \, \sigma \, \partial_\mu \phi \partial^\mu \phi.
\end{equation}
In dimensions $d > 2$, this current is only conserved when $\sigma = 0$. This is surprising, because we just showed that scale and special conformal transformations are also symmetries of the action, so why is the Noether current not conserved?

The reason is that the version of Noether's theorem given above does not straightforwardly apply to the case of a space-time dependent parameter $\varepsilon$. In fact, the energy-momentum tensor that we computed in this example is not unique: one can always add to it a piece proportional to
\begin{equation}
	\left( \partial^\mu \partial^\nu - g^{\mu\nu} \partial^2 \right)
	\phi^2,
\end{equation}
without affecting the conservation equation \eqref{eq:Tcanonical:conservation}, but changing the value of its trace. The combination
\begin{equation}
	T^{\mu\nu} = T^{\mu\nu}_c + \frac{d-2}{2(d-1)}
	\left( \partial^\mu \partial^\nu - g^{\mu\nu} \partial^2 \right)
	\phi^2
\end{equation}
is for instance traceless in any $d$.
It turns out that it is always possible in a field theory with conformal symmetry to construct an energy-momentum tensor that is:
\begin{itemize}
\item symmetric ($T^{\mu\nu} = T^{\nu\mu}$),
\item traceless ($g_{\mu\nu} T^{\mu\nu} = 0$), and
\item conserved ($\partial_\nu T^{\mu\nu} = 0$) once the equation of motions are imposed.
\end{itemize}
This is a non-trivial fact, but we will skip its proof. As already mentioned, we are interested in theories that are not necessarily defined through an action.
% \cite{Polchinski:1987dy}

Strictly speaking, Noether's theorem only applies to theories that have a Lagrangian description, but we will assume the existence of a traceless energy-momentum tensor in all cases (in some sense this is going to be one of the ``axioms'' of conformal field theory).
From this assumption, we can deduce that the theory is invariant under conformal transformations. A conserved current can always be built from the energy-momentum tensor and a conformal Killing vector $\varepsilon$ as
\begin{equation}
	J^\mu = \varepsilon_\nu T^{\mu\nu}
	\qquad\Rightarrow\qquad
	\partial_\mu J^\mu
	= \varepsilon_\nu \partial_\mu T^{\mu\nu}.
\end{equation}
As always, conserved charges can be constructed as the integral of the time component of a conserved current over space. The simplest example is 
\begin{equation}
	P^\mu = -i \int d^{d-1}\vec{x}\, T^{0\mu}(x).
	\label{eq:conservedcharge:P}
\end{equation}
This is a priori a function of $x^0$, but it is in fact constant in time, since
\begin{equation}
	\partial_0 P^\mu
	= -i \int d^{d-1}\vec{x}\, \partial_0 T^{0\mu}(x)
	= i \int d^{d-1}\vec{x}\, \partial_i T^{i\mu}(x) = 0.
\end{equation}
This conserved charge is the \emph{momentum}, associated with translation symmetry. Similarly, conserved charges associated with Lorentz transformations,
\begin{equation}
	M^{\mu\nu} = -i \int d^{d-1}\vec{x}\, \left[
	x^\mu T^{0\nu}(x) - x^\nu T^{0\nu}(x) \right]
	\label{eq:conservedcharge:M}
\end{equation}
with scale transformations,
\begin{equation}
	D = -i \int d^{d-1}\vec{x}\, x_\mu T^{0\mu}(x),
	\label{eq:conservedcharge:D}
\end{equation}
and with special conformal transformations
\begin{equation}
	K^\mu = -i \int d^{d-1}\vec{x} \,
	\left[ 2 x^\mu x_\nu T^{0\nu}(x) - x^2 T^{0\mu}(x) \right].
	\label{eq:conservedcharge:K}
\end{equation}
%
\begin{exercise}
	Show that the charges \eqref{eq:conservedcharge:M},
	\eqref{eq:conservedcharge:D} and \eqref{eq:conservedcharge:K}
	are conserved in time.
\end{exercise}

Note that the conservation of all these charges relies on vanishing divergence of the energy-momentum tensor, $\partial_\nu T^{\mu\nu}$, which itself relies on the equation of motion being satisfied.
This is certainly true in the absence of sources. But when source terms are added to the action, the equation of motion is modified.
In our example of the free scalar field theory, this would correspond to adding to the action \eqref{eq:freescalaraction} a source term of the form
\begin{equation}
	S_\text{source} = \int d^dx \, J(x) \phi(x),
\end{equation}
which modifies the equation of motion to
\begin{equation}
	\partial^2 \phi + J = 0.
\end{equation}
Instead of the conservation equation \eqref{eq:Tcanonical:conservation} for the canonical energy-momentum tensor, we must replace it with
\begin{equation}
	\partial_\nu T^{\mu\nu}(x) = - J \partial^\mu \phi.
\end{equation}
In the presence of such a source, the charges \eqref{eq:conservedcharge:P}--\eqref{eq:conservedcharge:K} are not conserved anymore.
However, let us assume that the source is local, say $J(x) = \delta^d(x - x_\odot)$, so that
\begin{equation}
	\partial_\nu T^{\mu\nu}(x) = -\delta^d(x - x_\odot)
	\partial^\mu \phi(x_\odot).
\end{equation}
Then we determine the change in one of the charge --- say $P^\mu$ ---
between a time $x^0 < x_\odot^0$ anterior to the local source, and a time $x^0 > x_\odot^0$ posterior to it, and call this difference the momentum of the source, $P_\mu(\phi_\odot)$. 
By our definition, this is equal to
\begin{equation}
	P_\mu(\phi_\odot) = -i \int\limits_{\partial\Sigma} d^{d-1} n_\nu
	T^{\mu\nu},
\end{equation}
where $\partial\Sigma$ is a codimension-one surface in $\mathds{R}^{1,d-1}$ that consists in two planes of constant $x^0$, joining at spatial infinity, and surrounding the point $x_\odot$.
By the divergence theorem, this is equal to 
\begin{equation}
	P_\mu(\phi_\odot) = -i \int\limits_{\Sigma} d^dx \, \partial_\nu
	T^{\mu\nu}
	= i \int\limits_{\Sigma} d^dx \, J \partial^\mu \phi
	= i \partial^\mu \phi(x_\odot).
	\label{eq:P:localsource}
\end{equation}
This equation says that the charge $P_\mu$ associated with a local source for the field $\phi$ is equal to $\partial^\mu \phi$. This is strikingly similar to the action of the generator \eqref{eq:P:fcts} on functions. In fact, it is easy to verify that the other charges \eqref{eq:conservedcharge:M}, \eqref{eq:conservedcharge:D}, and \eqref{eq:conservedcharge:K} also act on the classical field $\phi(x)$ exactly like the generators \eqref{eq:M:fcts}, \eqref{eq:D:fcts}, and \eqref{eq:K:fcts} respectively.

We worked with the free scalar field theory example, but it is straightforward to generalized this to arbitrary classical field theories. This shows that a traceless energy-momentum tensor can be used to give a field-theoretical realization of the conformal generators discussed before.

%note that scale invariance and unitarity generally imply that the trace of the energy-momentum tensor is zero; it can at most be derivative of virial current!
% advanced exercise


\section{Conformal quantum field theory}
\label{sec:quantum}

Let us now turn to quantum field theory and study the implications of conformal symmetry in that context. The most standard approach to quantum field theory is to think of a classical field theory, for which we now have a basic understanding of what conformal symmetry does, and then quantize it by promoting the fields to operators acting on some Hilbert space.
But this is not the approach we will take here. There will still be states and operators, but the latter will not \emph{necessarily} be associated with fields appearing in a Lagrangian.

\subsection{Non-perturbative quantum field theory}

To define a quantum field theory non-perturbatively, we need the following ingredients:
\begin{enumerate}

\item
The Minkowski space-time is foliated into surfaces of equal-time,
and to each time slice we associate a Hilbert space of quantum states.
This Hilbert space includes a vacuum state $\ket{0}$.

\item
There are a number of (in fact infinitely many) \emph{local} operators that act on this Hilbert space. For instance, let us take $\phi(x^0, \vec{x})$ to be an operator acting on the Hilbert space at time $t = x^0$.
We call this operator \emph{local} in $\vec{x}$ because it commutes with any other local operators on the same time slice inserted at a point $\vec{x}' \neq \vec{x}$:
\begin{equation}
	\left[ \phi(x^0, \vec{x}), \phi(x^0, \vec{x}') \right]
	= 0 
	\qquad\qquad
	\vec{x} \neq \vec{x}'.
	\label{eq:locality:equaltime}
\end{equation}

\item
One of the local operators of the theory is the energy-momentum tensor, and from it we can define conserved charges, including $P^\mu$ as in eq.~\eqref{eq:conservedcharge:P}. $P^\mu$ is conserved in time, so it is a good operator valid on all Hilbert spaces at any time $t$. 
In particular, we shall assume that the vacuum state has zero charge,
\begin{equation}
	P^\mu \ket{0} = 0.
	\label{eq:vacuuminvariance:P}
\end{equation}
The value of $P^\mu$ changes however every time an operator is inserted at some point $(x^0, \vec{x})$. In analogy with eq.~\eqref{eq:P:localsource}, this change is encoded in the commutator
\begin{equation}
	\left[ P^\mu, \phi(x) \right] = i \partial^\mu \phi(x).
	\label{eq:commutator:P}
\end{equation}
This equation is solved by 
\begin{equation}
	\phi(x) = e^{-i x \cdot P} \phi(0) e^{i x \cdot P}.
	\label{eq:P:exponentiated}
\end{equation}
Note that while $P^\mu$ is not a local operator, its commutator with any local operator is again local.

Since $P^\mu$ is Hermitian (this can be seen from \eqref{eq:commutator:P} in the case of a Hermitian local operator $\phi$), this equation indicates that translations are represented by a unitary operator on the Hilbert space.
This is true both of space- and time-translations, and therefore the Hamiltonian that defines the time-evolution of states is $P^0$: a state $\ket{\Psi}$ defined at time $t = 0$ is translated to another state at time $t$ by 
\begin{equation}
	\ket{ \Psi}_t = e^{i t P^0} \ket{ \Psi}.
\end{equation}
Since time evolution is unitary, the Hilbert space will be the same on all surfaces.

\item
Lorentz transformations are similarly realized as unitary transformations of the Hilbert space. The charge $M^{\mu\nu}$ associated with Lorentz symmetry is again a conserved operator whose commutator with any local operator can be expressed as another local operator. If we choose to decompose a local operator inserted at the origin of space-time into irreducible representations of the Lorentz group and denote these with $\phi^a(0)$, $a$ collectively denoting some Lorentz indices, then we must have
\begin{equation}
	\left[ M^{\mu\nu}, \phi_a(0) \right]
	= i \left( \mathcal{S}^{\mu\nu} \right)_{ab} \phi^b(0),
\end{equation}
where $\left( \mathcal{S}^{\mu\nu} \right)_{ab}$ is a matrix that satisfies the $SO(1, d-1)$ algebra. It vanishes for a scalar operator. For a vector operator with one Lorentz index, it is given by
\begin{equation}
	\left( \mathcal{S}^{\mu\nu} \right)_{\alpha\beta}
	= \delta^\mu_\alpha \delta^\nu_\beta
	- \delta^\mu_\beta \delta^\nu_\alpha.
	\label{eq:spinop:vector}
\end{equation}
When combined with eq.~\eqref{eq:P:exponentiated}, this implies
\begin{equation}
	\left[ M^{\mu\nu}, \phi^a(x) \right] = 
	-i \left( x^\mu \partial^\nu - x^\nu \partial^\mu \right) \phi^a(x)
	+ i \left( \mathcal{S}^{\mu\nu} \right)^a_{~b} \phi^b(0).
	\label{eq:commutator:M}
\end{equation}
As with translations, we shall also assume that the vacuum is invariant under rotations, namely
\begin{equation}
	M^{\mu\nu} \ket{0} = 0.
\end{equation}
The fact that the quantum theory is Poincaré invariant implies that the locality condition \eqref{eq:locality:equaltime} can be rewritten as
\begin{equation}
	\left[ \phi(x), \phi(y) \right] = 0
	\qquad
	\text{if}~(x-y)^2 > 0.
	\label{eq:causality}
\end{equation}
In other words, local operators commute as long as they are space-like separated.

\item
Finally, the last condition has to do with the positivity of energy.
Working in a basis of eigenstates of energy and momentum, 
\begin{equation}
	P^\mu \ket{p} = p^\mu \ket{p},
\end{equation}
then we require $p^0 \geq 0$. Since this must be true in any Lorentz frame, $p$ must be contained inside the forward light-cone, i.e.
\begin{equation}
	p^0 \geq \left| \vec{p} \right|
	\qquad \Leftrightarrow \qquad
	p^2 < 0 ~\text{and}~ p^0 > 0.
	\label{eq:forwardcone}
\end{equation}
This gives a strong constraint on local operators: eigenstates of $P^\mu$ can be constructed taking the Fourier transform of a local operator acting on the vacuum,
\begin{equation}
	\widetilde{\phi}(p) \ket{0}
	= \int d^dx \, e^{i p \cdot x} \phi(x) \ket{0}.
	\label{eq:momentumspace}
\end{equation}

\end{enumerate}
%
\begin{exercise}
	Using eq.~\eqref{eq:P:exponentiated}, show that 
	$$ 
		P^\mu \widetilde{\phi}(p) \ket{0}
		= p^\mu \widetilde{\phi}(p) \ket{0} 
	$$
\end{exercise}
%
Note that the operators $P^\mu$ and $M^{\mu\nu}$ are in fact topological charges (they can be defined on any codimension-1 surfaces, not only time slices), so the commutation relations \eqref{eq:commutator:P} and \eqref{eq:commutator:M} can be motivated in \emph{any} quantization of the theory, including on the Lorentzian cylinder with Hamiltonian $\frac{1}{2}(P^0 - K^0)$.
The points 1 to 4 in the list above can actually directly be used to define a Euclidean quantum field theory.
However, it is obvious from point 5 that the generator $P^0$ plays a special role in quantum field theory in Minkowski space-time: not all of the conformal generators are treated on the same footing.%
%
\footnote{On the other hand the existence of the conserved charge $M^{\mu\nu}$ is not needed: one can for instance consider quantum field theories in which the boost symmetry is broken}.

These five points are essentially the most general definition a non-perturbative quantum field theory. They are nearly in one-to-one correspondence with the so-called \emph{Wightman axioms}. The only difference (besides the level of mathematical rigor) is that the Wightman axioms do not rely on the existence of an energy-momentum tensor, but assume directly that the Poincaré transformations are realized as unitary transformations on the Hilbert space.

\subsection{Wightman functions}

With these axioms in mind, the next thing we can do is compute the vacuum expectation value of products of local operators,
\begin{equation}
	\bra{0} \phi_1(x_1) \cdots \phi_n(x_n) \ket{0}.
\end{equation}
This can be viewed as an overlap of the vacuum state ($\bra{0}$) with a  state created acting on the vacuum with a sequence of local operators. Note that these operators need not be time-ordered: the time evolution operator is unitary, so it can go both ways. This is in stark contrast with time-ordered correlation functions obtained from the path integral.

Correlators of this type are called \emph{Wightman functions}. They are really the fundamental observables in non-perturbative quantum field theory. In fact, it is even possible to completely define a quantum field theory just by its Wightman functions: the \emph{Wightman reconstruction theorem} states that the Hilbert space of a quantum field theory can be constructed from all its Wightman function.
A convenient perspective is therefore to forget about the Hilbert space and focus on correlation functions.

Symmetry is encoded in this correlation functions in the form of Ward identities. Given a conserved charge $G$ that annihilates the vacuum, $G \ket{0} = 0$ (this could be $P^\mu$ or $M^{\mu\nu}$), the following equation must be satisfied
\begin{align}
	& \bra{0} \left[ G, \phi_1(x_1) \right] \phi_2(x_2) \cdots 
	\phi_n(x_n) \ket{0}
	\nonumber \\
	& + \bra{0} \phi_1(x_1) \left[ G, \phi_2(x_2) \right] \cdots 
	\phi_n(x_n) \ket{0}
	\nonumber \\
	& + \ldots
	\nonumber \\
	& + \bra{0} \phi_1(x_1) \, \phi_2(x_2) \cdots 
	\left[ G, \phi_n(x_n) \right] \ket{0} = 0.
\end{align}
This is an equation that is obvious in the Hilbert space picture, but it is also valid as a differential equation for the Wightman function, since each commutator is again related to the local operator. Let us consider some examples

The simplest Wightman function is that involving a single scalar local operator, 
\begin{equation}
	\bra{0} \phi(x) \ket{0}.
\end{equation}
In this case the Ward identity associated with translations implies
\begin{equation}
	\frac{\partial}{\partial x^\mu}
	\bra{0} \phi(x) \ket{0} = 0,
\end{equation}
or in other words that the vacuum expectation value of the operator is a constant over all of space-time. Lorentz symmetry does not give more information about that constant, but it forbids a vacuum expectation value for any operator transforming non-trivially under the Lorentz group.

Let us consider next a Wightman 2-point function of identical scalar operators,
\begin{equation}
	\bra{0} \phi(x) \phi(y) \ket{0}.
\end{equation}
In this case, translation symmetry tells us that
\begin{equation}
	\left( \frac{\partial}{\partial x^\mu}
	+ \frac{\partial}{\partial y^\mu} \right)
	\bra{0} \phi(x) \phi(y) \ket{0}.
\end{equation}
If we think of the correlator as being a function of $x + y$ and $x - y$, then this Ward identities establishes that there is no dependence on the former, i.e.~
\begin{equation}
	\bra{0} \phi(x) \phi(y) \ket{0} = W(y - x),
	\label{eq:W}
\end{equation}
where $W$ denotes an arbitrary function.

In general, the consequences of translation symmetry are easier to see in momentum space, using the Fourier transform \eqref{eq:momentumspace} of the local operators. By this very definition, we can establish that
\begin{equation}
	\big[ P^\mu, \widetilde{\phi}(p) \big]
	= p^\mu \, \widetilde{\phi}(p),
	\label{eq:commutator:P:momentum}
\end{equation}
and therefore the Fourier transform of the Wightman 2-point function obeys
\begin{equation}
	(p^\mu + q^\mu)
	\bra{0} \widetilde{\phi}(p) \widetilde{\phi}(q) \ket{0}.
\end{equation}
From this, we conclude that
\begin{equation}
	\bra{0} \widetilde{\phi}(p) \widetilde{\phi}(q) \ket{0}
	= (2\pi)^d \delta^d(p + q) \widetilde{W}(q).
	\label{eq:Wtilde}
\end{equation}
$\widetilde{W}$ is another unknown function. As the notation suggests, it is actually the Fourier transform of $W$,
\begin{equation}
	\widetilde{W}(q) = \int d^dx \, e^{i p \cdot x} W(x).
\end{equation}
%
\begin{exercise}
	Verify that the Fourier transform of eq.~\eqref{eq:W}
	with respect to both $x$ and $y$ reproduces eq.~\eqref{eq:Wtilde}.
\end{exercise}



Taking into account Lorentz symmetry, one can also establish that the Wightman function $W(x)$ can only depend on the Lorentz-invariant distance $x^2$, although there is a subtlety: this can be a different function depending whether $x$ is space-like or time-like (future- or past-directed), as Lorentz transformations act separately on each of these regions.
In momentum space, the same arguments says that $\widetilde{W}(q)$ must be a function of $q^2$. In this case, the condition that only states of positive energy can exist requires that $\widetilde{W}(q)$ vanishes unless $q^2 < 0$ with $q^0 > 0$, and therefore therefore we can unambiguously write
\begin{equation}
	\widetilde{W}(q)
	= 2\pi \, \theta\left( q^0 - \left| \vec{q} \right| \right)
	\rho(-q^2),
	\label{eq:spectraldensity}
\end{equation}
where $\rho$ is a function of the positive quantity $-q^2$, and $\theta$ is the Heaviside step function ($\theta(a) = 0$ for $a < 0$, $\theta(a) = 1$ for $a > 1$).

The fact that the Wightman 2-point function \eqref{eq:Wtilde} is proportional to a delta function raises an important concern: in spite of their names, Wightman functions are \emph{not} functions but rather distributions (this is also part of the Wightman axioms: they are in fact \emph{tempered distributions}). Note also that the same function computes the overlap between the two states 
\begin{equation}
	\widetilde{\phi}(q) \ket{0}
	\qquad\text{and}\qquad
	\widetilde{\phi}(-p) \ket{0}
\end{equation}
(note that the Hermitian conjugation flips the sign of momenta). Therefore, the limit $p \to - q$ corresponds to the norm of either of these states. But this limit is clearly discontinuous, or the norm of the state infinite.
The resolution of this issue is that the objects $\phi(x)$ and its Fourier transform $\widetilde{\phi}(p)$ are not operators, but rather \emph{operator-valued distributions}. In other words, $\phi(x) \ket{0}$
and $\widetilde{\phi}(p) \ket{0}$ are \emph{not} states of the theory, as they have in fact infinite norm. Formally, these operator-valued distributions only make sense when they are integrated against test functions, defining
\begin{equation}
	\phi[f] = \int d^dx \, f(x) \phi(x),
\end{equation}
or 
\begin{equation}
	\widetilde{\phi}[\tilde{f}] = \int d^dp \, \tilde{f}(p)
	\widetilde{\phi}(p),
\end{equation}
where $f$ and $\widetilde{f}$ are Schwartz-class test functions (smooth functions decaying faster than any power at infinity).
When acting on the vacuum, these smeared operator give well-defined states, with finite norms. For instance, we have
\begin{align}
	\left\| \widetilde{\phi}[\tilde{f}] \ket{0} \right\|^2
	&= \int d^dp d^dq \tilde{f}^*(p) \tilde{f}(q)
	\bra{0} \widetilde{\phi}(-p) \widetilde{\phi}(q) \ket{0}
	\nonumber \\
	&= 2\pi
	\int\limits_{q^0 > \left| \vec{q} \right|} \!\!
	d^dq \left| \tilde{f}(q) \right|^2
	\rho(-q^2).
	\label{eq:norm}
\end{align}
Test functions will not appear further in these lectures. For physicists, they are mostly an annoyance that we prefer to avoid. However, it is important to know that there exists a mathematically rigorous way of dealing with Wightman functions.
For one thing, this gives a proper justification of why it is always fine to take the Fourier transform way and back between the position- and momentum-space representation, as tempered distribution (the proper name of such distributions) always admit a Fourier transform.
Note however that this is true of Wightman functions, but not of time-ordered correlators that will appear later!

\subsection{Spectral representation}

The norm \eqref{eq:norm} is also giving away important information: it is positive if and only if the function $\rho$ is positive,
\begin{equation}
	\rho(\mu^2) \geq 0
	\qquad\quad
	\forall \, \mu^2 > 0
\end{equation}
The function $\rho$ is in fact the \emph{spectral density} encountered in standard quantum field theory textbooks, where we can often find it in the form
\begin{equation}
	\bra{0} \phi(x) \phi(y) \ket{0}
	= 2\pi \int \frac{d^dk}{(2\pi)^d}
	\int\limits_0^\infty d\mu^2 \,
	e^{i k \cdot (x - y)} \theta(k^0) 
	\delta(k^2 + \mu^2) \rho(\mu^2).
	\label{eq:spectralrepresentation}
\end{equation}
The spectral density is an essential tool in non-perturbative quantum field theory. It can for instance be used to construct the time-ordered correlation function
\begin{equation}
	\langle \phi(x) \phi(y) \rangle_T
	\equiv 
	\theta(x^0 - y^0) \bra{0} \phi(x) \phi(y) \ket{0}
	+ \theta(y^0 - x^0) \bra{0} \phi(y) \phi(x) \ket{0}.
\end{equation}
Unlike the Wightman function, this is not a tempered distribution, because the $\theta$ functions are not differentiable at the origin.
Nevertheless, the time-ordered product admits the simple representation
\begin{equation}
	\langle \phi(x) \phi(y) \rangle_T
	= \int \frac{d^dk}{(2\pi)^d}
	\int\limits_0^\infty d\mu^2 \,
	e^{i k \cdot (x - y)}
	\frac{i}
	{ -k^2 - \mu^2 + i \varepsilon} \, \rho(\mu^2)
	\label{eq:KallenLehmann}
\end{equation}
where the limit $\varepsilon \to 0_+$ is understood.
This is the \emph{Källen-Lehmann representation} for the time-ordered 2-point function.
%
\begin{exercise}
	Derive the Källen-Lehmann representation.
	An elegant derivation is to first show that the 
	time-ordered 2-point function can be written as the difference
	between the Wightman function and the vacuum expectation value
	of a retarded commutator,
	$$
	\langle \phi(x) \phi(y) \rangle_T
	= 
	W(y - x)
	- \theta(y^0 - x^0)
	\bra{0} \left[ \phi(x), \phi(y) \right] \ket{0}.
	$$
	The next step is to Fourier transform both terms in $y$
	after setting $x = 0$.
	We know that the Wightman function \eqref{eq:Wtilde}
	has a nice Fourier transform
	$$
		\widetilde{W}(q)
		= 2\pi \int\limits_0^\infty d\mu^2 \,
		\theta(q^0) 
		\delta(q^2 + \mu^2) \rho(\mu^2),
	$$
	which can be equivalently written
	$$
		\widetilde{W}(q)
		= \int\limits_0^\infty d\mu^2 \,
		\theta(q^0) 
		\left[ \frac{i}{q^2 + \mu^2 + i \varepsilon}
		- \frac{i}{q^2 + \mu^2 - i \varepsilon} \right] \rho(\mu^2).
	$$
	The retarded commutator is only non-zero in the forward light cone
	in $y$, and therefore it also admits a Fourier transform
	that converges provided that give a small imaginary part to $q$.
	Compute this Fourier transform, and show that it is equal to
	$$
		\int\limits_0^\infty d\mu^2 \,
		\left[ \theta(q^0) \frac{i}{q^2 + \mu^2 + i \varepsilon}
		+ \theta(-q^0) \frac{i}{q^2 + \mu^2 - i \varepsilon}
		\right] \rho(\mu^2).
	$$
	The difference of these last two integrals can then easily be
	turned into the Källen-Lehmann representation
	\eqref{eq:KallenLehmann}.
\end{exercise}
%
The spectral representation for 2-point function gives familiar results in non-interacting theories. A massive scalar field has for instance the spectral density
\begin{equation}
	\rho(\mu^2) = \delta(\mu^2 - m^2),
\end{equation}
and from this we recover the known massive propagator
\begin{equation}
	\langle \phi(x) \phi(y) \rangle_T
	= \int \frac{d^dk}{(2\pi)^d}
	e^{i k \cdot (x - y)}
	\frac{i}
	{-k^2 - m^2 + i \varepsilon}.
\end{equation}
In an interacting theory, the spectral density will generically have a non-zero contribution above a certain threshold corresponding to particle production (see figure \ref{fig:spectraldensity}).

The discussion applied so far to a generic quantum field theory without conformal symmetry. Let us now examine the role of scale and special conformal invariance, starting with the former.


\subsection{Scale symmetry}

The assumption of scale invariance coincides with the existence of a third conserved charge besides $P^\mu$ and $M^{\mu\nu}$, namely the operator $D$.
We obtained before the commutator of a generic local operator with $M^{\mu\nu}$ assuming that it transforms in some irreducible representation of the Lorentz group at the origin $x = 0$
(an operator inserted at some other point is not in an irreducible representation because $P^\mu$ does not commute with $M^{\mu\nu}$).
Since $M^{\mu\nu}$ commutes with $D$, the same assumption can be made about scale: any local operator can be decomposed into irreducible representations of the group of scaling transformations, meaning that we can write
\begin{equation}
	\left[ D, \phi(0) \right] = -i \Delta \phi(0).
\end{equation}
$\Delta$ is called the \emph{scaling dimension} of the operator $\phi$ (each local operator of the theory has its own scaling dimension).
The factor of $i$ ensures that $\Delta$ is a real number when $\phi$ is a real operator.
As before, we can use eq.~\eqref{eq:P:exponentiated} to obtain the commutator at any other point $x$:
\begin{equation}
	\left[ D, \phi(x) \right]
	= -i \left( x^\mu \partial_\mu + \Delta \right) \phi(x).
	\label{eq:commutator:D}
\end{equation}
Note that this is precisely equivalent to the transformation rule \eqref{eq:freescalar:conformaltransformation:3} for a classical field:
the scaling dimension $\Delta$ coincides with the mass dimension of the operator $\phi$ in a free theory.

Using this new commutator and assuming that the vacuum state is invariant under scale transformations, a new Ward identity can be obtained for the Wightman 2-point function,
\begin{equation}
	\left( x^\mu \frac{\partial}{\partial x^\mu} 
	+ y^\mu \frac{\partial}{\partial y^\mu}  + 2\Delta \right)
	\bra{0} \phi(x) \phi(y) \ket{0} = 0,
\end{equation}
or equivalently
\begin{equation}
	\left( x^\mu \frac{\partial}{\partial x^\mu} + 2 \Delta \right) 
	W(x) = 0. 
\end{equation}
The corresponding condition on the momentum-space 2-point function
is obtained from the Fourier transform, using integration by parts, and we find
\begin{equation}
	\left( -q^\mu \frac{\partial}{\partial q^\mu} + 2 \Delta - d \right)
	\widetilde{W}(q) = 0. 
	\label{eq:2ptWardIdentity:scale}
\end{equation}
The solution to this equation consistent with the form \eqref{eq:spectraldensity} is unique, up to a constant multiplicative coefficient $C$,
\begin{equation}
	\widetilde{W}(q)
	= 2\pi C \, \theta\left( q^0 - \left| \vec{q} \right| \right)
	(-q^2)^{\Delta - d/2},
\end{equation}
implying that the spectral density is a power law,
\begin{equation}
	\rho(\mu^2) = C \, (\mu^2)^{\Delta - d/2}.
\end{equation}
This kind of spectral density is very different from that of a massive interacting theory: the operator $\phi$ has excitations of all masses.
At the same time, its simplicity is striking: it is characterized by a single number $\Delta$.
(see fig.~\ref{fig:})

It is instructive to compute the time-ordered 2-point function using the Källen-Lehmann representation: performing the integral over $\mu^2$, we find
\begin{equation}
	\langle \phi(x) \phi(y) \rangle_T
	= \frac{i \pi C}
	{\sin\left[ \pi \left( \Delta - \frac{d}{2} \right) \right]}
	\int \frac{d^dk}{(2\pi)^d}
	e^{i k \cdot (x - y)}
	\left( k^2 - i \varepsilon \right)^{\Delta - d/2}.
\end{equation}
The term $\left( k^2 - i \varepsilon \right)^{\Delta - d/2}$ in the integral looks like the propagator of a massless scalar field raised to an arbitrary power. This is in fact what we expect in perturbation theory when the beta function has a non-trivial fixed point: the renormalized 2-point function has logarithms that can be resummed into a power controlled by the anomalous dimension $\gamma$ of the field $\phi$,
\begin{equation}
	\frac{i}{-q^2} \left[ 1 + \gamma \log(-q^2) + \ldots \right]
	\approx i (-q^2)^{-1 + \gamma}.
\end{equation}
In this case, the scaling dimension of the scalar operator corresponding to the renormalized field is
\begin{equation}
	\Delta = \frac{d-2}{2} + \gamma.
\end{equation}
In the limit $\gamma \to 0$, we recover the free propagator mentioned above.
Note however that this requires the coefficient $C$ to obey
\begin{equation}
	C \approx \gamma = \Delta - \frac{d-2}{2}
	\label{eq:C:freelimit}
\end{equation}
in order to recover the ordinary, finite normalization of the propagator.
In this case, the spectral density obeys
\begin{equation}
	\rho(\mu^2) \approx \gamma \, (\mu^2)^{-1 + \gamma}
	\xrightarrow{\gamma \to 0} \delta(\mu^2).
\end{equation}
(This identity can be verified integrating both sides in $\mu^2$ and taking the limit $\gamma \to 0$ afterwards).
This is precisely the spectral density that is expected in the free scalar field theory. In this case (and this case only!), the operator $\phi$ describes a massless scalar particle.

It turns out that the case $\Delta = \frac{d-2}{2}$ is the lowest possible value for $\Delta$: for any $\Delta$ below that value, the spectral density is not integrable in the limit $\mu^2 \to 0$.
One might then also worry about the opposite limit $\mu^2 \to \infty$ in the integral: for any $\Delta > \frac{d}{2}$, the spectral density increases with $\mu^2$. However, remember that this spectral density is in fact a Wightman function, i.e.~a tempered distribution that should be understood as integrated against test functions that decay faster than any power at large $q^2$. Therefore, arbitrarily large values of $\Delta$ are possible.
The inequality
\begin{equation}
	\Delta \geq \frac{d-2}{2}.
	\label{eq:unitarybound:scalar}
\end{equation}
is known in the literature as the \emph{unitarity bound} for scalar operators. Note that any scalar operator that saturates this unitarity bound has
\begin{equation}
	\bra{0} \widetilde{\phi}(p) \widetilde{\phi}(q) \ket{0}
	\propto \delta(q^2),
\end{equation}
which implies
\begin{equation}
	q^2 \bra{0} \widetilde{\phi}(p) \widetilde{\phi}(q) \ket{0}
	= 0,
\end{equation}
or in position space
\begin{equation}
	\bra{0} \phi(x) \partial^2 \phi(y) \ket{0}.
\end{equation}
Since this is true for any $x$ and $y$, it implies that
\begin{equation}
	\partial^2 \phi(x) = 0
\end{equation}
is true as an operator equation. In other words, a theory in which $\Delta = \frac{d-2}{2}$ is a free field theory.

The simplicity of the momentum-space 2-point function in a scale-invariant theory also means that it can easily be Fourier transformed back to position space using
\begin{equation}
	W(x) = \int \frac{d^dq}{(2\pi)^d} e^{- i q \cdot x}
	\widetilde{W}(q).
\end{equation}
\begin{exercise}
	Perform the Fourier transform explicitly.
	Note that we can use the fact that the integral is Lorentz
	invariant to determine that $W(x)$ is in fact a function of $x^2$.
	Moreover, since the integrand only has support for $q$
	in the forward light-cone, this defines a function of $x$
	that is in fact analytic in $x$ as long as $\im x$ is contained in 
	the future light cone, since the integrand is then damped by the
	exponential $e^{q \cdot \im x}$ with $q \cdot \im x < 0$
	(this domain of analyticity is known as the ``future tube'').
	This means that we are free to evaluate the integral at a point 
	$x = (i \tau, 0)$, and then use $\tau^2 = x^2$ to recover the 
	general solution.
	The integral at that point is convergent for all $\Delta$
	satisfying the unitarity bound, and one finds
	$$
	W(\tau) = C \frac{2^{2\Delta} \Gamma(\Delta)
	\Gamma\left( \Delta - \frac{d-2}{2} \right)}
	{(4\pi)^{d/2}} \,
	\tau^{-2\Delta}
	$$
	This can be used to obtain the result below.
\end{exercise}
The result of this integral can be written as
\begin{equation}
	W(x) = \frac{C'}
	{\left[ -(x^0 + i \varepsilon)^2 + \vec{x}^2 \right]^\Delta}
\end{equation}
where the limit $\varepsilon \to 0_+$ is understood to make sense of the case in which $x^2 < 0$, and the coefficient $C'$ is related to $C$ by
\begin{equation}
	C' = \frac{2^{2\Delta} \Gamma(\Delta)
	\Gamma\left( \Delta - \frac{d-2}{2} \right)}
	{(4\pi)^{d/2}} C.
\end{equation}
Note that the proportionality factor is positive for all $\Delta$ satisfying the unitary bound \eqref{eq:unitarybound:scalar}. This implies that the 2-point correlation function is always decreasing with the distance, and not the other way around.

It is in fact customary in conformal field theory to normalize the scalar operator $\phi$ so that $C' = 1$. In this case, we have
\begin{equation}
	C = \frac{(4\pi)^{d/2}}
	{2^{2\Delta} \Gamma(\Delta)
	\Gamma\left( \Delta - \frac{d-2}{2} \right)}.
	\label{eq:standardnormalization}
\end{equation}
\begin{exercise}
	Verify that the the coefficient $C$ vanishes
	as in eq.~\eqref{eq:C:freelimit} in the limit $\Delta \to \frac{d-2}{2}$.
\end{exercise}

Finally, let us conclude the analysis of scale symmetry with a comment about one-point functions. We saw in the previous section that a constant vacuum expectation value for a scalar operator was compatible with Poincaré symmetry. However, the commutator \eqref{eq:commutator:D} requires then that $\Delta = 0$, which violates the unitarity bounds.
We conclude that all one-point functions must vanish in a scale-invariant theory.


\subsection{Special conformal symmetry}

As with scale symmetry, the presence of special conformal symmetry is associated with the existence of the conserved charges $K^\mu$ organized in a $d$-dimensional vector.
Unlike $D$, however, $K^\mu$ does not commute with $M^{\mu\nu}$, so it cannot be diagonalized at the point $x = 0$.
Nevertheless, we can use the conformal algebra to establish that, if $\phi$ is a local operator with scaling dimension $\Delta$, then $\left[ K^\mu, \phi \right]$ has scaling dimension $\Delta - 1$:
\begin{align}
	\big[ D, [ K^\mu, \phi(0) ] \big]
	&= \big[ K^\mu, [ D, \phi(0) ] \big]
	+ \big[ [ D, K^\mu ], \phi(0) \big]
	\nonumber \\
	&= \big[ K^\mu, -i \Delta \phi(0) \big]
	+ \big[ i K^\mu, \phi(0) \big]
	= -i (\Delta - 1) [ K^\mu, \phi(0) ]
\end{align}
This should be contrasted with the fact that the operator $\left[P^\mu, \phi(x)\right]$ has scaling dimension $\Delta + 1$,
\begin{equation}
	\big[ D, [ P^\mu, \phi(0) ] \big]
	= -i (\Delta + 1) [ P^\mu, \phi(0) ],
\end{equation}
consistent with the fact that the derivative $\partial^\mu$ has mass dimension $+1$.
The fact that $K^\mu$ lowers the scaling dimensions seems to be in contradiction with our findings of the last section that $\Delta$ is bounded below: given any local operator, one can always construct other local operators with arbitrarily smaller scaling dimension.

The only way out of this paradox is to assume that at some point the action of $K^\mu$ annihilates the operator. In other words, there exists some local operator such that
\begin{equation}
	\left[ K^\mu, \phi(0) \right] = 0.
\end{equation}
We call this local operator a \emph{primary}. Any other local operator obtained acting on a primary with $P^\mu$ is called a \emph{descendant}. Since the action of $P^\mu$ coincides with taking derivatives, a primary operator is simply an operator that cannot be written as the derivative of some other operator.
Unless specified otherwise, we shall from now on only consider Wightman correlation functions of primary operators. Descendants will be explicitly denoted with a derivative.

The transformation of a primary operator away from the origin can once again be obtained from eq.~\eqref{eq:P:exponentiated}. Note that since the commutator of $P^\mu$ and $K^\mu$ involves both $D$ and $M^{\mu\nu}$, this transformation depends on the scaling dimension and on the Lorentz representation of the operators, i.e.~on the eigenvalues $\Delta$ and $S^{\mu\nu}$. We find
\begin{equation}
	\left[ K^\mu, \phi(x) \right]
	= -i \left( 2 x^\mu x^\nu \partial_\nu - x^2 \partial^\mu 
	+ 2 \Delta x^\mu - 2 S^{\mu\nu} x_\nu \right) \phi(x).
	\label{eq:commutator:K}
\end{equation}
This equation also defines the commutator of a momentum-space operator: using integration by parts in the definition \eqref{eq:momentumspace}, one can show that this amounts to replacing $\partial_\mu \to -i q_\mu$ and $x^\mu \to -i \partial/ \partial q_\mu$,
so that
\begin{equation}
	\big[ K^\mu, \widetilde{\phi}(q) \big]
	= \left[ 2 \frac{\partial^2}{\partial q_\mu \partial q_\nu} q_\nu
	- \frac{\partial^2}{\partial q_\nu \partial q^\nu} q^\mu
	- 2 \Delta \frac{\partial}{\partial q_\mu}
	+ 2 S^{\mu\nu} \frac{\partial}{\partial q^\nu} \right]
	\widetilde{\phi}(q),
\end{equation}
or after permuting the derivatives with $q$,
\begin{equation}
	\big[ K^\mu, \widetilde{\phi}(q) \big]
	= \left[ 2 q^\nu \frac{\partial^2}{\partial q_\mu \partial q^\nu} 
	- q^\mu \frac{\partial^2}{\partial q_\nu \partial q^\nu}
	+ 2 (d - \Delta) \frac{\partial}{\partial q_\mu}
	+ 2 S^{\mu\nu} \frac{\partial}{\partial q^\nu} \right]
	\widetilde{\phi}(q).
\end{equation}
This is now a second-order differential acting on the operator expressed in momentum space. This is the price to pay for using a representation in which the commutator \eqref{eq:commutator:P:momentum} takes a simple diagonal form.

The first thing we can do with this commutator is to examine the related Ward identity for the Wightman 2-point function. Writing this function as
\begin{equation}
	W(x) = \bra{0} \phi(0) \phi(x) \ket{0},
\end{equation}
we can see that the commutator act trivially on the operator on the left, so that we must have (note that $S^{\mu\nu} = 0$ for scalar operators)
\begin{equation}
	\left( 2 x^\mu x^\nu \partial_\nu - x^2 \partial^\mu 
	+ 2 \Delta x^\mu \right) W(x) = 0.
\end{equation}
Using $W(x) = 1 /(x^2)^{\Delta}$ at space-like $x$, we have $\partial_\mu W(x) = - 2 \Delta W(x) x^\mu / x^2$, and therefore the differential equation is readily satisfied.
The same can be verified in momentum space: by definition, we have
\begin{equation}
	\widetilde{W}(q) = \bra{0} \phi(0) \widetilde{\phi}(q) \ket{0},
\end{equation}
where only the operator on the right is Fourier transformed while the one on the left is kept fixed at the origin, and so the commutator above implies that
\begin{equation}
	\left[ 2 q^\nu \frac{\partial^2}{\partial q_\mu \partial q^\nu} 
	- q^\mu \frac{\partial^2}{\partial q_\nu \partial q^\nu}
	+ 2 (d - \Delta) \frac{\partial}{\partial q_\mu} \right]
	\widetilde{W}(q) = 0.
\end{equation}
With $\widetilde{W}(q) = (-q^2)^{\Delta - d/2}$, this equation is again satisfied.

\paragraph{Distinct operators}

The fact that $W(x)$ and $\widetilde{W}(q)$ readily satisfy the constraint imposed by special conformal symmetry is very specific to identical scalar operators. In every other case, special conformal symmetry adds more constraints than Poincaré and scale symmetry alone.
The simplest example is that of a 2-point function of distinct scalar operators,
\begin{equation}
	\bra{0} \phi_1(x) \phi_2(y) \ket{0}.
\end{equation}
This is still a function of $(x-y)^2$ by Poincaré symmetry. But there are now two distinct scaling dimensions $\Delta_1$ and $\Delta_2$ corresponding to the operators $\phi_1$ and $\phi_2$, and the Ward identity for scale symmetry becomes (for simplicity setting $\phi_1$ at the origin)
\begin{equation}
	\left( x^\mu \frac{\partial}{\partial x^\mu}
	+ \Delta_1 + \Delta_2 \right)
	\bra{0} \phi_1(0) \phi_2(x) \ket{0} = 0.
\end{equation}
The solution is fixed up to a multiplicative constant to be
(assuming $x - y$ is space-like for simplicity)
\begin{equation}
	\bra{0} \phi_1(0) \phi_2(x) \ket{0}
	= \frac{C_{12}}{(x^2)^{(\Delta_1 + \Delta_2)/2}}.
\end{equation}
The Ward identity for special conformal transformation is obtained from the commutator \eqref{eq:commutator:K}, giving
\begin{equation}
	\left( 2 x^\mu x^\nu \partial_\nu - x^2 \partial^\mu 
	+ 2 \Delta_2 x^\mu \right)
	\bra{0} \phi_1(0) \phi_2(x) \ket{0}
	= 0.
\end{equation}
Using $\partial_\mu \left[ (x^2)^{-(\Delta_1 + \Delta_2)/2} \right] = -(\Delta_1 + \Delta_2) x^\mu / x^2$, this implies
\begin{equation}
	\left( \Delta_2 - \Delta_1 \right) C_{12} 
	\frac{x^\mu}{(x^2)^{(\Delta_1 + \Delta_2)/2}} = 0.
\end{equation}
When the scaling dimensions are different, $\Delta_1 \neq \Delta_2$, then the 2-point function must vanish. In conformal field theory, only primary operators of identical scaling dimensions can have non-zero 2-point functions.

In fact, if there are several scalar operators with the same scaling dimension $\phi_i$ with $i = 1, \ldots, N$, then 
\begin{equation}
	\bra{0} \phi_i(0) \phi_j(x) \ket{0}
	= \frac{C_{ij}}{(x^2)^{(\Delta_1 + \Delta_2)/2}},
\end{equation}
where $C_{ij}$ is a symmetric $N \times N$ matrix. By unitarity, this matrix must be positive-definite: if this were not the case, then one could define a negative-norm state by taking an appropriate linear combination of the $\phi_i$ and smearing. Therefore, it is always possible to choose a basis of operators in which $C_{ij}$ is diagonal. Moreover, the operators can be normalized so that $C_{ij} = \delta_{ij}$. From now on, we will therefore always assume that the only non-zero 2-point functions are those involving identical operators.


\paragraph{Operators with spin}

The other situation in which special conformal symmetry plays an essential role is for operators with spin.
Let us take as the simplest example a vector operator $A^\mu(x)$, and denotes its 2-point function by
\begin{equation}
	W^{\mu\nu}(x) = \bra{0} A^\mu(0) A^\nu(x) \ket{0}.
\end{equation}
As with scalars, one can also take the Fourier transform of this tempered distribution, defining
\begin{equation}
	\widetilde{W}^{\mu\nu}(p) = \int d^dx \, e^{i p \cdot x}
	\bra{0} A^\mu(0) A^\nu(x) \ket{0}
	= \bra{0} A^\mu(0) \widetilde{A}^\nu(p) \ket{0}.
	\label{eq:vector:mixedrep}
\end{equation}
Again, this function corresponds to the momentum-space correlation function without the delta-function imposing momentum conservation, namely
\begin{equation}
	\bra{0} \widetilde{A}^\mu(p) \widetilde{A}^\nu(q) \ket{0}
	= (2\pi)^d \delta^d(p + q) \widetilde{W}^{\mu\nu}(q).
\end{equation}
By Lorentz symmetry, this function of a single momentum can be decomposed into two different tensor structures multiplying scalar functions,
\begin{equation}
	\widetilde{W}^{\mu\nu}(p)
	= (p^\mu p^\nu - p^2 g^{\mu\nu}) \widetilde{W}_1(p)
	+ p^\mu p^\nu \widetilde{W}_0(p).
	\label{eq:vector:polarizations}
\end{equation}
Moreover, using scale symmetry and the spectral condition, we can infer that the functions $\widetilde{W}_{1,0}$ are just powers of $p^2$ over the forward light cone,
\begin{equation}
	\widetilde{W}_{1,0}(p)
	= \theta\left( p^0 - \left| \vec{p} \right| \right)
	(-p^2)^{\Delta - d/2 - 1} C_{1,0},
\end{equation}
where $\Delta$ is the scaling dimensions of the operator $A^\mu$ and $C_1$, $C_0$ two constants that cannot be related by scale and Poincaré symmetry only.

There is a good reason for using precisely the two tensor structures in eq.~\eqref{eq:vector:polarizations} and not, say, $g^{\mu\nu}$. Thanks to the spectral condition, it is always possible to choose a Lorentz frame in which $\vec{p} = 0$.%
%
\footnote{This is similar to going to a massive particle's rest frame in Wigner's group-theoretical approach to quantum field theory.}
%
In this frame, the momentum is invariant under the group $\SO(d-1)$ of spatial rotations, and therefore the 2-point function can be decomposed into irreducible representations of that group: The part proportional to $W_0$ only appears in the component $\widetilde{W}^{00}$, and it transforms like a scalar under rotations. Conversely, the part proportional to $W_1$ only has non-zero entries for spatial Lorentz indices; it is in fact proportional to the identity in the $d-1$ subspace, i.e.~it is the invariant tensor for the vector representation of $\SO(d-1)$.
In particle physics language, we would call these two parts respectively \emph{longitudinal} and \emph{transverse}.

Being able to use irreducible representations of $\SO(d-1)$ is an advantage of working in momentum space: there is no obvious Lorentz frame in which such a decomposition can be made in position space, since the 2-point function has support over all of Minkowski space-time. The disadvantage of working in momentum space is that the Ward identity for special conformal transformation is a second-order differential equation in $p$, while it is a first-order differential in position space.
Nevertheless, this Ward identity can be straightforwardly applied to eq.~\eqref{eq:vector:polarizations}, in which case it yields a relation between the longitudinal and transverse parts, i.e.~between the coefficients $C_0$ and $C_1$, given by (see exercise)
\begin{equation}
	C_0 = \frac{\Delta - d + 1}{\Delta - 1} \, C_1.
	\label{eq:vector:polarizationrelations}
\end{equation}
This is a very important consequences of special conformal symmetry: 
while in a scale-invariant theory the longitudinal and transverse polarizations are independent, in a conformal theory they are related.
%
\begin{exercise}
	Using the definition \eqref{eq:vector:mixedrep} for the function
	$\widetilde{W}^{\mu\nu}(p)$, show that it satisfies a special
	conformal Ward identity
	$$
	\left( 2 p^\beta
	\frac{\partial^2}{\partial p_\alpha \partial p^\beta}
	- p^\alpha
	\frac{\partial^2}{\partial p_\beta \partial p^\beta}
	+ 2 (d - \Delta) \frac{\partial}{\partial p_\alpha}
	+ 2 S^{\alpha\beta} \frac{\partial}{\partial p_\beta} \right)
	\widetilde{W}^{\mu\nu}(p) = 0,
	$$
	where $S^{\alpha\beta}$ is given by eq.~\eqref{eq:spinop:vector},
	and use this to prove the relation
	\eqref{eq:vector:polarizationrelations}.
\end{exercise}
%
This has consequences on the possible values that $\Delta$ can take. As before, this 2-point function computes the norm of a state, and its positivity requires:
\begin{itemize}

\item
$\Delta > \frac{d}{2}$ so that the 2-point function is integrable in all cases;

\item
$C_0$ and $C_1$ are both positive so that the norm is positive for any choice of external polarization vector (i.e.~the tensor $\widetilde{W}^{\mu\nu}$ is positive-definite). This requires that $\Delta - 1$ and $\Delta - d + 1$ have the same sign.

\end{itemize}
%
The combination of these two conditions in $d > 2$ dimensions (spin is treated differently in $d = 2$) imply
\begin{equation}
	\Delta \geq d - 1.
\end{equation}
This is known as the unitarity bound for a vector operator.

As in the scalar case, something special happens when the unitarity bound is saturated ($\Delta = d - 1$). In this case the 2-point function has no longitudinal component, $C_0 = 0$,
and $\widetilde{W}^{\mu\nu}$ vanishes when contracted with $p_\mu$ or $p_\nu$. This implies that the longitudinal part of the state is null,
\begin{equation}
	p_\mu \widetilde{A}^\mu(p) \ket{0} = 0,
\end{equation}
or equivalently that $\widetilde{A}^\mu(p) \ket{0}$ is a state that only describes transverse excitations.
The position-space equivalent is that 
\begin{equation}
	\partial_\mu A^\mu(x) \ket{0} = 0,
\end{equation}
or in other words that $A^\mu$ is a conserved current.
The equivalence goes both way: any vector operator with $\Delta = d - 1$ is a conserved current, and any conserved current must have scaling dimension $\Delta = d - 1$. This also shows that conserved currents are  primary operators: they cannot be written as $\partial^2$ acting on another current (that current would have $\Delta = d - 3$), nor as $\partial^\mu \phi$ where $\phi$ is a scalar operator, because the conservation requirement implies $\partial^2 \phi = 0$, which is only possible if $\phi$ has scaling dimensions $d/2 - 1$ and thus the current $\Delta = d/2$.


The fact that 2-point functions of primary operators are completely fixed by conformal symmetry up to a coefficient that corresponds to a choice of normalization is not specific to scalar and vector operators. In fact, any local operator specified by a representation under the Lorentz group and a scaling dimension defines an irreducible representation of the conformal group $\SO(d,2)$, and as such its 2-point function is fixed by group theory (up to a conventional normalization). This also explains on more general grounds why 2-point functions of distinct operators vanish.
The construction of all unitary representations of the conformal group in $d = 4$ dimensions was performed by Mack in 1975~\cite{Mack:1975je}, and similar constructions can be done in other dimensions. 
Some Lorentz representations are specific to a given dimension $d$, and others exist in any $d$, like the scalar and all symmetric traceless representations with spin $\ell$, of which the vector discussed above is a special case corresponding to $\ell = 1$. All such tensors satisfy the unitarity bound
\begin{equation}
	\Delta \geq d - 2 + \ell,
	\label{eq:unitaritybound:symmetricops}
\end{equation}
and they are in general described by $\ell + 1$ distinct polarizations (irreducible representations of rotation group), except when the bound is saturated, where there is just a single, transverse polarization. In this case the operators are higher-spin conserved currents.
In terms of representations of the conformal group, generic operators are said to belong to \emph{long multiplets}, whereas special cases such as scalars with $\Delta = d/2 - 1$ or symmetric tensors with $\Delta = d - 2 + \ell$ are said to be in \emph{short multiplets} (there are fewer states). 

%note about representation theory: 
%one way of constructing the representations of the conformal group is to diagonalize the largest possible subgroup;
%this is what was done by diagonalizing both scale and Lorentz generators at the point $x = 0$;
%note that in Euclidean space this is also the largest compact subgroup: $\SO(d) \times \SO(1,1) \subset \SO(d + 1, 1)$ (true?)
%
%however, to impose unitarity (positivity of norm) it is better to work in momentum space, where we diagonalize $P^\mu$

\begin{exercise}
	(difficult)
	Construct explicitly the 2-point function of a 2-index
	symmetric traceless operator $B^{\mu\nu}(x)$.
	As a starting point, let us decompose the momentum-space
	correlation functions into tensors that transform covariantly
	under rotations in the rest frame. Using the transverse projector 
	$$ g_\perp^{\mu\nu} = g^{\mu\nu} - \frac{p^\mu p^\nu}{p^2}$$
	satisfying $p_\mu g_\perp^{\mu\nu} = 0$, this can be done as
	\begin{align*}
		\bra{0} B^{\mu\nu}(0) \widetilde{B}^{\rho\sigma} \ket{0}
		= \bigg[ & \frac{1}{2}
		\left( g_\perp^{\mu\rho} g_\perp^{\nu\sigma}
		+ g_\perp^{\mu\sigma} g_\perp^{\nu\rho} - \text{traces} \right)
		C_2
		\\
		& + \frac{1}{4}
		\left( g_\perp^{\mu\rho} \frac{p^\nu p^\sigma}{p^2}
		+ \text{permutations} \right) C_1
		\\
		& + \frac{p^\mu p^\nu p^\rho p^\sigma}{(p^2)^2}
		C_0 \bigg]
		\theta\left( p^0 - \left| \vec{p} \right| \right)
		(-p^2)^{\Delta - d/2 + 1}.
	\end{align*}
	Then write down the Ward identity for special conformations
	(including the spin operator for a 2-index tensor),
	and show that it leads to the conditions
	\begin{align*}
		C_1 &= 2 \frac{\Delta - d}{\Delta} \, C_0,
		\\
		C_2 &= \frac{d}{d-1} \frac{(\Delta - d) (\Delta - d + 1)}
		{\Delta (\Delta - 1)}.
	\end{align*}
	Argue that this gives rise to the unitarity bound $ \Delta \geq d$,
	in agreement with eq.~\eqref{eq:unitaritybound:symmetricops}.
	Conclude that the energy-momentum tensor $T^{\mu\nu}$
	is a primary operator with $\Delta = d$.
\end{exercise}


\subsection{UV/IR divergences and anomalies}

The discussion has been focused so far on Wightman functions. Beside having a Hilbert-space interpretation and satisfying conformal Ward identities that give strong constraints on their possible form, Wightman functions are also free of divergences.
In momentum space, regularity at small momenta (IR) is enforced by the unitarity bound, whereas the power-law growth at large momenta (UV) is compatible with the damping provided by Schwartz-class test functions.
In position space, the apparent singularity at short distance (UV) is in fact resolved by $i \varepsilon$ that provides a unequivocal prescription for deforming a contour of integration.

These properties are not true of time-ordered products. Let us consider for instance once again the scalar 2-point function. Using the standard CFT normalization \eqref{eq:standardnormalization} and the Källen-Lehmann representation, we find
\begin{equation}
	\langle \phi(x) \phi(y) \rangle_T
	= -i
	\frac{(4\pi)^{d/2} \Gamma\left( \frac{d}{2} - \Delta \right)}
	{2^{2\Delta} \Gamma(\Delta)}
	\int \frac{d^dk}{(2\pi)^d}
	e^{i k \cdot (x - y)}
	\left( k^2 - i \varepsilon \right)^{\Delta - d/2}.
	\label{eq:UVdivergence}
\end{equation}
This expression diverges whenever $\Delta = \frac{d}{2} + n$ with integer $n$ due to the $\Gamma$-function multiplying the integral.
This corresponds precisely to a case in which the 2-point function has scaling dimension $d + 2n$, and is therefore compatible with contact terms of the form
\begin{equation}
	(\partial^2)^n \delta^d(x - y).
\end{equation}
These contact terms are obviously covariant under Poincaré and scale transformations, so the can be added to the correlation function without affecting the Ward identities at separated points.
This is the only possible contact term appearing in a scalar 2-point function, but more terms can appear in higher-point correlation functions, also involving operators that carry spin.

When Fourier transformed to momentum space, all such contact terms become polynomials in the momenta. In the scalar 2-point function case, these are $(p^2)^n$. Polynomial terms are incompatible with the spectral condition for the Wightman function, as they have support over all causal regions. But there are allowed (and in fact required) in time-ordered products. Time-ordered products are defined in position space as the product of Wightman functions, which are tempered distributions, with step functions, which are not: therefore they do not necessarily have a Fourier transform. Contact terms can be seen as a way of ``fixing'' the time-ordered product so that they can be Fourier transformed. 

In path integral language, this is the situation in which the source field $J$ for the operator $\phi$ has scaling dimension, $d - \Delta = \frac{d}{2} - n$ and therefore contact terms of the form $J (\partial^2)^n J$ can (and must) be added to the action. This can be understood by analytic continuation in scaling dimension $\Delta$. Let us assume that our scalar operator has scaling dimension $\Delta = \frac{d}{2} + n - \epsilon$, and take the limit $\epsilon \to 0$ (note that this $\epsilon$ is different from the one in the $i \varepsilon$ limit). Then the time-ordered 2-point function in momentum space takes the form
\begin{equation}
	\Gamma(-n + \epsilon) (p^2)^{n - \epsilon}
	+ Z (p^2)^n,
\end{equation}
where the first term with a pole in $\epsilon$ comes from the expression \eqref{eq:UVdivergence} valid at $x \neq y$, and the second term is the counterterm added to the action.
Choosing $Z \propto \epsilon^{-1}$ allows to cancel the divergence as $\epsilon \to 0$, but it also gives rise to a logarithm, 
\begin{equation}
	(p^2)^n \log\left( \frac{p^2}{\mu^2} \right).
\end{equation}
The fact that we need to introduce a dimensionful quantity is the sign of a \emph{conformal anomaly}: a 2-point function of this form does not  satisfy the Ward identity for scale transformations \eqref{eq:2ptWardIdentity:scale},
\begin{equation}
	\left[ -p^\mu \frac{\partial}{p^\mu} + 2n \right]
	(p^2)^n \log\left( \frac{p^2}{\mu^2} \right)
	= -2 (p^2)^n \neq 0.
\end{equation}
However, the anomalous term on the right-hand side is a polynomial in the momenta, corresponding to a contact term, indicating that the anomaly is local. In QFT language, this is typical of a (renormalized) UV divergence.

Note that the presence of contact term is associated with a very special type of operators of (half-)integer scaling dimension. Such operators are generically absent in an interacting conformal field theory, where the scaling dimensions take arbitrary real values. 
However, an exception to this rule concerns conserved currents in even space-time dimension $d$. For instance, a conserved current in $d = 4$ has scaling dimensions $\Delta = 3$; it is therefore associated in the path integral language with a source $a_\mu$ with scaling dimension 1. This source is also subject to a gauge symmetry, $a_\mu \sim a_\mu + \partial_\mu \alpha$, and therefore a possible contact term is generated by $f_{\mu\nu} f^{\mu\nu}$, where $f_{\mu\nu} = \partial_\mu a_\nu - \partial_\nu a_\mu$. As in the scalar case, this term must be used as a counterterm to cancel a divergence arising in the time-ordered correlation function, leading generically to logarithms in correlation functions involving the transverse polarization of the current.

Note however that time-ordered correlation functions involving a conserved current do not only have transverse polarization: while the conservation condition implies the vanishing of the state
\begin{equation}
	\partial_\mu J^\mu(x) \ket{0} = 0,
\end{equation}
and thus of Wightman correlation functions constructed from that state, this is not true of the divergence of $J^\mu$ appearing in a time-ordered product: the conservation equation $\partial_\mu J^\mu(x) = 0$ is only true as an \emph{operator equation}, namely away from coincident points. On general grounds, one expects
\begin{align}
	\langle \phi_1(x_1) \cdots \phi_n(x_n)
	\partial_\mu J^\mu(y) \rangle_T
	&= - \delta^d(x_1 - y)
	\langle \delta\phi_1(x_1) \cdots \phi_n(x_n)  \rangle_T
	\nonumber \\
	& \quad - \ldots
	\nonumber \\
	& \quad -\delta^d(x_n - y)
	\langle \phi_1(x_1) \cdots \delta\phi_n(x_n)  \rangle_T,
\end{align}
where $\delta\phi_i$ indicates the charge of the field $\phi$ under $J^\mu$.%
%
\footnote{Note that it is conventional to normalize the conserved current $J$ so that this equation is true, which means that the 2-point function of $J$ cannot be arbitrarily normalized like a scalar operator in eq.~\eqref{eq:standardnormalization}. The same is true of the energy-momentum tensor, and of higher-spin conserved currents. For all of these operator the normalization of the 2-point function involves a coefficient of physical relevance.}
%

In the case of 2-point functions, however, the divergence of the current is zero because there are no 1-point correlation functions in conformal field theory. Therefore, for a conserved current the representation~\eqref{eq:vector:polarizations} is still valid
\begin{align}
	\bra{0} J^\mu(p) J^\nu(q) \ket{0}
	&= (2\pi)^d \delta^d(p + q) 
	\nonumber \\
	& \quad \times C_J
	\theta\left( q^0 - \left| \vec{q} \right| \right)
	( q^\mu q^\nu - q^2 g^{\mu\nu} )
	(-q^2)^{d/2 - 2}.
\end{align}
In $d = 4$ dimensions, the vanishing power of $q^2$ indicates the presence of a logarithm in the time-ordered correlator,
\begin{equation}
	\langle J^\mu(p) J^\nu(q) \rangle_T
	\propto (2\pi)^4 \delta^4(p + q)
	( q^\mu q^\nu - q^2 g^{\mu\nu} )
	\log\left( \frac{q^2}{\mu^2} \right).
\end{equation}
As in the scalar case, this is the manifestation of a UV divergence. However, the logarithm also implies that the limit $q^2 \to 0$ diverges: this is what we would call an IR divergence in quantum field theory. 
In the context of CFT, there is no clear distinction between UV and IR divergences, as the two are closely related. They both arise from an ambiguity in taking the Fourier transform. UV divergences can be cured with local counterterms, at the price of introducing a reference scale, but IR divergences are not.


%%%%%%%%%%%%%%%%%%%%%%%%%%%%%%%%%%%%%%%%%%%%%%%%%%%%%%%%%%%%%%%%%%%%%%

\section{Conformal correlation functions}
\label{sec:correlators}

The method presented in the last section using conformal Ward identities and the Wightman axioms could in principle be used to determine 3- and higher-point correlation functions.%
%
\footnote{The conformal Ward identities give differential equations and the Wightman axioms boundary conditions. Together, these are sufficient to constrain the conformal correlation functions. Note however that the spectral condition alone is not sufficient: the micro-causality axiom is required as well (see e.g.~ref.~\cite{Gillioz:2021sce} for a discussion at the level of 3-point functions).}
%
However, it is highly inconvenient, and it hides the simplicity of the result. To put things in perspective, note that the Wightman 3-point function of scalar operators in momentum space was only constructed in 2019~\cite{Gillioz:2019lgs}, whereas the position-space correlator has been known since the work of Polyakov in 1970~\cite{Polyakov:1970xd} using the Euclidean construction presented in this section.


\subsection{From Minkowksi space-time to Euclidean space}

For a start, let us go back to the Wightman 2-point function of scalar operators, given in position space by
\begin{equation}
	W(x) = \frac{1}{\left[ -(x^0 + i \varepsilon)^2
	+ \vec{x}^2 \right]^\Delta}.
\end{equation}
The very definition of this 2-point function includes an $i \varepsilon$ prescription with its time component and suggests therefore that $x^0$ should be thought of as a complex variable: as a function of a complex $x^0$, $W$ is analytic in the upper-half complex plane. Taking $x^0$ to be purely imaginary, i.e.
\begin{equation}
	x^0 = i \tau,
	\qquad\qquad
	\tau > 0,
\end{equation}
we obtain the \emph{Schwinger function}
\begin{equation}
	\langle \phi(0) \phi(x) \rangle
	= \frac{1}{( \tau^2 + \vec{x}^2 )^\Delta}
	\equiv \frac{1}{(x_E^2)^\Delta}.
	\label{eq:2pt:Schwinger}
\end{equation}
We denote with $x_E = (\tau, \vec{x})$ the Euclidean vector that is contracted with the $d$-dimensional Euclidean metric. Schwinger functions will always be written using the ``average'' notation $\langle \cdots \rangle$, as opposed to the Wightman functions $\bra{0} \cdots \ket{0}$ that can be interpreted as vacuum expectation values; since these are Euclidean functions by definition, we denote the position of the operator by $x$ instead of $x_E$, even though the latter is implied.

Note that the Schwinger function transforms covariantly under the Euclidean conformal group $\SO(d + 1, 1)$ which is obtained by replacing the Minkowski metric by the Euclidean one. This includes both $\SO(d)$ rotations and translations, and as a consequence we have
\begin{equation}
	\langle \phi(0) \phi(x) \rangle
	\stackrel{\text{rotation}}{=} \langle \phi(0) \phi(-x) \rangle
	\stackrel{\text{translaiton}}{=} \langle \phi(x) \phi(0) \rangle,
\end{equation}
showing that it is symmetric under the exchange of the order of the two operators.
Also note that the Schwinger 2-point function is positive, and that it is not defined at the point $x_E = 0$, unlike the Wightman function whose $i \varepsilon$ prescription indicates how to approach any null point.
%
\begin{exercise}
	There is another way to get to the same result, starting from
	the momentum-space representation of the 2-point function. 
	The Wightman function is not well-suited to do so
	(at least not without exploring its analyticity properties 
	following from the micro-causality axiom),
	but the time-ordered function is: starting from the
	Källen-Lehmann representation~\eqref{eq:UVdivergence},
	which in momentum space becomes (using a notation in which the 
	momentum-conservation delta function is implicit)
	$$
	\langle \phi(-p) \phi(p) \rangle_T
	= -i
	\frac{(4\pi)^{d/2} \Gamma\left( \frac{d}{2} - \Delta \right)}
	{2^{2\Delta} \Gamma(\Delta)}
	\left( p^2 - i \varepsilon \right)^{\Delta - d/2},
	$$
	one can perform a Wick rotation in which $x^0$ and $p^0$
	are simultaneously rotated in the complex plane in opposite
	directions, to arrive at the Euclidean result
	$$
	\langle \phi(-p) \phi(p) \rangle
	= 
	\frac{(4\pi)^{d/2} \Gamma\left( \frac{d}{2} - \Delta \right)}
	{2^{2\Delta} \Gamma(\Delta)}
	\left( p^2  \right)^{\Delta - d/2}.
	$$
	Perform the Fourier transform in $p$ to recover the 
	Schwinger function \eqref{eq:2pt:Schwinger}.
	Note that you will need to make assumptions about $\Delta$
	for the Fourier integral to converge. The result is 
	however analytic in $\Delta$, so you can argue a posteriori
	that it must be valid for all scaling dimensions.
\end{exercise}

This construction of a Euclidean function by analytic continuation from the Wightman function can in fact be generalized to any number of operator insertions. Consider the $n$-point Wightman function parameterized as
\begin{equation}
	\bra{0} \phi_n(x_n - x_{n-1}) \cdots
	\phi_2(x_2 - x_1) \phi_1(x_1) \ket{0},
\end{equation}
and complexify all time components $x_i^0$. This function of several complex variables (and many more real ones) is in fact analytic in every upper-half complex plane in $x_1^0$ to $x_n^0$, because it can be written as a Fourier transform of a function that only has support when the dual variable $p_1^0$ to $p_n^0$ are all positive. Therefore, going to purely imaginary times $x_i^0 = i \tau_i$, one obtains the Schwinger $n$-point function
\begin{equation}
	\langle \phi_n(x_n - x_{n-1}) \cdots
	\phi_2(x_2 - x_1) \phi_1(x_1) \rangle,
\end{equation}
in which the Euclidean times all satisfy $\tau_i > 0$, i.e.~the operators are ordered along the Euclidean time direction. 
This latter observation is however irrelevant, because the ordering of operators does not matter in a Schwinger function: the analytic continuation from real to imaginary Minkowski times can be performed starting with a configuration in which all real Minkowski times are equal, in which case the operators commute by micro-causality. 
The observation made on the 2-point function is therefore valid more generally: 
\begin{itemize}

\item
Schwinger functions are symmetric under the exchange of operators.

\end{itemize} 
The other observations made before are also true in general: 
\begin{itemize}

\item
Schwinger functions transform covariantly under the Euclidean conformal group $\SO(d+1, 1)$. This property will be very useful because it means that we can use finite conformal transformations that act nicely in Euclidean space (including $\infty$ as a point) to simplify our computations.

\item
Schwinger functions are ill-defined at coincident points. This is not a bug but a feature: using functions that are only defined at separated points means that we do not need to worry about contact terms and UV divergences.%
%
\footnote{It is in fact possible to describe contact terms (local and semi-local ones) using a generalization of the embedding formalism described in the next section~\cite{Nakayama:2019mpz}. But these contact terms are ambiguous: they carry more information than the Schwinger or Wightman functions themselves, and as such are unphysical.}
%
It means however that we cannot simply take the Fourier transform of these functions to obtain momentum-space Schwinger functions. From now on we shall focus on position space. 

\item
Schwinger functions enjoy a property called \emph{reflection positivity}: if the operators are organized in a configuration that is invariant under reflection across a plane (e.g.~four points on the edges of a square; or trivially any two points), then the correlator is positive.%
%
\footnote{This property would not be needed if we had been studying Euclidean conformal field theory from the start. Indeed, there are interesting critical fixed points in condensed matter or statistical physics that are described by CFTs that are not reflection-positive (often called non-unitary).
However, the conformal bootstrap described in section~\ref{sec:bootstrap} relies on this property in an essential way.}
%

\end{itemize}

So far we have only showed that the Wightman functions let us define Schwinger functions by analytic continuation. But it turns out that the opposite is also true: the Osterwalder-Schrader theorem states that the properties of Schwinger functions listed above are sufficient to reconstruct Wightman functions.%
%
\footnote{There is in fact another property that is needed in the proof, called \emph{linear growth condition}. This property is very difficult to establish in quantum field theory. But within conformal field theory it has recently been shown that the linear growth condition can be replaced by a set of ``bootstrap axioms'', which are otherwise equivalent with the Osterwalder-Schrader axioms, and from which the Wightman axioms can be recovered~\cite{Kravchuk:2020scc, Kravchuk:2021kwe}.}
%
This means that we can in fact focus on the Euclidean Schwinger functions for all our purposes, as any other physical observable can be reconstructed from them (we did not say that this reconstruction is necessarily easy, though).

The only piece of information that we shall take with us from the analysis of section~\ref{sec:quantum} is the unitarity bound on the scaling dimension of operators. Working out these bounds purely from the Schwinger functions is possible, but it the procedure is not particularly enlightening.


\subsection{From Euclidean space to embedding space}

Once we are dealing with correlation functions that transform under the Euclidean conformal group $\SO(d + 1, 1)$, it makes a lot of sense to try and make use of our particle physicist's understanding of the Lorentz group in $d+2$ dimensions to gain mileage. We already introduced in section~\ref{sec:classical} a set of coordinates in this  $(d+2)$-dimensional, \emph{embedding} space: the line element obeys
\begin{equation}
	\eta_{MN} dX^M dX^N
	= (dX^\mu)^2 + (dX^{d+1})^2 - (dX^{d+2})^2.
\end{equation}
Note that the directions labeled with indices $\mu$ are all space-like,  while the time-like direction is $X^{d+2}$ (this is arguably confusing).

The question is how do we go from $X^M \in \mathbb{R}^{d+1,1}$ to $x^\mu \in \mathbb{R}^d$ without explicitly breaking the Lorentz symmetry. This can be done in two steps:
%
\begin{enumerate}

\item
Restrict our attention to the future light-cone $X^2 = 0$ with $X^{d+2} > 0$, which is an invariant subspace.

\item
Identify any two points related by a scale transformation on this light cone, i.e.~$X^M \sim \lambda X^M$ with $\lambda > 0$.

\end{enumerate}
%
This means that we are essentially considering a map between a point $x^\mu$ in $d$-dimensional Euclidean space and a light-ray in a $(d + 2)$-dimensional Minkowski space-time.
To make the map explicit, we chose a section of the cone, which we will take to be $X^{d+1} + X^{d+2} = 0$,
and identify
\begin{equation}
	X^\mu = x^\mu,
	\qquad
	X^{d+1} = \frac{1 - x^2}{2},
	\qquad
	X^{d+2} = \frac{1 + x^2}{2}.
	\label{eq:embedding:section}
\end{equation}
Conformal transformations now act linearly in embedding space,
\begin{equation}
	X^M \to X'^M = \Lambda^M_{~N} X^N
\end{equation}
To obtain the action on $x^\mu$, we first map it to our preferred section using eq.~\eqref{eq:embedding:section}, apply the Lorentz transformation on $X^M$, then perform a (space-time dependent) rescaling $X'^M \to \lambda(X') X'^M$ to get back on the section, and read off $x'^\mu$ from the coordinate $\lambda(X') X'^\mu$.
%Note that the rescaling in the last step is equivalent to a Weyl transformation of the metric.

Local operators in Euclidean space must also be lifted to the embedding space, or at least to the null cone. On our preferred section, we declare
\begin{equation}
	\phi(X) \equiv \phi(x),
	\qquad 
	\left( X^2 = 0, ~ X^{d+1} + X^{d+2} = 1 \right).
\end{equation}
Then primary operators are defined on the rest of the cone by the scaling rule
\begin{equation}
	\phi(\lambda X) = \lambda^{-\Delta} \phi(X).
	\label{eq:embeddingconescaling}
\end{equation}
Note that this rule can only apply to \emph{primary} operators: descendants obtained acting with derivatives will not satisfy the same scaling property.
This choice gives the right transformation rules for primary operators under all infinitesimal conformal transformations. This can be verified explicitly (see the exercise below), or argued as follows: a conformal transformation is a composition of a $(d+2)$-dimensional Lorentz transformation, which locally acts on the operator like a rotation or a boost, followed by a position-dependent scale transformation to get back to the preferred section, which by the rule \eqref{eq:embeddingconescaling} amounts to a local scale transformation with weight given by the scaling dimension $\Delta$ of the operator. The combination of these two local transformations is precisely what we expect from the conformal transformation of an operator.
%
\begin{exercise}
	Verify that a Lorentz boost in the direction of $X^{d+1}$
	corresponds to a scale transformation in Euclidean space,
	in agreement with eq.~\eqref{eq:embeddingspacealgebra},
	and that operators transform accordingly. 
\end{exercise}

A bit more care is required to define operators that carry spin on the projective null cone, as the additional Lorentz indices in the directions of $X^{d+1}$ and $X^{d+2}$ imply that there is more degrees of freedom that need to be examined. This can be done imposing transversality in embedding space, together with a gauge symmetry condition, but we do want to dive into this here. We will therefore restrict our attention to 3- and higher-point functions of scalar primary operators only.

Once the transformation properties of local operators are clear, the construction of correlation functions of $n$ points $X_1$ to $X_n$ follows two simple rules:
\begin{itemize}

\item
The correlators must depend on Lorentz-invariant quantities in embedding space, i.e.~scalar products of the form $X_i \cdot X_j$. Since $X_i^2 = 0$ on the null cone, only scalar products with $i \neq j$ can appear.

\item
Applying a local scale transformation $X^M \to \lambda(X) X^M$ under which all primary operators $\phi_i$ with scaling dimension $\Delta_i$ satisfy
\begin{equation}
	\phi_i(X_i) \to \lambda(X_i)^{-\Delta_i} \phi_i(X_i),
\end{equation} 
the correlator must transform homogeneously as
\begin{equation}
	\langle \phi_1(X_1) \cdots \phi_n(X_n) \rangle
	\to \lambda(X_1)^{-\Delta_1} \cdots \lambda(X_n)^{-\Delta_n}
	\langle \phi_1(X_1) \cdots \phi_n(X_n) \rangle.
\end{equation}


\end{itemize}
%
This rules imply immediately that there cannot be one-point functions, as there simply is no corresponding Lorentz-invariant quantity on the projective null cone. The simplest non-trivial case is that of a 2-point function, which must obey
\begin{equation}
	\langle \phi(X_1) \phi(X_2) \rangle
	\propto (X_1 \cdot X_2)^{-\Delta}.
\end{equation}
Note that this is only consistent with the homogeneous scaling rule if both operators have the same scaling dimension, another property that was derived in the hard way in section~\ref{sec:quantum}.
To recover the dependence on the $d$-dimensional Euclidean coordinates, one simply needs to use the identification \eqref{eq:embedding:section}, yielding
\begin{equation}
	X_1 \cdot X_2
	= x_1 \cdot x_2 - \frac{1}{2} \left(x_1^2 + x_2^2 \right)
	= -\frac{1}{2} (x_1 - x_2)^2.
\end{equation}
Upon fixing the proportionality factor in the above result, one recovers the expected result
\begin{equation}
	\langle \phi(x_1) \phi(x_2) \rangle = 
	\frac{1}{\left[ (x_1 - x_2)^2 \right]^\Delta}.
\end{equation}

\subsection{3-point functions}

The embedding space formalism becomes very interesting when examining the 3-point function
\begin{equation}
	\langle \phi_1(X_1) \phi_2(X_2) \phi_3(X_3) \rangle,
\end{equation}
where the 3 scalar operators have now a priori distinct scaling dimensions $\Delta_1$, $\Delta_2$ and $\Delta_3$.
Solving the conformal Ward identities for this 3-point function would be an annoying task. Instead, by the above rules we immediately know that this is a function of the only 3 invariant quantities 
\begin{equation}
	X_1 \cdot X_2,
	\qquad
	X_1 \cdot X_3,
	\qquad
	X_2 \cdot X_3.
\end{equation}
Moreover, by the homogeneous scaling rule, the only possible form of the 3-point function is
\begin{equation}
	\langle \phi_1(X_1) \phi_2(X_2) \phi_3(X_3) \rangle
	\propto
	(X_1 \cdot X_2)^{\alpha_{12}}
	(X_1 \cdot X_3)^{\alpha_{13}}
	(X_2 \cdot X_3)^{\alpha_{23}}
\end{equation}
with the exponents satisfying 
\begin{align}
	\alpha_{12} + \alpha_{13} &= - \Delta_1,
	\nonumber \\
	\alpha_{12} + \alpha_{23} &= - \Delta_2,
	\\
	\alpha_{13} + \alpha_{23} &= - \Delta_3.
	\nonumber
\end{align}
This system of equations admits as unique solution
\begin{align}
	\alpha_{12} &= - \frac{\Delta_1 + \Delta_2 - \Delta_3}{2},
	\nonumber \\
	\alpha_{13} &= - \frac{\Delta_1 + \Delta_3 - \Delta_2}{2},
	\\
	\alpha_{23} &= - \frac{\Delta_2 + \Delta_3 - \Delta_1}{2}.
\end{align}
In terms of Euclidean coordinates, this can be written
\begin{equation}
	\langle \phi_1(x_1) \phi_2(x_2) \phi_3(x_3) \rangle
	= \frac{\lambda_{123}}
	{(x_{12}^2)^{\Delta_{12,3}}
	(x_{13}^2)^{\Delta_{13,2}}
	(x_{23}^2)^{\Delta_{23,1}}},
	\label{eq:3ptfunction}
\end{equation}
where we have introduced the compact notation
\begin{equation}
	x_{ij}^2 = (x_i - x_j)^2,
\end{equation}
and
\begin{equation}
	\Delta_{ij,k} = \frac{\Delta_i + \Delta_j - \Delta_k}{2}.
\end{equation}

One cannot overemphasize how amazing eq.~\eqref{eq:3ptfunction} is. It should be compared with the most general 3-point function invariant under Poincaré and scale symmetry only: In this case, any term of the form
\begin{equation}
	(x_{12}^2)^\alpha
	(x_{13}^2)^\beta
	(x_{23}^2)^\gamma
\end{equation}
with
\begin{equation}
	\alpha + \beta + \gamma = \frac{\Delta_1 + \Delta_2 + \Delta_3}{2}
\end{equation}
satisfies the symmetry requirement, so that the 3-point function  could take the form
\begin{equation}
	\langle \phi_1(x_1) \phi_2(x_2) \phi_3(x_3) \rangle
	= \sum_i \frac{c_i}
	{(x_{12}^2)^{\alpha_i}
	(x_{13}^2)^{\beta_i}
	(x_{23}^2)^{\gamma_i}},
\end{equation}
with infinitely many free coefficients $c_i$. Instead, the conformal 3-point function \eqref{eq:3ptfunction} is fixed up to a \emph{unique} multiplicative coefficient $\lambda_{123}$.

To give another point of comparison, let us examine a 3-point function that involves a descendant operator. Since the focus is on scalar operators, let us act on the first operator in \eqref{eq:3ptfunction} with $\partial_\mu \partial^\mu$, after which we obtain
\begin{align}
	\langle \partial^2\phi_1(x_1) \phi_2(x_2) \phi_3(x_3) \rangle
	= 4 \lambda_{123} \bigg[ &
	\frac{\left( \Delta_1 - \frac{d-2}{2} \right) \Delta_{12,3}}
	{(x_{12}^2)^{\Delta_{12,3} + 1}
	(x_{13}^2)^{\Delta_{13,2}}
	(x_{23}^2)^{\Delta_{23,1}}}
	\nonumber \\
	& + \frac{\left( \Delta_1 - \frac{d-2}{2} \right) \Delta_{13,2}}
	{(x_{12}^2)^{\Delta_{12,3}}
	(x_{13}^2)^{\Delta_{13,2} + 1}
	(x_{23}^2)^{\Delta_{23,1}}}
	\nonumber \\
	& + \frac{\Delta_{12,3} \Delta_{13,2}}
	{(x_{12}^2)^{\Delta_{12,3} + 1}
	(x_{13}^2)^{\Delta_{13,2} + 1}
	(x_{23}^2)^{\Delta_{23,1} - 1}} \bigg].
	\label{eq:3pt:descendant}
\end{align}
Unlike the correlation function of primary operators, this is now the sum of three terms with distinct powers of the distances $x_{ij}^2$, all of which are individually consistent with Poincaré and scale symmetry.
However, the peculiar form of the coefficients multiplying all three terms implies that there are special cases in which this 3-point function takes the general form of eq.~\eqref{eq:3ptfunction}.
These are not accidental, but correspond to physically interesting cases:
\begin{itemize}

\item
If $\Delta_1 = \left| \Delta_2 - \Delta_3 \right|$, then either $\Delta_{12,3} = 0$ or $\Delta_{13,2} = 0$, and in both cases two terms on the right-hand of eq.~\eqref{eq:3pt:descendant} vanish. This is a very special situation in which the primary 3-point function factorizes into a product of 2-point function, e.g.~when $\Delta_3 = \Delta_1 + \Delta_2$,
\begin{equation}
	\langle \phi_1(x_1) \phi_2(x_2) \phi_3(x_3) \rangle
	\propto \langle \phi_1(x_1) \phi_1(x_1) \rangle
	\langle \phi_2(x_2) \phi_2(x_2) \rangle.
\end{equation}
This situation is realized in generalized free field theory (definition in section~\ref{sec:bootstrap} below) when $\phi_3$ is a composite operator $\phi_3 \approx \phi_1 \phi_2$.

\item
If $\Delta_1 = \frac{d-2}{2}$, then the first two terms in eq.~\eqref{eq:3pt:descendant} vanish. As we saw in section~\ref{sec:quantum}, only a free scalar field satisfying the equation of motion $\partial^2 \phi_1(x) = 0$ can have such a scaling dimension. Since then obviously $\left[ K^\mu, \partial^2 \phi_1(x) \right] = 0$, it is natural that the correlator involving the equation of motion takes the form of a primary 3-point function. However, we also know that it must vanish identically, which implies that either the 3-point coefficient $\lambda_{123}$ vanishes, or that the additional condition $\Delta_1 = \left| \Delta_2 - \Delta_3 \right|$ is satisfied: this is for instance the case of a 3-point function involving the primary composite operator $\phi^2$,
\begin{equation}
	\langle \phi(x_1) \phi(x_2) \phi^2(x_3) \rangle,
\end{equation}
which is non-zero but vanishes under the action of $\partial^2/(\partial x_1)^2$.

\end{itemize}

Finally, let us go back to Minkowski space-time through an analytic continuation in the opposite direction of what we did before. It is not hard to see that the Wightman function of scalar primary operators satisfies
\begin{equation}
	\bra{0} \phi_1(x_1) \phi_2(x_2) \phi_3(x_3) \ket{0}
	= \frac{\lambda_{123}}
	{(x_{12}^2)^{\Delta_{12,3}}
	(x_{13}^2)^{\Delta_{13,2}}
	(x_{23}^2)^{\Delta_{23,1}}},
\end{equation}
where now the Minkowski distances between two points are defined by
\begin{equation}
	x_{ij}^2 = -(x_i^0 - x_j^0 - i \varepsilon)
	+ (\vec{x}_i - \vec{x}_j)^2.
	\label{eq:Minkowskidistance}
\end{equation}
Note that $x_{ij}^2 \neq x_{ji}^2$, and as a consequence the Wightman 3-point function is not symmetric under the exchange of operators.
Nevertheless, for real operators satisfying $\phi_i(x)^\dagger = \phi_i(x)$, we must have
\begin{equation}
	\bra{0} \phi_1(x_1) \phi_2(x_2) \phi_3(x_3) \ket{0}
	= \bra{0} \phi_3(x_3) \phi_2(x_2) \phi_1(x_1) \ket{0}^*,
\end{equation}
which is only compatible with the $i \varepsilon$ prescriptions appearing above if the coefficient $\lambda_{123}$ is real.
This property will in fact be essential in the conformal bootstrap discussed in section~\ref{sec:bootstrap}.


\subsection{4-point functions}

The embedding space technique can used for higher-point functions as well, but we shall see that things get more complicated.
To avoid dealing with four distinct operators and as many scaling dimensions, let us focus our attention to the case of 4 identical scalar operators,
\begin{equation}
	\langle \phi(X_1) \phi(X_2) \phi(X_3) \phi(X_4) \rangle.
\end{equation}
In these case there are 6 Lorentz invariants $X_i \cdot X_j$ with $i \neq j$. It is easy to see that a term of the form
\begin{equation}
	\frac{1}{(X_1 \cdot X_2)^\Delta (X_3 \cdot X_4)^\Delta}
\end{equation}
satisfies all the constraints of conformal symmetry. It is however definitely not unique: 
\begin{equation}
	\frac{1}{(X_1 \cdot X_3)^\Delta (X_2 \cdot X_4)^\Delta}
\end{equation}
does too, and so does a third term with yet a different combination of indices. This shows right away that there is no hope of constraining the 4-point function as much as we did with the 3-point function.
In fact, one can construct two invariant quantities out of the $X_i$:
\begin{equation}
	u = \frac{(X_1 \cdot X_2) (X_3 \cdot X_4)}
	{(X_1 \cdot X_3) (X_2 \cdot X_4)},
	\qquad\qquad
	v = \frac{(X_1 \cdot X_4) (X_2 \cdot X_3)}
	{(X_1 \cdot X_3) (X_2 \cdot X_4)}.
\end{equation}
These are usually called \emph{conformal cross-ratios}. 
Therefore, any function of $u$ and $v$ is conformally invariant, and the most general 4-point function must take the form
\begin{equation}
	\langle \phi(X_1) \phi(X_2) \phi(X_3) \phi(X_4) \rangle
	\propto
	\frac{g(u,v)}{(X_1 \cdot X_2)^\Delta (X_3 \cdot X_4)^\Delta}.
\end{equation}
In terms of Euclidean coordinates, this can be written
\begin{equation}
	\langle \phi(x_1) \phi(x_2) \phi(x_3) \phi(x_4) \rangle
	= \frac{g(u,v)}{\left[ x_{12}^2 x_{34}^2 \right]^\Delta}
	\label{eq:4pt}
\end{equation}
with
\begin{equation}
	u = \frac{x_{12}^2 x_{34}^2}{x_{13}^2 x_{24}^2},
	\qquad\qquad
	v = \frac{x_{14}^2 x_{23}^2}{x_{14}^2 x_{23}^2}.
	\label{eq:crossratios}
\end{equation}
To see that this is the most general result, let us consider the following argument based on a sequence of finite conformal transformations:
\begin{enumerate}

\item
Using translations, it is always possible to choose a reference frame in which $x_1 = 0$.

\item
Using a special conformal transformation, one can then send $x_4 \to \infty$ without moving $x_1$.

\item
A rotation can then be used to place $x_3$ along a last axis, followed by a scale transformation to get $x_3 = (0, \ldots, 0, 1)$, without touching the origin nor the point at infinity.

\item
Finally, there is still a subset of rotations that do not affect $x_3$, those around the last axis. They can be used to move the point $x_2$ to the position $x_2 = (b, 0, \ldots, 0, a)$, shown in figure~\ref{fig:conformalframe:z}.

\end{enumerate}
%
\begin{figure}
	\centering
	%
	\caption{A conformal frame involving 4 points, in which 3 points
	have been mapped to $0$, $(1, 0, \ldots, 0)$ and $\infty$,
	and the fourth point is placed at $(a, b, 0, \ldots 0)$,
	characterized by the complex variable $z = a + i b$.
	}
	\label{fig:conformalframe:z}
\end{figure}
%
Note that the first 3 steps can be used to fix entirely the kinematics of a 3-point function, which explains why its only freedom is in the form of a multiplicative coefficient. Instead, with a 4-point function we are left with two quantities, $a$ and $b$, which correspond to the freedom parameterized by the two conformal cross ratios.
It is in fact convenient to replace these two real numbers with a complex $z = a + i b$ and its conjugate $\bar{z} = a - i b$.
The cross-ratios are then related to $z$ and $\bar{z}$ by
\begin{equation}
	u = z \bar{z},
	\qquad\qquad
	v = (1 - z) (1 - \bar{z}).
\end{equation}
Since Schwinger functions are analytic at all non-coincident configurations of points, the function $g(z)$ is a single-valued function over the complex plane minus the points $\{ 0, 1, \infty \}$. Note that $g$ is also subject to the crossing symmetry of the 4-point function, which requires
\begin{equation}
	g(u, v) =
	g\left( \frac{u}{v}, \frac{1}{v} \right)
\end{equation}
upon exchanging the operators $\phi(x_1)$ and $\phi(x_2)$, as well as
\begin{equation}
	g(u,v) = \left( \frac{u}{v} \right)^\Delta g(v, u),
\end{equation}
upon $\phi(x_1) \leftrightarrow \phi(x_3)$.

The situation is however much more complicated in Minkowski space-time. One can still define cross ratios by eq.~\eqref{eq:crossratios}, with Minkowski distances defined through $i \varepsilon$ prescriptions as in \eqref{eq:Minkowskidistance}. But the Wightman functions are not analytic (they lie at the boundary of the domain of analyticity in complexified time), and therefore $g$ is a multi-valued function. Its values can be obtained by analytic continuation from a configuration in which all 4 operators live on a time slice, in which case in coincides with the Schwinger function. For instance, keeping $x_1$, $x_3$ and $x_4$ on that time slice but letting the time component $b$ of $x_2$ become imaginary, one ends up in a configuration in which $z$ and $\bar{z}$ are both real but distinct, $z \neq \bar{z}$. This procedure is in general tedious, and it is fair to say that the current understanding of conformal Wightman 4-point functions is incomplete.%
%
\footnote{Working in momentum space is not helping, even though the spectral condition reduces the number of non-trivial configurations: since the Ward identity for special conformal transformation is a second-order differential equation, the solution cannot be formulated in terms of invariants. The fact that there exists conformal invariants in position space is intimately connected with Ward identities being first-order differential equations.}


%%%%%%%%%%%%%%%%%%%%%%%%%%%%%%%%%%%%%%%%%%%%%%%%%%%%%%%%%%%%%%%%%%%%%%%

\section{State-operator correspondence and OPE}
\label{sec:OPE}

The construction of correlation functions could be continued beyond 4 points, but there is a good reason to stop there (at least in this course): we shall now see that any $n$-point function can be reduced to a $(n-1)$-point function using the operator product expansion (OPE), and that this procedure can be iterated until everything is expressed in terms of 2- and 3-point functions. The 4-point function will be used as a typical situation in which this OPE can be applied, and higher-point functions will not be considered further.


\subsection{The OPE in quantum field theory}

The operator product expansion in quantum field theory is the statement that when two local operators are ``sufficiently close'' to each other, they can be replaced by another local operator, or a sum of them:
\begin{equation}
	\phi_1(x) \phi_2(y) \xrightarrow{x \to y} \sum_i C_i(x-y) \phi_i(y),
	\label{eq:OPE:operators}
\end{equation}
It does not matter whether the operator $\phi_i$ on the right-hand side is inserted at $x$ or at $y$, or at the middle point $(x+y)/2$, since this is understood in the limit in which $x$ and $y$ coincide. In fact, the proportionality factor $C_i$ might diverge in the limit $x \to y$, so this is in general to be understood in the sense of an asymptotic limit, whose radius of convergence is strictly-speaking zero.

In non-perturbative quantum field theory, the OPE can be formulated in terms of the Hilbert space: when the product of operators in eq.~\eqref{eq:OPE:operators} is acting on the vacuum, then the completeness of the Hilbert space implies that we can write
\begin{equation}
	\phi_1(x) \phi_2(y) \ket{0}
	= \sum_{\ket{\Psi}} \ket{\Psi}
	\bra{\Psi} \phi_1(x) \phi_2(y) \ket{0},
\end{equation}
where the sum is over states $\ket{\Psi}$ forming an orthonormal basis in the Hilbert space. Among these are states obtained acting with a local operator $\phi_i(y)$ on the vacuum, so that we have 
\begin{equation}
	\phi_1(x) \phi_2(y) \ket{0}
	= \sum_{i} C_i(x-y) \phi_i(y) \ket{0} + \ldots,
\end{equation}
where the constant $C_i$ is related to a ratio of Wightman functions
\begin{equation}
	C_i(x-y) \approx \lim_{z \to y}
	\frac{\bra{0} \phi_i(z) \phi_1(x) \phi_2(y) \ket{0}}
	{\bra{0} \phi_i(z) \phi_i(y) \ket{0}}.
\end{equation}
This formulation remains however very heuristic, and it is not obvious how to make it rigorous on general grounds.

In conformal field theory, however, the operator product expansion takes a completely new importance, thanks to the following observations:
%
\begin{itemize}

\item
2-point functions of primary operators are \emph{diagonal} (i.e.~only identical primaries have a 2-point function), which implies that states created by different primaries are readily orthogonal. The norm of primary states is also known in terms of the operators' normalization. Working with momentum eigenstates, which are moreover orthogonal in the momenta, we can therefore write 
\begin{equation}
	\phi_1(x) \phi_2(y) \ket{0}
	= \sum_i \int\limits_{p^0 > \left| \vec{p} \right|} \!\! 
	\frac{d^dp}{(2\pi)^d}
	\widetilde{\phi}_i(p) \ket{0}
	\frac{\bra{0} \widetilde{\phi}_i(-p) \phi_1(x) \phi_2(y) \ket{0}}
	{(-p^2)^{\Delta - d/2}}
	+ \ldots,
	\label{eq:OPE:momentum}
\end{equation}
where we have used the scalar 2-point function in the denominator. States created by local operators that carry spin can be included in the same way after inverting the Lorentz tensor appearing in the 2-point function, which is positive definite by unitarity.

\item
3-point functions are known (see section~\ref{sec:correlators}), which means that the proportionality coefficient $C_i$ are in fact fixed up to an overall multiplicative factor.%
%
\footnote{There is more than one factor if the operators $\phi_1$ and $\phi_2$ carry themselves spin.}
%

\item
There are no other contributions to the OPE beyond those of local operators acting on the vacuum. This property is due to the state/operator correspondence that is discussed next.

\end{itemize}
%

\subsection{The state/operator correspondence}
%\subsection{Radial quantization}

In conformal field theory there exists a one-to-one correspondence between the states on a given time slice and local operators defined by their scaling dimension and representation under the Lorentz group.

The fact that local operators define states is obvious in Minkowski space-time. But it is also true in the analytic continuation to Euclidean space. To see this, remember that a local operator inserted at a Minkowski coordinate $x$ can be expressed using eq.~\eqref{eq:P:exponentiated} as an operator living on the surface $x^0 = 0$ and evolved unitarily,
\begin{equation}
	\phi(x) \ket{0}
	= e^{i x^0 P^0} e^{-i \vec{x} \cdot \vec{P}} \phi(0) \ket{0}.
\end{equation}
Since the spectrum of $P^0$ is non-negative, this can be analytically continued to any value of $x^0$ in the upper-half complex plane; on the contrary, if $x^0$ had a negative imaginary part, then there are states of arbitrarily high energy that would become singular. 
By going to purely imaginary values, $x^0 = i \tau$, we can therefore define a state of the theory through an operator insertion in Euclidean space, provided that this operator is inserted at Euclidean time $\tau > 0$.

This has an immediate interpretation in terms of Euclidean path integral: 
a state on the Euclidean surface $\tau = 0$ (which is \emph{identical} to the original Minkowski time slice $x^0 = 0$) is defined by a path integral over all field configurations restricted to the region $\tau < 0$. This is also valid for any number of operators inserted at separated points of Euclidean time $\tau < 0$, including the case of no operators, corresponding to the vacuum state. But the identification also works the other way: a given state on the surface $\tau = 0$ defines a boundary condition for the path integral at $\tau < 0$, which might correspond to some number of local operators inserted in the bulk (or a superposition of such configurations).

% in this way, any Euclidean correlation function in which the operators are ordered in Euclidean time $\tau$ has a Hilbert space interpretation!



% once we adopt the Euclidean path integral perspective, any quantization is equivalent to another; originally use $P^0$ as Hamiltonian, but we could use anything else, for instance the conformal Hamiltonian $\frac{1}{2} (P^0 \pm K^0)$, or even the radial quantization $D$, related to it by a conformal transformation; takes the Euclidean plane $\tau = 0$ to the sphere of unit radius

% but there is an important difference: $D$ as well as $P^0 \pm K^0$ have fixed-point: evolving back in time, the path integral shrinks to a ball of arbitrarily small radius around origin (or south pole); any state on the unit sphere (or on the plane $\tau = 0$) can be obtained by Hamiltonian evoluation from a state \emph{localized} at the Euclidean origin, and therefore equivalend to a local operator insterted there
% this shows that the state/operator goes both way: a local operator defines a state, but any state also defines a local operator (a linear combination of irreps)

%%%%



%conjugation in Euclidean space?
%takes $x_d \to - x_d$
%
%inversion $r \to 1/r$
% not needed in our approach
%
%generators $P_\mu$ and $K_\mu$ are conjugate under inversion!
%
%
%note: operator inside the unit sphere need not be smeared anymore; they have a well-defined norm!
%
%unitarity bound can be worked out in radial quantization
%
%


\subsection{The conformal OPE}

% the momentum-space formulation \eqref{eq:OPE:momentum} is correct, but it is inconvenient: could be evaluated by Fourier transform in $x$, and $y$, but still inconvenient: on the one hand because the momentum-space 3-point correlation function is complicated, and on the other hand because it obscures the convergence properties in the limit $x \to y$

%derive OPE from radial quantization
%
%relate this to Lorentzian OPE on vacuum
%
%
%use in 3-point function: relates OPE coefficient with 3-pt fct coefficient
%(structure constants of the operator algebra)
%
%exercise: compute function entering the OPE


\subsection{Convergence properties}

%Hilbert-space argument
%
%
%radius? 
%
%absolute convergence as long as one can find a sphere separating the points:
%
%note that a plane is also a sphere with radius infinity
%
%
%%%%
%
%
%concluding remarks:
%
%
%CFT data: spectrum of operators and OPE coefficients
%completely determines the theory!
%
%OPE reduces any higher-point function to 2-point


%%%%%%%%%%%%%%%%%%%%%%%%%%%%%%%%%%%%%%%%%%%%%%%%%%%%%%%%%%%%%%%%%%%%%%%

\section{The conformal bootstrap}
\label{sec:bootstrap}

%any CFT data defines a good theory? no!
%
%different quantization surfaces gives different OPEs in the same theory
%
%OPE associativity
%
%often called ``crossing symmetry'' (note: quite different from crossing symmetry of scattering amplitudes)
%
%(in fact this argument shows that Schwinger functions are symmetric!)
%
%focus on 4-point functions:
%all 4-point functions of all operators contain all the associativity constraints
%
%
%simplifications:
%only 4 identical operators, all scalars


\subsection{Conformal blocks}

%
%diagrammatics (not to be confused with Feynman diags)
%
%
%
%square configuration and expansion in rho coordinates


\subsection{The crossing equation and simple solutions}

%generalized free field theory

%aka mean free theory, Gaussian theory
%
%double-twist operator
%
%corresponds to large-N limit of gauge theory





\subsection{The numerical bootstrap}

%discussion about universality: see Poland, Rychkov, Vichi
%also Shai Chester's notes
%
%
%
%bound on scaling dimensions
%
%also spectrum at the boundary

%
%\subsection{Analytical bootstrap}
%
%Euclidean $z^* = \bar{z}$
%$z = \bar{z} = \frac{1}{2}$ gives bootstrap equations that constrain operators of low scaling dimension $\Delta$
%
%Minkowski: $z \neq \bar{z}$, both real 
%light-cone limit $z \to 0$, $\bar{z}$ fixed is dominated by operators of low twist 
%
%rigorous bound in the limit $\ell \to \infty$
%
%Regge trajectories
%
%
%light-cone limit?


%%%%%%%%%%%%%%%%%%%%%%%%%%%%%%%%%%%%%%%%%%%%%%%%%%%%%%%%%%%%%%%%%%%%%%%
%
%\section{Selected advanced topics}
%
%Other things to fit in:
%
%- Composite operators and anomalies?
%- IR divergences? see di Francesco ex. 4.5
%- cluster decomposition of 4-point function (di Francesco ex. 5.2)
%- Massless regularization of propagator in 2D; Bessel integral form of propagator, with IR and UV divergences
%
%
%
%Virasoro symmetry in 2d:
%
%
%conformal anomalies: anomalies are contact term in the action for the source
%
%
%superconformal bootstrap? note classification of all superconformal algebras
%
%
%\subsection{The 2d conformal bootstrap}
%
%historical presentation
%
%see Rychkov section 4.3.1
%
%advantage 1:
%full Virasoro symmetry
%
%advantage 2:
%easier to deal with spin: left- and right-moving, i.e.~holomorphic and anti-holomorphic
%
%by looking at just the energy-momentum tensor, already get strong constraints on its 2-point function (the central charge)
%
%for $c < 1$, only minimal models possible:
%match Ising, \ldots
%
%finitely many primaries in 2d (still infinitely many quasi-primaries)
%
%note: infinitely many primaries for theories with $c > 1$
%
%also can put the theory on torus and study modular properties of partition function
%
%
%other point to discuss: non-unitarity!
%many possibilities:
%\begin{itemize}
%
%\item
%scaling dimension below unitarity bound
%
%\item
%several operators with same scaling dimension forming a matrix that is not positive-definite (logarithmic CFTs)
%
%\item
%\ldots
%
%\end{itemize}
%
%
%
%\subsection{How to continue from here}
%
%
%\begin{itemize}
%
%\item
%holography: see Joao's TASI lectures
%
%\item
%superconformal: Tajikawa pedestrian lectures
%
%\item
%bootstrap: Shai Chester's lectures
%
%\item
%condensed matter:
%note that unitarity is unnecessary: many non-unitary fixed points, yet they have conformal invariance (why?)
%
%\end{itemize}
%
%
%
%
%
%
%\begin{center}	
%	
%\parbox{11.8cm}{
%\emph{%
%[Courses] are fantastically good for learning physics. The lecturer learns a lot of physics. After my first few studies, just about everything I learned about physics came from teaching it. I don’t know if the students learned a lot, but I certainly did. So I consider teaching physics very important.}
%--- Leonard Susskind \cite{Susskind}}
%
%\end{center}


\bibliography{Bibliography}
\bibliographystyle{utphys}

\end{document}
